%%%%%%%%%%%%%%%%%%%%%%%%%%%%%%%%%%%%%%%%%%%%%%%%%%%%%%%%%%%%
%% filename:	vortrag.tex
%% template:	Mon, 07 May 2012 13:01:14 +0200
%% author:	Hermann Lorenz
%% date:	20. Nov 2012 10:00
%%%%%%%%%%%%%%%%%%%%%%%%%%%%%%%%%%%%%%%%%%%%%%%%%%%%%%%%%%%%
\documentclass[%
	ngerman%
	]{beamer}

%\setbeameroption{show only notes} 	% note{} -- Anmerkungsfolien


%%%%%%%%%%%%%%%%%%%%%%%%%%%%%%%%%%%%%%%%%%%%%%%%%%%%%%%%%%%%
%% Lokalisierung %%%%%%%%%%%%%%%%%%%%%%%%%%%%%%%%%%%%%%%%%%%
%%%%%%%%%%%%%%%%%%%%%%%%%%%%%%%%%%%%%%%%%%%%%%%%%%%%%%%%%%%%
\usepackage[utf8]{inputenc}	% Umlaute direkt eingeben
\usepackage[T1]{fontenc}	% Wörter mit Umlaute umbrechen
\usepackage[ngerman]{babel}	% deutsche Bezeichner
\usepackage[babel,german=guillemets]{csquotes}	% \enquote{}
\usepackage{libertine}
\usepackage[scaled=0.83]{beramono}


%%%%%%%%%%%%%%%%%%%%%%%%%%%%%%%%%%%%%%%%%%%%%%%%%%%%%%%%%%%%
%% Bilder %%%%%%%%%%%%%%%%%%%%%%%%%%%%%%%%%%%%%%%%%%%%%%%%%%
%%%%%%%%%%%%%%%%%%%%%%%%%%%%%%%%%%%%%%%%%%%%%%%%%%%%%%%%%%%%
\usepackage{graphicx}	% \includegraphics{bild.pdf}
\usepackage{tikz}

%-----------------------------------------------------------
% TikZ library: arrows, positioning

\usetikzlibrary{arrows,positioning,shapes.geometric,shapes,backgrounds}
\tikzset{
    %Define standard arrow tip
    >=stealth',
    %Define style for boxes
    punkt/.style={
           rectangle,
           rounded corners,
           draw=black, very thick,
           text width=6.5em,
           minimum height=2em,
           text centered},
    % Define arrow style
    pil/.style={
           ->,
           thick,
           shorten <=2pt,
           shorten >=2pt,}
}
\usetikzlibrary{fit}

%%%%%%%%%%%%%%%%%%%%%%%%%%%%%%%%%%%%%%%%%%%%%%%%%%%%%%%%%%%%
%% pdf-links %%%%%%%%%%%%%%%%%%%%%%%%%%%%%%%%%%%%%%%%%%%%%%%
%%%%%%%%%%%%%%%%%%%%%%%%%%%%%%%%%%%%%%%%%%%%%%%%%%%%%%%%%%%%
\usepackage[ngerman]{varioref}	% \vpageref{}

%%%%%%%%%%%%%%%%%%%%%%%%%%%%%%%%%%%%%%%%%%%%%%%%%%%%%%%%%%%%
%% eigene Macros %%%%%%%%%%%%%%%%%%%%%%%%%%%%%%%%%%%%%%%%%%%
%%%%%%%%%%%%%%%%%%%%%%%%%%%%%%%%%%%%%%%%%%%%%%%%%%%%%%%%%%%%
\usepackage{xspace}

\newcommand{\zB}{z.\,B.\xspace}
\newcommand{\conclusion}{\(\to\)\xspace}

\newcommand{\todo}[1]{\textcolor{red}{TODO: #1}}
\newcommand{\prove}[1]{\textcolor{red}{TODO/PROVE(#1)}}
%\newcommand{\todo}[1]{}
%\newcommand{\prove}[1]{}

\definecolor{lgray}{gray}{.3}
\newcommand{\notall}[1]{\textcolor{lgray}{#1}}
\newcommand{\eg}[1]{\textcolor{lgray}{#1}}

\newcommand{\noteparagraph}[1]{\smallskip \textbf{#1}\,\,}


\newcommand{\myContentOverviewFrame}{%
	\begin{frame}{Agenda}%
		% show only sections, no subsections
		\tableofcontents[hideallsubsections]%
	\end{frame}%
	}

\newcommand{\myContentDiscussionFrame}{%
	\begin{frame}{Diskussion und Vorführung}%
		% show only sections, no subsections
		\tableofcontents[hideallsubsections]%
	\end{frame}%
	}

\usepackage{ifthen}
% use: \myContentSectionFrame
% use: \myContentSectionFrame[1-2]
\newcommand{\myContentSectionFrame}[1][\empty]{%
	\begin{frame}{\insertsection{} -- Übersicht}%
		\ifthenelse{\equal{#1}{\empty}}%
			% show only sections, no subsections
			{\tableofcontents[currentsection,currentsubsection,hideothersubsections]}%
			% show only the given sections, but show only subsections for currentsection
			{\tableofcontents[sections={<#1>},currentsection,currentsubsection,hideothersubsections]}%
	\end{frame}%
	}

\setbeamertemplate{frametitle}{
	\hspace{-1.5em}
	\insertframetitle\\
	\hspace{-.5em}\scriptsize\insertframesubtitle\hfill\insertpart\\[-.9em]
	\rule{\textwidth}{.1pt}
}
\setbeamertemplate{frametitle}{%
	\renewcommand{\arraystretch}{0.5}
	\begin{tabular}{@{}l}
		\hspace{-1.5em}
		\insertframetitle \tabularnewline
		\hspace{-.5em}
		\scriptsize\insertframesubtitle \tabularnewline
	\end{tabular}
	\hfill%
	{\scriptsize\insertpart}\\[-.5em]
	\rule{\textwidth}{.1pt}
}
\setbeamertemplate{footline}{
	\usebeamercolor[fg]{structure}
	\hspace*{.5cm}\raisebox{3pt}{
	\begin{tikzpicture}
		\draw [draw opacity=0.0] (-2pt,0) -- (.85\textwidth + 2pt,0) -- (.85\textwidth + 2pt,-5pt) -- (-2pt,-5pt) -- cycle;
		\draw (0,0) -- (.85\textwidth,0);
		\ifnum\inserttotalframenumber>1
		\if \insertpartstartpage \insertpartendpage
		\else
		\draw [fill,xshift=.85\textwidth / (\insertpartendpage - \insertpartstartpage) * (\insertpagenumber - \insertpartstartpage)] (0,0) -- (2pt,-5pt) -- (-2pt,-5pt) -- cycle;
		\fi
		\fi
	\end{tikzpicture}
	}
	\hfill\raisebox{3pt}{\insertframenumber/\inserttotalframenumber\hspace{3pt}}
}

\definecolor{hllgreen}{rgb}{.2,.7,.2}
\definecolor{hllgreenbg}{rgb}{.9,1,.9}
\definecolor{hllblue}{rgb}{.2,.2,.7}
\definecolor{hllbluebg}{rgb}{.9,.9,1}
\definecolor{hllorange}{rgb}{1,0.482,0}
\definecolor{hllorangebg}{rgb}{1,0.782,.4}
\setbeamertemplate{blocks}[rounded]
\setbeamercolor{structure}{fg=hllblue}
\setbeamercolor{normal text}{fg=black}
\setbeamercolor{alerted text}{fg=hllorange}
\setbeamercolor{block title alerted}{fg=black,bg=hllorange}
\setbeamercolor{block body alerted}{fg=black,bg=hllorangebg}
\setbeamercolor{block title}{fg=white,bg=hllblue}
\setbeamercolor{block body example}{fg=black,bg=hllgreenbg}
\setbeamercolor{block title example}{fg=white,bg=hllgreen}
\setbeamercolor{block body}{fg=black,bg=hllbluebg}

\newcommand{\prosymbol}{%
	\raisebox{-.1\baselineskip}{%
		\begin{tikzpicture}%
			\draw [line width=.1\baselineskip] (-.25\baselineskip,0) -- (.25\baselineskip,0);%
			\draw [line width=.1\baselineskip] (0,-.25\baselineskip) -- (0,.25\baselineskip);%
		\end{tikzpicture}%
	}%
	}
\newcommand{\contrasymbol}{%
	\raisebox{.15\baselineskip}{%
		\begin{tikzpicture}%
			\draw [line width=.1\baselineskip] (-.25\baselineskip,0) -- (.25\baselineskip,0);%
		\end{tikzpicture}%
	}%
	}
\newcommand{\proconsymbol}{%
	\raisebox{-.1\baselineskip}{%
		\begin{tikzpicture}%
			\draw [line width=.05\baselineskip] (-.125\baselineskip,0) -- (-.125\baselineskip,.25\baselineskip);%
			\draw [line width=.05\baselineskip] (-.25\baselineskip,.125\baselineskip) -- (0,.125\baselineskip);%
			\draw [line width=.05\baselineskip] (0,-.125\baselineskip) -- (.25\baselineskip,-.125\baselineskip);%
			\draw [line width=.02\baselineskip] (-.25\baselineskip,-.25\baselineskip) -- (.25\baselineskip,.25\baselineskip);%
		\end{tikzpicture}%
	}%
	}
\definecolor{procolor}{rgb}{0,.8,0}
\definecolor{contracolor}{rgb}{.8,0,0}
\definecolor{proconcolor}{rgb}{0,0,.6}
%         |
%         |
%         |
%   ------+------
%         |
%         |
%         |
%
% h = .5\baselineskip
%
%
%   -------------
% h = .1\baselineskip
%
%
\newenvironment{proconlist}%
	{%
		\begin{list}{?}{}%
		\newcommand{\pro}{\item[\textcolor{procolor}{\prosymbol}]}%
		\newcommand{\contra}{\item[\textcolor{contracolor}{\contrasymbol}]}%
		\newcommand{\procon}{\item[\textcolor{proconcolor}{\proconsymbol}]}%
	}{%
		\end{list}%
	}


\usepackage{listings}
\lstset{basicstyle=\ttfamily\scriptsize,backgroundcolor=\color[rgb]{.9,.9,.9}}
\usepackage{dirtree}

\usepackage{tabularx}
\usepackage{dcolumn}
\usepackage{multirow}
\usepackage{booktabs}
\usepackage{bbding}
\usepackage[squaren]{SIunits}

\newcommand{\theadhll}[1]{\emph{\scriptsize #1}}

%-------------------------------------------------------------------------------
% Inhaltsverzeichnis bei jeder Section
% (funktioniert leider nicht in allen Latex-Versionen einwandfrei)
%\AtBeginSection{%
%	\begin{frame}{Übersicht}%
%		\tableofcontents[currentsection]%
%	\end{frame}%
%}

%-------------------------------------------------------------------------------
% Navigationsleiste ausblenden
\setbeamertemplate{navigation symbols}{}

%-------------------------------------------------------------------------------
% Diesen Abschnitt über \begin{document} lassen, damit die PDF-Informationen
% korrekt gesetzt werden.
\title{%
	Überprüfung der Realisierbarkeit einer offenen
	Hausautomatisierungslösung durch Anwendung bestehender
	Informationstechnologien in einem Sensornetz
	}
\date{31.\,Januar~2012}
\author{Angelos~Drossos \and Hermann~Lorenz
	\and Ulrich~Meckel \and Martin~Doenicke
	\and Thomas~Bettermann \and Robert~Krampe \newline
	\and Marcus~Kupke \and Markus~Fischer
	\and Enrico~Uhlig}
\institute{Hochschule für Technik und Wirtschaft Dresden\\%
			Master Angewandte Informationstechnologien\\%
			Forschungsprojekt Sensornetze\\%
			Prof. Dr. J. Vogt%
}
%-------------------------------------------------------------------------------

\begin{document}
%-------------------------------------------------------------------------------
\begin{frame}[plain]
	\maketitle
\end{frame}

%-------------------------------------------------------------------------------
\myContentOverviewFrame
\section{Einführung}

\begin{frame}{Einführung}{Forschungsprojekt Sensornetze}
		\begin{block}<1->{Was wollen wir machen?}
			\begin{itemize}
			\item<1-> 	Aufstellen eines Heimautomatisierungsservers
			\item<1-> 	Steuerung von Sensorknoten (aktive und passive)
			\item<2-> 	Konzipierung eines Sensornetzes (6LoWPAN)
			\item<2-> 	aber auch Nutzung proprietärer Sensornetze
					\eg{(FS20, Homatic)}
			\item<3-> 	Entwicklung eines \enquote{neuen} Open-Source-Standards
			\item<3-> 	Einbindung vorhandener Technologien
			\end{itemize}
		\end{block}
		\begin{block}<4->{Warum wollen wir dies machen?}
			\begin{itemize}
			\item 	Vermeidung mehrerer Technologien in einem Haushalt
					\eg{(erschwert oder verhindert die Steuerung des Haushalts)}
			\item 	Abhängigkeit von bestimmten Herstellern verringern 
					\eg{(Homatic)}
			\item 	Hersteller können sich spezialisieren 
					\eg{(Server, Sensorknoten)}
			\item 	Ingenieurbüros können sich leichter beteiligen
			\end{itemize}
		\end{block}
\end{frame}


\section[Ansatz]{Ansatz}
\myContentSectionFrame[\thesection - 6]

%----------------------------------------------------------
%----------------------------------------------------------

\subsection[Anforderungen]{Anforderungen an das Hausautomatisierungssystem}

%----------------------------------------------------------

\begin{frame}{\insertsection}{Anforderungen I}
	\begin{block}<+->{Dedizierter Netzknoten zur Regelung und Kontrolle}
		\begin{itemize}
		\item 	für den Hausbewohner leicht konfigurierbar und wartbar
		\item 	Netzknoten befindet sich außerhalb des Sensornetzes
		\end{itemize}
	\end{block}
	\vfill
	\begin{block}<+->{Integration bestehender Sensornetze}
		\begin{itemize}
		\item 	Gewährleistung der Interoperabilität
		\item 	Verringerung der Anschaffungskosten bei vorhandenen Sensornetzen
		\end{itemize}
	\end{block}
\end{frame}

%----------------------------------------------------------

\begin{frame}{\insertsection}{Anforderungen II}
	\begin{block}<+->{Verwendung \enquote{klassischer} Internetprotokolle}
		\begin{itemize}
		\item 	Verwendung auf Vermittlungs-, Transport- und Anwendungsschicht
		\item 	Erhöhung der Transparanz und Annahmebereitschaft der Hausbewohner
		\item 	Einsparung von zusätzlichem Entwicklungsaufwand
				% man betrachte heutzutage die vielen unterschiedlichen Betriebssyteme
				% Mac OS, Windows, Linux Derivate, Android, iOS, BlackBerry, Firefox OS, Chrome OS
		\item 	Analyse klassischer Internetprotokolle\newline
				für den Einsatz in Sensornetzen
				% Batterielebensdauer / Radio Duty Cycle / etc.
		\end{itemize}
	\end{block}
	\vfill
	\begin{block}<+->{Ausfall und Erweiterbarkeit von Netzknoten im Sensornetz}
		\begin{itemize}
		\item 	Wunsch nach Sicherheit (verschlüsselte Übertragung und Authentifizierung)
		\item 	Einfache Installation neuer Sensor-/Aktorknoten
		\item 	Flexibles Routing im Sensornetz
		\end{itemize}
	\end{block}
\end{frame}

%----------------------------------------------------------

\subsection{Verwandte Arbeiten}

%----------------------------------------------------------

	\subsubsection{FHEM}

%----------------------------------------------------------

\begin{frame}{\insertsubsection}{\insertsubsubsection}
	\uncover<1->{\enquote{FHEM ist ein Hausautomations-Server,
		geschrieben von Rudolf Koenig et al. in Perl,
		um FS20-Komponenten und andere Geräte zu steuern.}}
	\uncover<2->{\begin{proconlist}
	\pro 	Dedizierter Netzknoten zur Regelung und Kontrolle
	\pro 	Integration bestehender Sensornetze (FS20, HomeMatic)
	\contra Ausfall und Erweiterbarkeit von Netzknoten:
			\begin{itemize}
			\item 	Regelverarbeitung durch Skriptsprache
			\item 	Komplexe Regeln müssen in Perl implementiert werden
			\end{itemize}
	\contra Verwendung \enquote{klassischer} Internetprotokolle:
			\begin{itemize}
			\item 	es müssen Gateways bereitgestellt werden,
					die beide Netzwerkstacks implementieren
			\item 	keine Transparenz vorhanden
			\item 	Gateways wirken der Interoperabilität entgegen
			\end{itemize}
	\end{proconlist}
	\begin{itemize}
	\item 	Projekt-Seite: \url{http://www.fhemwiki.de/wiki/FHEM}
	\item 	Lizenz: GPLv2
	\end{itemize}}
\end{frame}

%----------------------------------------------------------

	\subsubsection[Hexabus]{Hexabus -- the Kranken of Automation}

%----------------------------------------------------------

\begin{frame}{\insertsubsection}{\insertsubsubsection}
	\uncover<1->{\enquote{An IPv6-based home automation bus.}}
	\uncover<2->{\begin{proconlist}
	\contra 	befindet sich noch im Aufbau
	\contra Kein dedizierter Netzknoten zur Regelung und Kontrolle
			\begin{itemize}
			\item 	Regelverarbeitung verteilt auf verschiedene Knoten
			\item 	erfordert eine Übersicht über alle Knoten beim Anwender
			\end{itemize}
	\contra Keine Integration bestehender Sensornetze
	\pro 	Ausfall und Erweiterbarkeit von Netzknoten durch Routing-Protokoll gelöst
	\pro 	Es werden \enquote{klassische} Internetprotokolle verwendet:
			\begin{itemize}
			\item 	es wird die IPv6-Technologie angewandt: 6LoWPAN
			\item 	Transparenz auf Vermittlungs- und Transportschicht
			\item 	Anwendungsschicht (Hexabus Protokoll): nicht klassisch
			\end{itemize}
	\end{proconlist}
	
	\begin{itemize}
	\item 	Projekt-Seite: \url{https://github.com/mysmartgrid/hexabus/wiki}
	\end{itemize}}
\end{frame}

%----------------------------------------------------------

\subsection{Projekteinteilung}

%----------------------------------------------------------

\begin{frame}{\insertsection}{\insertsubsection}
	\begin{block}{dedizierter Regelungsserver}
		\begin{enumerate}
		\item 	DB zur Verwaltung der Sensoren/Aktoren
		\item 	Integrierung der Anwendungsschicht (CoAP)
		\item 	Regelverarbeitung (inkl. Weboberfläche)
		\end{enumerate}
	\end{block}
	\begin{block}{Gateway zur Integration bestehender Sensornetze}
		\begin{enumerate}
		\item 	FS20
		\item 	HomeMatic
		\end{enumerate}
	\end{block}
	\begin{block}{Eigenes drahtloses Sensornetz (WSN)}
			\begin{enumerate}
			\item 	Border Router
			\item 	Routerknoten
			\item 	Netzknoten mit Sensoren
			\item 	Netzknoten mit Aktoren
			\end{enumerate}
	\end{block}
\end{frame}

%----------------------------------------------------------

%\subsection{Aufgabenstellung}

%----------------------------------------------------------

%\begin{frame}{\insertsubsection}{}
%	\begin{itemize}
%	\item	prüfen, ob mit bestehenden, offenen Technologien der Aufbau
%		einer Heimautomatisierung möglich erscheint
%	\item	als Sensoren und Aktoren sollen Funkmodule genutzt werden
%	\item	Datensicherheit im Konzept von vornherein vorsehen
%	\item	Praxisnähe
%		\begin{itemize}
%		\item	bestehende Technologien kombinieren um Aufwand zu sparen
%		\item	offene Standards verwenden, um anderen die Mitarbeit
%			kostengünstig zu ermöglichen
%		\item	auf Energiesparsamkeit achten, da Knoten mit Batterien
%			laufen
%		\end{itemize}
%	\end{itemize}
%\end{frame}

%-------------------------------------------------------------------------------

% Ausblick
% - Entwicklung eines "neuen" MAC-Protokolls in Contiki, welches dem 6LoWPAN-Standard entspricht
%   und auch energie-effizient arbeitet und einer bestimmten MAC-Protokollgruppe angehört.


\section{Hausautomatisierungslösung}
\myContentSectionFrame[\thesection - 6]

%----------------------------------------------------------

\subsection{Systemarchitekur}{Dedizierter Steuerungsserver}

%----------------------------------------------------------

\begin{frame}[label=netzwerkaufbau]{\insertsubsection}{}
\begin{tikzpicture}[funk/.style={dashed},
	knoten/.style={draw,rectangle,font=\scriptsize},
	pc/.style={draw,rectangle},
	gateway/.style={draw,ellipse,font=\scriptsize},
	protokoll/.style={pos=.5,above,sloped,font=\tiny},
	node distance=2cm]
	%,every node/.style={scale=.5}]

%-------------------------------------------------------------------------------
% IPv6 - Network
%-------------------------------------------------------------------------------
\coordinate (ipv6anfang)	at (0,0);
\coordinate (ipv6mitte)		at (2,0);
\coordinate (ipv6ende)		at (4,0);
\draw [very thick] (ipv6anfang) -- node [protokoll,below] {IPv6} (ipv6mitte) -- (ipv6ende);

\node (internet) [draw, cloud, aspect=2, below of=ipv6mitte,yshift=.5cm,xshift=.5cm]	{Internet};
\draw (internet) -- (2.5,0);
\node (server) [pc,above of=ipv6mitte,yshift=-.8cm]			{Server};
\draw (server) -- (ipv6mitte);

%-------------------------------------------------------------------------------
% 6LoWPAN - Network
%-------------------------------------------------------------------------------
\node (6lpproxy) [gateway,	above of=ipv6ende,yshift=-.8cm]		{Proxy};
\draw (6lpproxy) -- (ipv6ende);

\node (6lpbr) [knoten, above of=6lpproxy,yshift=-.5cm]	{BR};
\draw (6lpbr) -- node [protokoll] {USB} (6lpproxy);

\node (6lpr) [knoten, above right of=6lpbr] {R};
\draw [funk] (6lpr) -- node [protokoll] {6LoWPAN} (6lpbr);

\node (6lpl1) [knoten, right of=6lpr] {L};
\draw [funk] (6lpl1) -- node [protokoll] {6LoWPAN} (6lpr);
\node (6lpl2) [knoten, below right of=6lpbr] {L};
\draw [funk] (6lpl2) -- node [protokoll] {6LoWPAN} (6lpbr);

\begin{pgfonlayer}{background}
	\node[fill=green!20,rounded corners] (background) [fit=(6lpproxy) (6lpbr) (6lpr) (6lpl1) (6lpl2)] {};
	\node [rotate=-90,anchor=south] at (background.east) {6LoWPAN};
\end{pgfonlayer}

%-------------------------------------------------------------------------------
% FS20 - Network
%-------------------------------------------------------------------------------
\node (fs20gw) [gateway,	above of=ipv6anfang,yshift=-.8cm]		{GW};
\draw (fs20gw) -- (ipv6anfang);

\node (fs20br) [knoten, left of=fs20gw, xshift=.3cm]	{BR};
\draw (fs20br) -- node [protokoll] {USB} (fs20gw);

\node (fs20l1) [knoten, above of=fs20br] {L};
\draw [funk] (fs20l1) -- node [protokoll] {FS20} (fs20br);
\node (fs20l2) [knoten, above right of=fs20br] {L};
\draw [funk] (fs20l2) -- node [protokoll] {FS20} (fs20br);

\begin{pgfonlayer}{background}
	\node[fill=red!20,rounded corners] (background) [fit=(fs20gw) (fs20br) (fs20l1) (fs20l2)] {};
	\node [rotate=90,anchor=south] at (background.west) {FS20};
\end{pgfonlayer}

%-------------------------------------------------------------------------------
% HomeMatic - Network
%-------------------------------------------------------------------------------
\node (hmgw) [gateway,	below of=ipv6anfang]		{GW};
\draw (hmgw) -- (ipv6anfang);

\node (hmbr) [knoten, left of=hmgw, xshift=.3cm]	{BR};
\draw (hmbr) -- node [protokoll] {USB} (hmgw);

\node (hml) [knoten, above of=hmbr,yshift=-.8cm] {L};
\draw [funk] (hml) -- node [protokoll] {HM} (hmbr);

\begin{pgfonlayer}{background}
	\node[fill=blue!20,rounded corners] (background) [fit=(hmgw) (hmbr) (hml)] {};
	\node [rotate=90,anchor=south] at (background.west) {HomeMatic};
\end{pgfonlayer}
\end{tikzpicture}

\end{frame}

%----------------------------------------------------------

\subsection[REST]{REST Programmierparadigma im Sensornetz}


\begin{frame}{\insertsubsection}{}
% Warum ist das REST Programmierparadigma sinnvoll im Sensornetz
		\begin{block}<+->{Adressierbarkeit}
			\begin{itemize}
        		\item jeder REST-Dienst eines Servers hat eindeutige Adresse (URI)
        		\item Sensoren und Aktoren haben feste Adresse % unter der sie erreichbar sind
    			\end{itemize}
		\end{block}
		\begin{block}<+->{Zustandslosigkeit}
    			\begin{itemize}
        		\item zustandslose Anfragen
        		\item jeder Request ist in sich geschlossen
        		\item Request enthält alle Informationen für die Verarbeitung
    			\end{itemize}
		\end{block}
		\begin{block}<+->{Operationen}
    			\begin{itemize}
        		\item REST-Server beschreiben ihre Dienste auf Anfrage
        		\item standardisierte Anfragemöglichkeiten\newline
				(GET / PUT / POST / DELETE)
        		\item Sensoren/Aktoren können mit GET abgefragt werden
        		\item Aktoren können mit PUT geschaltet werden
    			\end{itemize}
		\end{block}
\end{frame}



% CoAP : http://datatracker.ietf.org/doc/draft-ietf-core-coap/
\begin{frame}{Constrained Application Protocol (CoAP)}{}
        \begin{proconlist}
            \pro REST Paradigma, aber \enquote{leichtgewichtiger} als HTTP
            \pro einfaches HTTP-Mapping möglich, laut RFC-draft vorgesehen
            \pro optionale Zuverlässigkeitsmechanismen (Confirmable / Non-Confirmable)
            \pro Interoperabilität
            \pro entspricht dem Ansatz \emph{Internet der Dinge}
            \pro geringer Overhead gegenüber
HTTP\footnote{\url{http://www.comnets.uni-bremen.de/itg/itgfg521/aktuelles/fg-workshop-29092011/ITG_HH_thomas_poetsch.pdf}}
            \procon noch in Entwicklung (wird ständig verbessert, jedoch ggf. veraltete Implementierungen)
            \contra größerer Overhead als Eigenentwicklung\newline
			(jedoch mangelnde Interoperabilität)
        \end{proconlist}
\end{frame}



%----------------------------------------------------------

\subsection[Regelwerk]{Regelverarbeitungssystem}

%----------------------------------------------------------
\begin{frame}{\insertsubsection}{Ziel}
	\begin{itemize}
	\item	automatische Ansteuerung von Knoten über vorher festgelegte
		Regeln
		\begin{itemize}
		\item	Festlegung von Bedingungen
		\item	Ausführen einer (oder mehrerer) Aktionen bei Erfüllung
		\item	(evtl. alternative Aktionen bei Nichterfüllung)
		\end{itemize}
	\item	flexible Anpassung dieser Regeln
		\begin{itemize}
		\item	manuell für erfahrene Benutzer
		\item	über einfach zu bedienende (Web-)Oberfläche für
			Endanwender
		\end{itemize}
	\end{itemize}
\end{frame}
%----------------------------------------------------------
\begin{frame}{\insertsubsection}{Aufbau}
	\begin{itemize}
	\item	Beschreibung von Regeln in XML-Datei
	\item	Syntax in DTD definiert
	\item	flexible und eindeutige Kommunikation
	\end{itemize}
	\lstinputlisting[language=xml,tabsize=4]{bsp/regeln.dtd}
\end{frame}
%----------------------------------------------------------
\begin{frame}{\insertsubsection}{Kommunikation}
	\definecolor{gruen}{rgb}{0,.8,0}
\definecolor{orangig}{rgb}{1,.452,0}
\begin{tikzpicture}[>=stealth',pc/.style={rectangle,minimum width=1.5cm, minimum height=5cm,draw},%
	mynode/.style={rectangle,draw, minimum width=1.5cm},
	nachricht/.style={font=\scriptsize,auto}
	]
	\node (client) [pc]		{Client};
	\node (webserver) [pc]		[right of=client, xshift=2.5cm] {\parbox{\widthof{Server}}{Web-Server}};
	\node (pythonserver) [pc]	[right of=webserver, xshift=2.5cm] {\parbox{\widthof{Python-}}{Python-Server}};
	\node (db) [rectangle,draw, minimum height=.8cm, minimum width=2.5cm, below of=webserver,xshift=1.8cm,yshift=-3cm] {Datenbank};

	\node (node1) [mynode,right of=pythonserver,xshift=2cm,yshift=2.25cm]	{Node};
	\node (node2) [mynode,below of=node1]		{Node};
	\node (node3) [mynode,below of=node2]		{Node};

	\draw [<->,shorten >=1mm,shorten <=1mm] (client.north east) -- node [auto] {HTTP} (webserver.north west);
	\draw [<->,shorten >=1mm,shorten <=1mm] (webserver.north east) -- node [auto] {Web-Socket} (pythonserver.north west);
	\draw [<->,shorten >=1mm,shorten <=1mm] (pythonserver.north east) -- node [auto] {6LoWPAN} (node1.north west);

	\draw [<->,shorten >=1mm,shorten <=1mm] (db.north west) -- node [sloped,above,font=\tiny] {SQLite} (webserver);
	\draw [<->,shorten >=1mm,shorten <=1mm] (db.north east) -- node [sloped,above,font=\tiny] {SQLite} (pythonserver);

	\uncover<2->{
		\draw [->,orangig] (client) -- node [nachricht] {Web-Request} (webserver);
		\draw [->,orangig] (webserver) -- node [nachricht] {\parbox{\widthof{Nachricht}}{\centering Socket-Nachricht}} (pythonserver);
		\draw [orangig] (pythonserver.east){}+(0.1,0.1) edge [bend left, ->, shorten >=1mm] node [above, font=\scriptsize] {\parbox{1.5cm}{\centering CoAP-Anfrage}} (node3.west);
	}

	\uncover<3->{
		\draw [gruen] (pythonserver.south east){}+(0,-0.1) edge [bend left,->,shorten >=1mm] node [auto,font=\scriptsize] {Eintrag in DB} (db.east);
		\draw [gruen] (db.west){}+(-0.1,0) edge [bend left,->,shorten >=1mm] node [below,font=\scriptsize,xshift=-.2cm,yshift=-.3cm] {\parbox{2cm}{\centering Auslesen der Sensorwerte aus DB}} (webserver.south west);
		\draw [gruen] (client.south east){}+(0.1,0.1) edge [bend left,->,shorten >=1mm,font=\scriptsize] node [above] {\parbox{1.6cm}{\centering Zyklisches Nachladen der Seite}} (webserver.south west);
		\draw [gruen] (webserver.south west){}+(-0.1,-0.1) edge [bend left,->,shorten >=1mm] (client.south east);
		\draw [gruen] (node3.west){}+(-0.1,-0.1) edge [bend left, ->, shorten >=1mm] node [below, font=\scriptsize] {\parbox{1.5cm}{\centering CoAP-Antwort}} (pythonserver.east);
	}
\end{tikzpicture}

\end{frame}
%----------------------------------------------------------
\begin{frame}{\insertsubsection}{Umsetzung}
	\begin{columns}
	\column{.45\textwidth}
		\begin{block}{zentral}
		\begin{proconlist}
		\pro	Verwaltung an zentraler Stelle
		\pro	einfache Client-Anwendungen
		\pro	weniger Kommunikation im Netz
		\end{proconlist}
		\end{block}
	\column{.45\textwidth}
		\begin{block}{dezentral}
		\begin{proconlist}
		\pro	Ausfall eines Steuerungsknotens tolerierbar
		\pro	Autonome Systeme möglich
		\contra	benötigt Broadcast-Kommunikation
		\end{proconlist}
		\end{block}
	\end{columns}
\end{frame}
%----------------------------------------------------------

\section{Bestehende Sensornetze} % wie FS 20 und HM
\myContentSectionFrame[\thesection - 6]

\againframe{netzwerkaufbau}
%-------------------------------------------------------------------------------

\begin{frame}{\insertsubsection}{Interessante Fragestellungen}
	\begin{itemize}
	\item 	Ist es möglich, bestehende Sensornetztechnologien in unseren Ansatz zu integrieren?
	\item 	Welche Vor- und Nachteile bietet die Integration solcher Technologien?
	\item 	Was bringt der Einsatz von CoAP zum Ansprechen der Fremdkomponenten?
	\end{itemize}
	\vspace{0.5em}
	\begin{center}
		\includegraphics[scale=0.38]{pic/gateway_ueberblick}
	\end{center}
\end{frame}

%-------------------------------------------------------------------------------

\begin{frame}{Gateway in bestehenden Sensornetzen}{Lösungsansätze HomeMatic/FS20}
	\begin{itemize}
	\item 	Reverse-Engineering
		\begin{itemize}
		\item 	Mitlesen von Nachrichten
		\item 	Analyse des HomeMatic-Protokolls
		\item 	Heranziehen alternativer Informationsquellen
		\end{itemize}
	\item Erkenntnisse in Programm umsetzen
		unter Berücksichtigung der Anbindung zum Gateway
\end{itemize}
\end{frame}


%-------------------------------------------------------------------------------

\begin{frame}{\insertsubsection}{Ergebnisse I}
	\begin{itemize}
	\item 	Integration von bestehenden Sensornetztechnologien ist möglich, aber aufwendig
	\end{itemize}
	\vspace{1em}
	Nutzung von bestehenden Sensornetzen:
	\begin{proconlist}
	\pro 	Reduzierung Entwicklungsaufwand der Hardware
	\pro 	Vorteile des jeweiligen Systems nutzbar
	\pro 	Systemübergreifender Einsatz
	\contra Verzicht auf Funktionalität
	\contra hoher Aufwand bei geschlossenen Systemen
	\end{proconlist}
\end{frame}

%-------------------------------------------------------------------------------

\begin{frame}{\insertsubsection}{Ergebnisse II}
	\begin{itemize}
	\item 	Gateway (CoAP $ \leftrightarrow $ bestehende Sensornetze)
	\end{itemize}
	\vspace{1em}
	\begin{proconlist}
	\pro 	standardisierte Schnittstelle für den Zugriff auf die Fremdkomponenten
	\contra für jedes System und Geräteklasse ist Ressource (PUT/GET) zu implementieren
	\end{proconlist}
\end{frame}

\section{6LoWPAN}

\subsection{Routing (RPL)}

\begin{frame}{Routing in 6LoWPAN}{RPL}
		\begin{block}{}
			\begin{itemize}
			\item 	dynamisches Routing-Protokoll
					für energiearme und verlustbehaftete Netzwerke
			\item 	Route-Over-Protokoll
			\item 	Protokoll auf Basis eines Distanzvektor-Algorithmus:
					\begin{itemize}
					\item 	DODAG (Destination Oriented Directed Acyclic Graph)
					\item 	jeder Knoten besitzt einen Rang
					\item 	spezielle Nachrichten zum Austausch der
							Routing-Informationen \eg{(DIO, DIS, DAO)}
					\item 	Zielfunktion berechnet Rang
							und wählt bevorzugten Elternknoten
					\end{itemize}
			\item 	weitere Besonderheiten:
					\begin{itemize}
					\item 	Local und Global-Repair (z.B. bei Ausfall eines Knotens)
					\item 	Verwendung des Trickle Timer Algorithmus
					\end{itemize}
			\item 	Implementation in Contiki: ContikiRPL
					\begin{itemize}
					\item 	keine im Standard definierten Sicherheitsmechanismen implementiert
					\end{itemize}
			\end{itemize}
		\end{block}
\end{frame}

\begin{frame}{Routing in 6LoWPAN}{RPL}
	\begin{figure}
	\centering
	\includegraphics[width=0.5\linewidth]{Dodag}
	\linebreak
	\emph{Ein Beispielnetz}
	\end{figure}
\end{frame}
\subsection{Sensorknoten}

\begin{frame}{Sensorknoten letztes Semester}
	\begin{itemize}
	\item 	Sensor Terminal Board
			mit rcb128rfa1 Modul
	\item 	AVR ATmega128rfa1 Microcontroller
	\item 	Ansteuerung der Sensoren
			ohne Betriebssystem:
			\begin{itemize}
			\item 	Feuchtigkeitssensor
			\item 	Drucksensor
			\item 	Geschwindigkeitssensor
			\item 	Ansteuerung per I2C-Bus
			\end{itemize}
	\item 	Einarbeitung in Contiki:
			\begin{itemize}
			\item 	Wie können die Sensoren sinnvoll implementiert werden?
			\item 	Beachtung der Trennung von
					Core, CPU und Platform
			\item 	Nutzung bereits implementierter Schnittstellen
			\end{itemize}
	\end{itemize}
\end{frame}

\begin{frame}{Sensorknoten dieses Semester}
	\begin{itemize}
	\item 	deRFmega128-Board (Batteriebetrieb)
	\item 	AVR ATmega128rfa1 Microcontroller
	\item 	Verifizierung der I2C-Schnittstelle in Contiki durch neuen Sensor (Lichtsensor)
	\item 	Ansteuerung eines Aktors (Heizungsthermostat):
			\begin{itemize}
			\item 	zwei Boards, die per UART miteinander kommunizieren
			\item 	Software des Heizungsthermostat
					durch Bachelor-Projekt
					bereitgestellt
			\item 	das deRFmega128-Board stellt die Funk-Kommunikation bereit
			\end{itemize}
	\end{itemize}
\end{frame}


\section[Fazit]{Schlussbetrachtungen}
\myContentSectionFrame

%-------------------------------------------------------------------------------

\subsection{Einsatz mehrerer Sensornetze}

%-------------------------------------------------------------------------------

\begin{frame}{\insertsubsection}{}
	\begin{itemize}
	\item 	Nutzer kann vorhandene Geräte einbinden (Kostenminderung)
	\end{itemize}

	\begin{itemize}
	\item 	Jedes Sensornetz kann spezifische Eigenschaften besitzen,
			aber dennoch transparent mit dem Steuerungsserver verbunden sein:
			\begin{enumerate}
			\item 	Energieeffizienz
			\item 	Kommunikationsgeschwindigkeit
			\item 	Authentifizierung (sicherheitskritische Sensorknoten)
			\item 	Einteilung nach logischer oder physikalischer Topologie
			\end{enumerate}
	\end{itemize}
\end{frame}

%-------------------------------------------------------------------------------

%-------------------------------------------------------------------------------

\subsection{Auswertung und Ausblick}

%-------------------------------------------------------------------------------

\begin{frame}{\insertsubsection}{}
	\begin{block}<+->{Constrained Application Protokoll (CoAP)}
		\begin{proconlist}
		\pro 	ist zur Versendung der Sensorwerte geeignet
		\pro 	leichtgewichtig gegenüber HTTP
		\pro 	URIs eignen sich, um logische Strukturen aufzubauen
		\pro 	CoAP-Overhead ist tolerierbar
		\contra CoAP-Proxys zum Cachen von Sensorinformationen nützlich
			(ist aber noch nicht vollständig standardisiert)
		\end{proconlist}
	\end{block}
	\vfill
	\begin{block}<+->{6LoWPAN-Sensornetz}
		\begin{proconlist}
		\pro 	Knoten verbinden sich automatisch mit dem Coordinator
		\contra Authentifizierung (noch) nicht vorhanden
		\contra Verschlüsselung der Nachrichten sollte untersucht werden
		\contra Energieeinsparung kann erhöht werden (MAC/RDC-Protokoll)
		\contra Contiki: Dokumentation in vielen Bereichen dürftig
		\end{proconlist}
	\end{block}
\end{frame}

%-------------------------------------------------------------------------------

\againframe{netzwerkaufbau}

\subsection{Quellen}
%-------------------------------------------------------------------------------
\begin{frame}{Ausgewählte Quellen}{}
	\begin{thebibliography}{contiki12}
	\bibitem[contiki12]{OSforWSN:2011}
		Farooq, O. M. \& Kunz, T.:
		\emph{Operating Systems for Wireless Sensor Networks: A Survey},
		\url{www.mdpi.com/1424-8220/11/6/5900/pdf},
		2011
	\bibitem[contiki12]{ContikiMAC:RDC}
		Adam Dunels:
		\emph{The ContikiMAC Radio Duty Cycling Protocol},
		\url{http://dunkels.com/adam/dunkels11contikimac.pdf},
		Technical Report 2011.
	\bibitem[contiki12]{dunkels06:2006}
		Matthias Kovatsch, Simon Duquennoy, and Adam Dunkels:
		\emph{A Low-power CoAP for Contiki}
		\url{http://dunkels.com/adam/kovatsch11low-power.pdf},
		IEEE IoTech 2011.
	\bibitem[contiki13]{bal}
		Mathilde Durvy, et al.:
		\emph{Making Sensor Networks IPv6 Ready}
		\url{http://dunkels.com/adam/durvy08making.pdf},
		ACM SenSys 2008.
	\end{thebibliography}
\end{frame}
%-------------------------------------------------------------------------------
\begin{frame}{Hinweis zur Lehrveranstaltung \emph{Sensornetze}}{}
	Im Rahmen der Veranstaltung \emph{Sensornetze}\, dieses Semester haben wir
	Vorträge zu den folgenden Themen ausgearbeitet:
	\begin{itemize}
		\item 	Contiki-OS
		\item 	Routing mit RPL
		\item 	CoAP
		\item 	HomeMatic
		\item 	KNX
	\end{itemize}
	\vspace{1em}
	Zugehörige Papers werden noch erscheinen.
\end{frame}

\myContentDiscussionFrame
%-------------------------------------------------------------------------------

\end{document}
