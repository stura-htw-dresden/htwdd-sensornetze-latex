% if have a single appendix:
%\appendix[Proof of the Zonklar Equations]
% or
%\appendix  % for no appendix heading
% do not use \section anymore after \appendix, only \section*
% is possibly needed

% use appendices with more than one appendix
% then use \section to start each appendix
% you must declare a \section before using any
% \subsection or using \label (\appendices by itself
% starts a section numbered zero.)
%

%\appendices
\appendix[Vorlesung Sensornetze]

%===========================================================
% ....
%===========================================================

%\section{Vorlesung Sensornetze}

	In der Vorlesung \emph{Sensornetze}
	wurden Themen behandelt, die gewinnbringend in das Forschungsprojekt
	einfließen konnten.
	Folgende Themen waren für das Forschungsprojekt besonders interessant:
	\begin{itemize}
	\item 	Contiki-OS: Implementationsaspekte und Netzwerk-Stack
	\item 	RPL (Routing Protocol Layer)
	\item 	CoAP
	\item 	HomeMatic
	\item 	KNX
	\end{itemize}
	Zu diesen Themen wurden Vorträge gehalten und es wurden auch
	Papers geschrieben, die über Prof. Vogt mit Zustimmung der jeweiligen
	Authoren eingesehen werden können.


%===========================================================
% ....
%===========================================================

% you can choose not to have a title for an appendix
% if you want by leaving the argument blank
%\section{}

%	Appendix two text goes here.

%===========================================================
% Acknowledgements
%===========================================================

% use section* for acknowledgement
\ifCLASSOPTIONcompsoc
  % The Computer Society usually uses the plural form
  %\section*{Acknowledgments}
  \section*{Danksagungen}
\else
  % regular IEEE prefers the singular form
  %\section*{Acknowledgment}
  \section*{Danksagung}
\fi

	Die Autoren wollen Prof. Dr.-Ing. Jörg Vogt für das Anbieten und Betreuen des
	Forschungsprojektes \emph{Sensornetze} sowie den anderen Teilnehmern
	für ihre Mitarbeit danken.
	Es ist ein sehr interessantes Forschungsgebiet und wir wünschen
	den noch verbleibenden sowie den neu hinzukommenden Teilnehmern
	viel Spaß bei der Fortführung dieses Forschungsprojektes.
	Für Fragen stehen wir gerne zur Verfügung.

