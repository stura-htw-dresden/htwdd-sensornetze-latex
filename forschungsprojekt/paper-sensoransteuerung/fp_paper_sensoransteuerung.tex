%%%%%%%%%%%%%%%%%%%%%%%%%%%%%%%%%%%%%%%%%%%%%%%%%%%%%%%%%%%%
%% filename:	contiki_paper.tex
%% template:	Mon, 07 May 2012 13:01:14 +0200
%% author:	Hermann Lorenz
%% date:	26. Nov 2012 15:10
%%%%%%%%%%%%%%%%%%%%%%%%%%%%%%%%%%%%%%%%%%%%%%%%%%%%%%%%%%%%
\documentclass[%
	journal, 	% document type (journal, technote, conference,
			% 	peerreview, peerreviewca)
	a4paper, 	% A4 Paper
	%10pt, 		% Schriftgröße (9pt - 12pt)
	final, 		% doc. mode (draft, draftcls, draftclsnofoot, final)
	%oneside, 	% oneside, twoside
	%twocolumn, 	% onecolumn, twocolumn
	%compsoc, 	% IEEEtran mimics the format of the publications
			% 	of the IEEE Computer Society
	]{IEEEtran}
% http://www.ctan.org/tex-archive/macros/latex/contrib/IEEEtran
% Dort gibt es Latex-Beispiele

%%%%%%%%%%%%%%%%%%%%%%%%%%%%%%%%%%%%%%%%%%%%%%%%%%%%%%%%%%%%
%% Lokalisierung %%%%%%%%%%%%%%%%%%%%%%%%%%%%%%%%%%%%%%%%%%%
%%%%%%%%%%%%%%%%%%%%%%%%%%%%%%%%%%%%%%%%%%%%%%%%%%%%%%%%%%%%
\usepackage[utf8]{inputenc}	% Umlaute direkt eingeben
\usepackage[T1]{fontenc}	% Wörter mit Umlaute umbre-
				% chen
\usepackage[ngerman]{babel}		% deutsche Bezeichner
\usepackage[babel,german=quotes]{csquotes}	% \enquote{}
%\usepackage{libertine}
\usepackage[scaled=0.83]{beramono}
\usepackage{microtype}

%%%%%%%%%%%%%%%%%%%%%%%%%%%%%%%%%%%%%%%%%%%%%%%%%%%%%%%%%%%%
%% IEEEtran tools %%%%%%%%%%%%%%%%%%%%%%%%%%%%%%%%%%%%%%%%%%
%%%%%%%%%%%%%%%%%%%%%%%%%%%%%%%%%%%%%%%%%%%%%%%%%%%%%%%%%%%%
%\usepackage{IEEEtrantools}

%%%%%%%%%%%%%%%%%%%%%%%%%%%%%%%%%%%%%%%%%%%%%%%%%%%%%%%%%%%%
%% Tabellen %%%%%%%%%%%%%%%%%%%%%%%%%%%%%%%%%%%%%%%%%%%%%%%%
%%%%%%%%%%%%%%%%%%%%%%%%%%%%%%%%%%%%%%%%%%%%%%%%%%%%%%%%%%%%
\usepackage{tabularx}
\usepackage{booktabs}	% \toprule\midrule\bottomrule
			% \addlinespace
\usepackage{threeparttable} % Tablenotes
\usepackage{rccol}

%%%%%%%%%%%%%%%%%%%%%%%%%%%%%%%%%%%%%%%%%%%%%%%%%%%%%%%%%%%%
%% Bilder %%%%%%%%%%%%%%%%%%%%%%%%%%%%%%%%%%%%%%%%%%%%%%%%%%
%%%%%%%%%%%%%%%%%%%%%%%%%%%%%%%%%%%%%%%%%%%%%%%%%%%%%%%%%%%%
\usepackage{graphicx}	% \includegraphics{bild.pdf}
\usepackage{rotating}
\usepackage{tikz}
\usetikzlibrary{arrows}
\usetikzlibrary{shapes}
\usetikzlibrary{fit}
\usetikzlibrary{backgrounds}
\usetikzlibrary{datavisualization}
\usepackage{circuitikz}

% Paket zur Anpassung von Titeln von Gleitobjekten
%\usepackage[figureposition=bottom,tableposition=above]{caption}
% Tabelle 1: ABC
%            XYZ
%\captionsetup{format=hang}


% Paket, um Grafiken / Tabellen zu gruppieren (subfigure is obsolete)
% ---> Workaround
% The problem is that there does not seem to be a way to prevent subcaption
% from taking control of the main caption formatting away from IEEEtran like
% the caption=false option does under subfig.sty. IEEEtran has to format
% captions differently depending on its mode.
%\makeatletter
%\let\MYcaption\@makecaption
%\makeatother
%\usepackage[font=footnotesize]{subcaption}
%\makeatletter
%\let\@makecaption\MYcaption
%\makeatother
% see fig/subfigure-example.tex

%%%%%%%%%%%%%%%%%%%%%%%%%%%%%%%%%%%%%%%%%%%%%%%%%%%%%%%%%%%%
%% Farben %%%%%%%%%%%%%%%%%%%%%%%%%%%%%%%%%%%%%%%%%%%%%%%%%%
%%%%%%%%%%%%%%%%%%%%%%%%%%%%%%%%%%%%%%%%%%%%%%%%%%%%%%%%%%%%
\usepackage{xcolor}
% hard colors
\definecolor{hllblue}{rgb}{.2,.2,.7}
% symbolic colors
\definecolor{todocolor}{named}{red}
\definecolor{linkcolor}{named}{hllblue}
\definecolor{lstbg}{gray}{.9}

%%%%%%%%%%%%%%%%%%%%%%%%%%%%%%%%%%%%%%%%%%%%%%%%%%%%%%%%%%%%
%% Mathematische Symbole %%%%%%%%%%%%%%%%%%%%%%%%%%%%%%%%%%%
%%%%%%%%%%%%%%%%%%%%%%%%%%%%%%%%%%%%%%%%%%%%%%%%%%%%%%%%%%%%
\usepackage{amssymb}
\usepackage{amsmath}
\usepackage{amsfonts}

%%%%%%%%%%%%%%%%%%%%%%%%%%%%%%%%%%%%%%%%%%%%%%%%%%%%%%%%%%%%
%% Sonstige Symbole %%%%%%%%%%%%%%%%%%%%%%%%%%%%%%%%%%%%%%%%
%%%%%%%%%%%%%%%%%%%%%%%%%%%%%%%%%%%%%%%%%%%%%%%%%%%%%%%%%%%%
\usepackage{eurosym}
\usepackage{xspace}
\usepackage{textcomp}
\usepackage[italian,	% \unita, conflict with babel
	squaren,	% \squaren, conflict with amssymb
	binary		% \byte usw.
	]{SIunits}

%%%%%%%%%%%%%%%%%%%%%%%%%%%%%%%%%%%%%%%%%%%%%%%%%%%%%%%%%%%%
%% Quellcodes %%%%%%%%%%%%%%%%%%%%%%%%%%%%%%%%%%%%%%%%%%%%%%
%%%%%%%%%%%%%%%%%%%%%%%%%%%%%%%%%%%%%%%%%%%%%%%%%%%%%%%%%%%%
\usepackage{listings}
%\usepackage{scrhack}	% Warnungen im Zusammenhang mit
			% listings verhindern, zusätzlich
			% für Listings das Babelpaket
			% aktivieren, usw.
\lstdefinestyle{nummeriert}{numbers=left}
\lstdefinestyle{monospace}{basicstyle=\ttfamily\scriptsize}
\lstdefinestyle{block}{style=monospace,
	emphstyle=\bfseries,
	breaklines=true,
	breakatwhitespace=false,
	prebreak=\raisebox{0ex}[0ex][0ex]{\ensuremath{\hookleftarrow}}
	}
\lstdefinestyle{float}{style=block,float}
%\lstdefinestyle{block}{style=nummeriert,%
	%style=monospace,%
	%backgroundcolor=lstbg}
\lstset{basicstyle=\ttfamily}

%%%%%%%%%%%%%%%%%%%%%%%%%%%%%%%%%%%%%%%%%%%%%%%%%%%%%%%%%%%%
%% pdf-links %%%%%%%%%%%%%%%%%%%%%%%%%%%%%%%%%%%%%%%%%%%%%%%
%%%%%%%%%%%%%%%%%%%%%%%%%%%%%%%%%%%%%%%%%%%%%%%%%%%%%%%%%%%%
\usepackage{varioref}	% \vpageref{}
\usepackage[%
	pdftex,	% in Links Umbrüche erlauben
	bookmarks,
	bookmarksopen=false
	]{hyperref}
\hypersetup{
	pdfauthor={Angelos Drossos, Hermann Lorenz},
	pdftitle={Ansteuerung der Sensoren und Aktoren in einem 6LoWPAN-Sensornetz}
	}	% \autoref{}
\hypersetup{%
	%ocgcolorlinks,	% beim Drucken die Linkfarben ignorieren
			% kann unter Umständen Probleme bereiten
			% Links können nicht umgebrochen werden
	colorlinks,%
	linkcolor=linkcolor,%
	urlcolor=linkcolor,%
	linktoc=all	% in Verzeichnissen Zahlen und Text verlinken
	}


%%%%%%%%%%%%%%%%%%%%%%%%%%%%%%%%%%%%%%%%%%%%%%%%%%%%%%%%%%%%
%% Fortsetzung Symbole %%%%%%%%%%%%%%%%%%%%%%%%%%%%%%%%%%%%%
%%%%%%%%%%%%%%%%%%%%%%%%%%%%%%%%%%%%%%%%%%%%%%%%%%%%%%%%%%%%
%\usepackage{ellipsis}	% solve \dots problems


%%%%%%%%%%%%%%%%%%%%%%%%%%%%%%%%%%%%%%%%%%%%%%%%%%%%%%%%%%%%
%% Literaturverzeichnis %%%%%%%%%%%%%%%%%%%%%%%%%%%%%%%%%%%%
%%%%%%%%%%%%%%%%%%%%%%%%%%%%%%%%%%%%%%%%%%%%%%%%%%%%%%%%%%%%
% recommend:
%\usepackage[ngerman]{babel}
%\usepackage[babel,german=quotes]{csquotes}

% Bibliographien erstellen mit biblatex (Teil 1)
% see: http://biblatex.dominik-wassenhoven.de/download/DTK-2_2008-biblatex-Teil1.pdf

\usepackage[%
	backend=biber,    % (bibtex, biber)
	%bibencoding=utf8, % wenn .bib in utf8, sonst ascii
	%safeinputenc,     % Das inputenc Package hat nur einen begrenzten UTF8-Satz
	%bibwarn=true,     % Warnung bei fehlerhafter bibDatei
	%sortlocale=de,    % Deutsche Sortierung aktivieren
	%style=alphabetic, % Zitierstil [alphabetic, numeric-comp, etc.]
	style=ieee,       % Zitierstil IEEE, ab biblatex-1.2c
	%style=ieee-alphabetic,       % Zitierstil IEEE, ab biblatex-1.2c
	%isbn=false,                % ISBN nicht anzeigen, gleiches geht
	%                           %    mit nahezu allen anderen Feldern
	%pagetracker=true,          % ebd. bei wiederholten Angaben
	%                           %    (false=ausgeschaltet, page=Seite,
	%                           %    spread=Doppelseite, true=automatisch)
	%maxbibnames=50,            % maximale Namen, die im Literaturverzeichnis
	%                           %    angezeigt werden (ich wollte alle)
	%maxcitenames=3,            % maximale Namen, die im Text angezeigt werden,
	%                           %    ab 4 wird u.a. nach den ersten Autor angezeigt
	%autocite=inline,           % regelt Aussehen für \autocite (inline=\parancite)
	%block=space,               % kleiner horizontaler Platz zwischen den Feldern
	%backref=true,              % Seiten anzeigen, auf denen die Referenz vorkommt
	%backrefstyle=three+,       % fasst Seiten zusammen, z.B. S. 2f, 6ff, 7-10
	%date=short,                % Datumsformat
]{biblatex}

% Abstände
%\setlength{\bibitemsep}{1em}   % Abstand zwischen den Literaturangaben
%\setlength{\bibhang}{2em}      % Einzug nach jeweils erster Zeile

% Bibtex-Datei wird schon in der Preambel eingebunden
\addbibresource{literature.bib} % Biber
%\bibliography{liteature}        % BibTex

% bibtex:
% \usepackage{cite}

% BibLatex/Biber Commands aufheben (weil Biber nicht installiert ist)
%\usepackage{xargs}
%\renewcommandx{\cite}[3][1=, 2=]{\textcolor{red}{#1 [CITE(#3)] #2}}
%\renewcommandx{\Cite}[3][1=, 2=]{\textcolor{red}{#1 [CITE(#3)] #2}}
%\renewcommandx{\autocite}[3][1=, 2=]{\textcolor{red}{#1 [AUTOCITE(#3)] #2}}
%\renewcommandx{\Autocite}[3][1=, 2=]{\textcolor{red}{#1 [AUTOCITE(#3)] #2}}
%\renewcommandx{\parencite}[3][1=, 2=]{\textcolor{red}{(#1 [PARENCITE(#3)] #2)}}
%\renewcommandx{\Parencite}[3][1=, 2=]{\textcolor{red}{(#1 [PARENCITE(#3)] #2)}}
%\renewcommandx{\citetitle}[3][1=, 2=]{\textcolor{red}{#1 CITETITLE(#3) #2}}

%%%%%%%%%%%%%%%%%%%%%%%%%%%%%%%%%%%%%%%%%%%%%%%%%%%%%%%%%%%%
%% Abkürzungsverzeichnis %%%%%%%%%%%%%%%%%%%%%%%%%%%%%%%%%%%
%%%%%%%%%%%%%%%%%%%%%%%%%%%%%%%%%%%%%%%%%%%%%%%%%%%%%%%%%%%%

%\usepackage[% Optionen
	%footnote, % die Langform als Fußnote ausgeben
	%nohyperlinks, % Wenn hyperref geladen ist, wird die Verlinkung unterbunden
%	printonlyused, % nur Abkürzungen auflisten, die tatsächlich verwendet werden
	%	withpage, % im printonlyused-Modus: Seitenzahl der ersten Verwendung angeben
	%smaller, % Text soll kleiner erscheinen (Package relsize vorausgesetzt)
	%dua, % es wird immer die Langform ausgegeben
	%nolist, % es wird keine Liste mit allen Abkürzungen ausgegeben
%	]{acronym}

%%%%%%%%%%%%%%%%%%%%%%%%%%%%%%%%%%%%%%%%%%%%%%%%%%%%%%%%%%%%
%% eigene Macros %%%%%%%%%%%%%%%%%%%%%%%%%%%%%%%%%%%%%%%%%%%
%%%%%%%%%%%%%%%%%%%%%%%%%%%%%%%%%%%%%%%%%%%%%%%%%%%%%%%%%%%%
\newcommand{\vautoref}[1]{\autoref{#1}\vpageref{#1}}
\newcommand{\todo}[1]{%
	\marginpar{\footnotesize\textcolor{todocolor}{TODO}}%
	\textcolor{todocolor}{TODO: #1}%
	}
\newcommand{\todox}[1]{%
	\marginpar{\footnotesize\textcolor{todocolor}{#1}}%
	\textcolor{todocolor}{#1}%
	}
\newcommand{\zB}{z.\,B.\xspace}
\newcommand{\dhx}{d.\,h.\xspace}
\newcommand{\idR}{i.\,d.\,R.\xspace}
\newcommand{\iA}{i.\,A.\xspace}
\newcommand{\uU}{u.\,U.\xspace}
\newcommand{\ItC}{I\textsuperscript{2}C\xspace}

% use: Hallo (\engl{hello})
\newcommand{\engl}[1]{engl.\,\emph{#1}\,\xspace}

% \contikifile{path/to/file}{version}
% use: \contikifile{core/dev/radio.h}{2.6}
\newcommand{\contikifile}[2]{\emph{#1}}

\newcommand*{\changefont}[3]{\fontfamily{#1}\fontseries{#2}\fontshape{#3}\selectfont}
\newcommand{\func}[1]{{\changefont{cmtt}{m}{sc}#1}}
\newcommand{\event}[1]{{\changefont{cmtt}{m}{sc}#1}}

\newcommand{\theadhll}[1]{\emph{\scriptsize #1}}

\newcommand{\tabref}[1]{Tabelle~\ref{tab:#1}}
\newcommand{\figref}[1]{Abbildung~\ref{fig:#1}}
\newcommand{\secref}[1]{Kapitel~\ref{sec:#1}}


\newcommand{\siehe}[1]{siehe \autoref{#1}}
\newcommand{\parensiehe}[1]{(\siehe{#1})}

\newcommand{\ieeeframe}{IEEE~802.15.4}
\newcommand{\thead}[1]{\itshape\scriptsize#1}

%%%%%%%%%%%%%%%%%%%%%%%%%%%%%%%%%%%%%%%%%%%%%%%%%%%%%%%%%%%%
%% PDF Bookmarks %%%%%%%%%%%%%%%%%%%%%%%%%%%%%%%%%%%%%%%%%%%
%%%%%%%%%%%%%%%%%%%%%%%%%%%%%%%%%%%%%%%%%%%%%%%%%%%%%%%%%%%%
\makeatletter
\usepackage{etoolbox}
\pretocmd{\tableofcontents}{%
	\if@openright\cleardoublepage\else\clearpage\fi
	\pdfbookmark[0]{\contentsname}{toc}%
}{}{}%
\makeatother

% Trennmuster
\hyphenation{IEEE Be-triebs-span-nung}


\makeatletter

%% Workaround: Error message
%%      Package babel Error: You haven't defined the language USENGLISH yet.
%% or something similar
\def\markboth#1#2{%
	\def\leftmark{\@IEEEcompsoconly{\sffamily}#1}%
	\def\rightmark{\@IEEEcompsoconly{\sffamily}#2}}

\makeatother

\newcommand{\projecttitle}{\emph{%
	Überprüfung der Realisierbarkeit einer offenen
	Hausautomatisierungslösung durch Anwendung bestehender
	Informationstechnologien in einem Sensornetz}}


\begin{document}
	% IEEE Title and authors
	\title{Ansteuerung von Sensoren und Aktoren in der Hausautomatisierung}
	\author{%
		Angelos~Drossos %
		und~Hermann~Lorenz%
		}

	% IEEE Paper Headers: der 2. Parameter wird nur in zweiseitigen
	% Dokumenten benötigt
	\markboth{Hochschule für Technik und Wirtschaft Dresden, Master
	Angewandte Informationstechnologien, Forschungsprojekt Sensornetze,
	WS 2012/13}%
	{A. Drossos \MakeLowercase{und} H. Lorenz: Ansteuerung von Sensoren und
	Aktoren in der Hausautomatisierung}

	%%%%%%%%%%%%%%%%%%%%%%%%%%%%%%%%%%%%%%%%%%%%%%%%%%%%
	%\frontmatter
	%%%%%%%%%%%%%%%%%%%%%%%%%%%%%%%%%%%%%%%%%%%%%%%%%%%%
	%%%%%%%%%%%%%%%%%%%%%%%%%%%%%%%%%%%%%%%%%%%%%%%%%%%%%%%%%%%%%
%% Titelseite %%%%%%%%%%%%%%%%%%%%%%%%%%%%%%%%%%%%%%%%%%%%%%
%%%%%%%%%%%%%%%%%%%%%%%%%%%%%%%%%%%%%%%%%%%%%%%%%%%%%%%%%%%%
\begin{titlepage}
	\sffamily
	\makebox[5em][r]{
		\color[gray]{.5}
		\begin{sideways}
			\makebox[\textheight][r]{
			\Huge\sffamily\itshape\bfseries Belegarbeit
			}
		\end{sideways}
		\mbox{\rule{2pt}{\textheight}}
	}
	\parbox[b][\textheight]{\dimexpr\textwidth-5em}{
		\raggedleft
		\vspace{2em}
		{\Huge Einführung in Contiki-OS}\par
		\vspace{1em}
		{\Large\itshape Implementierungsaspekte und Netzwerk-Stack}\par
		\vspace{2em}
		{Angelos Drossos\linebreak und\linebreak Hermann Lorenz}\par

		\vfill

		Belegarbeit im Modul\linebreak
		{\scshape\Large Sensornetze}

		\vspace{1cm}

		des Masterstudienganges\linebreak
		{\scshape\Large Angewandte Informationstechnologien}

		\vspace{1cm}

		betreut durch Prof.\,Dr.-Ing.\,Jörg Vogt

		im Wintersemester 2012/2013

		\vfill

		Hochschule für Technik und Wirtschaft Dresden\linebreak
		Fakultät Informatik/Mathematik
	}
\end{titlepage}

	
	% Abstract (IEEE-compsoc-mode):
	% Der Abstract befindet sich über dem zweispaltigen Text.
	\IEEEtitleabstractindextext{\begin{abstract}
Eine (offene) Hausautomatisierung erfordert Sensoren und Aktoren,
die von einer Steuerungseinheit angesteuert werden. Daher behandelt dieses
Dokument die Ansteuerung der Sensoren und Aktoren in einem Sensornetz.
Ziel ist es CoAP-Anfragen auf einem ATmega128RFA1 mit dem Betriebssystem
Contiki über 6LoWPAN zu beantworten.
\end{abstract}

\begin{IEEEkeywords}
freie Hausautomatisierung,
Drahtloses Sensornetz,
Internet der Dinge,
6LoWPAN,
CoAP,
Contiki-OS,
ATmega128RFA1.
\end{IEEEkeywords}
}

	\maketitle

	% display non tilte abstract index text
	% Dieser Befehl versetzt den Abstract abhängig vom IEEE-mode an die
	% richtige Stelle
	\IEEEdisplaynontitleabstractindextext

	%\section*{\pdfbookmark[1]{Abstract}{abstract}Abstract}
	%\input{abstract_en}
	%\vfill
	%\section*{\pdfbookmark[1]{Zusammenfassung}{zusammenfassung}Zusammenfassung}
	%\input{abstract_de}
	%\vfill
	%\thispagestyle{empty}

	%\bgroup
	%	\clearpage
	%	\renewcommand{\clearpage}{\relax}
		% Inhaltsverzeichnis
	%	\tableofcontents

		% Abbildungsverzeichnis
	%	\listoffigures

		% Tabellenverzeichnis
	%	\listoftables
	%\egroup

	% Abkürzungsverzeichnis
	%\chapter*{\pdfbookmark[0]{Abkürzungsverzeichnis}{abkuerzungsverzeichnis}Abkürzungsverzeichnis}

% use:
% \acs{Kurzform} -- fügt Kurzform ein, mit Verlinkung
% \acl{Kurzform} -- fügt Langform ein, ohne Verlinkung
% \acf{Kurzform} -- fügt Langform (Kurzform) ein, NetStack ist verlinkt
% \ac{Kurzform}  -- fügt \acs{Kurzform} ein,
%                   außer beim ersten Aufruf, hier wird \acf{Kurzform} verwendet

% Definition: \acro{Kurzform}[Kurzform mit Math]{Langform}
\begin{acronym}[6LoWPAN]
	% Abstand zwischen zwei Einträgen
	%\setlength{\itemsep}{-\parsep}
	% Schrift ändern
	%\renewcommand{\bflabel}[1]{\normalfont{\normalsize{#1}}\hfill}

	% Internet / Netzwerk
	% Mikrocontroller
	\acro{6LoWPAN}{IPv6 over Low power Wireless Personal Area Networks}
	\acro{CCA}{Clear Channel Assessment}
	\acro{CDMA}{Code Division Multiple Access}
	\acro{CoAP}{Constrained Application Protocol}
	\acro{CoAP}{Constrained Application Protocol}
	\acro{CRC}{Cyclic Redundancy Code}
	\acro{CSMA/CA}{Carrier Sense Multiple Access with Collision Avoidance}
	\acro{CSMA}{Carrier Sense Multiple Access}
	\acro{CSMA/CD}{Carrier Sense Multiple Access with Collision Detection}
	\acro{CTDMA}{Code Time Division Multiple Access}
	\acro{FCS}{Frame Check Sequence}
	\acro{FFD}{Full Function Device}
	\acro{IPC}{Inter-Prozess Communication}
	\acro{IP}{Internet Protocol}
	\acro{ISR}{Interrupt Service Routine}
	\acroplural{ISR}[ISRs]{Interrupt Service Routinen}
	\acro{LLC}{Logical Link Control}
	\acro{LPP}{Low Power Probing}
	\acro{MAC}{Medium Access Control}
	\acro{NetStack}{Netzwerk-Stack}
	\acro{RDC}{Radio Duty Cycle}
	\acro{RFD}{Reduced Function Device}
	\acro{SP}{Sensornet Protocol}
	\acro{TCP}{Transmission Control Protocol}
	\acro{TDMA}{Time Division Multiple Access}
	\acro{UDP}{User Datagram Protocol}
	\acro{uIP}{micro IP, open-source TCP/IP-Stack}
\end{acronym}


	%%%%%%%%%%%%%%%%%%%%%%%%%%%%%%%%%%%%%%%%%%%%%%%%%%%%
	%\mainmatter
	%%%%%%%%%%%%%%%%%%%%%%%%%%%%%%%%%%%%%%%%%%%%%%%%%%%%

	% Gliederung

	%\section{dafür suchen wir noch einen platz}
	%In einer Hausautomatisierungslösung ist das Auslesen von
	%Sensorinformationen und die Reaktion auf diese Informationen mit Hilfe
	%von Aktoren nötig.  Diese Sensoren und Aktoren bilden üblicherweise
	%Sensornetze.  Die Steuerungseinheit der Hausautomatisierung muss sich
	%hierbei nicht im Sensornetz befinden, doch muss sie mit den Netzknoten
	%des Sensornetzes kommunizieren können.

	%Eine freie Hausautomatisierung hat die Förderung der
	%\emph{Interoperabilität} zum Ziel. Daher ist eine Netzwerktechnologie
	%wichtig, die standardisierte Protokolle zugrunde legt.
	%Weitere Ziele sind geringe Installationskosten, ein überschaubarer
	%Wartungsaufwand sowie eine einfach konfigurierbare Regelverarbeitung.


	\section{Einführung}
%\subsection{Themengebiet}
	Die Hausautomatisierung ist ein Teilgebiet der Gebäudeautomatisierung.
	Ein primäres Ziel der Hausautomatisierung ist die finanzielle
	Einsparung, \zB durch intelligente Heizsysteme.
	Zusätzlich soll sie aber auch die Bequemlichkeit der Bewohner fördern.

	Im Bereich der Hausautomatisierung gibt es bereits eine Vielzahl an
	sowohl offenen als auch proprietären Lösungsansätzen.  Jedoch
	erfüllen sie die Bedingungen an eine freie und erweiterbare
	Hausautomatisierungslösung nur bedingt.

	Offene Varianten sind häufig \enquote{Ein-Mann}-Projekte, die
	sehr spezielle Probleme verfolgen oder nur sehr unzureichend
	dokumentiert sind.  Teilweise stützen sie sich aber auch nur darauf,
	vorhandene proprietäre Systeme zu verwalten und miteinander zu
	verbinden.

	Diese proprietären Systeme sind für Fremdanbieter häufig aufgrund hoher
	Lizenzkosten unattraktiv.  Des Weiteren setzen sie Netzwerkprotokolle
	ein, die energieeffizienter als standardisierte Protokolle wie HTTP, TCP
	und IPv6 arbeiten. Die Protokolle stehen aber durch ihre Geschlossenheit
	der Interoperabilität entgegen.


	Diese Probleme begründen die Notwendigkeit des hier vorangetriebenen
	Projektes.

\subsection{Projektziele}
	Im Rahmen des Forschungsprojektes \emph{Sensornetze} an der HTW Dresden,
	geleitet durch Prof. Dr.-Ing. Jörg Vogt, soll überprüft werden, inwiefern
	eine offene Heimautomatisierung durch Anwendung bestehender
	Informationstechnologien in einem Sensornetzwerk realisierbar ist.

	Dabei sind für die Konzipierung im Rahmen unseres Forschungsprojektes
	die folgenden Zielstellungen erarbeitet worden:
	\begin{itemize}
	\item 	Es sollen nur quelloffene und freie Module verwendet werden.
		Dabei ist \enquote{frei} im Sinne der Definition der Free
		Software Foundation zu sehen.
	\item	Das Nutzen existierender, energieeffizienter
		Standardtechnologien \autocite{dunkels04ercim, dunkels08ipso}
		bilden die Grundlage für die
		\emph{Interoperabilität} (Herstellerunabhängigkeit) sowie
		die Langlebigkeit batteriebetriebener Sensorknoten.
	\item 	Die Erweiterbarkeit des Hausautomatisierungssystems
		und damit die Integration anderer -- auch proprietärer --
		Sensornetze ist zu untersuchen.
	\item 	Es ist wünschenswert, dass neue Geräte sich möglichst
		automatisch im System anmelden. Der Ausfall einzelner
		Sensorknoten darf das System nicht gefährden.
		Somit ist eine Skalierbarkeit des Hausautomatisierungssystems
		von Nöten.
	\end{itemize}

\subsection{Projektansatz}
	Die konkret untersuchte Hausautomatisierungslösung besteht aus
	einem dedizierten Netzknoten (Server), der die Regelung des Systems
	übernimmt.
	Auf diesem befindet sich folglich auch die Regelverarbeitung.

	Das Sensornetz soll auf dem Funkstandard 6LoWPAN basieren.
	Als Sensorknoten sollen ATmega128RFA1-Mikrocontroller mit dem
	Betriebssystem Contiki-OS eingesetzt werden.

	Der Informationsaustausch zwischen den einzelnen Sensorknoten und dem
	Server soll über das Constrained Application Protocol (CoAP) realisiert
	werden.
	CoAP ist ein im Vergleich zu HTTP energieeffizientes Protokoll, welches
	dennoch URIs unterstützt.

	Bestehende Sensornetze, wie FS20 oder HomeMatic,
	werden mit Hilfe von Gateways integriert,
	welche ein Mapping zwischen dem entsprechenden Sensornetzprotokoll
	und CoAP durchführen.

\subsection{Projekteinteilung}
	Zur Konzipierung der offenen Hausautomatisierung wurde
	das Projekt in folgende Teilprojekte gegliedert:
	\subsubsection{Steuerung des Systems}
		Es gilt zu untersuchen, ob ein zentraler oder dezentraler
		Hausautomatisierungsansatz in Hinblick auf die Bedürfnisse
		der Nutzer des Systems sinnvoll ist. Danach ist
		die Steuerung des Systems zu entwickeln
		und ein Anwendungsprotokoll zu finden, das sowohl im Sensornetz
		als auch auf normalen Computern eingesetzt werden kann.
	\subsubsection{Regelverarbeitung}
		Sofern die Steuerung des Systems definiert ist, kann
		mit der Regelverarbeitung begonnen werden.
		Hierbei sind die Fähigkeiten verschiedener Nutzer
		besonders zu berücksichtigen und eine einfach konfigurierbare
		Regelverarbeitung zu finden.
	\subsubsection{Sensornetz-Kommunikation}
		Neben der Steuerung muss die Kommunikation im Sensornetz
		analysiert werden und es muss ein Netzwerkstack gefunden
		werden, der sowohl die Anforderungen an das
		Hausautomatisierungssystems erfüllt
		wie auch diejenigen Anforderungen an
		\emph{Tiny-Networked-Sensors} \autocite{dunkels04contiki}.
	\subsubsection{Sensoransteuerung}
		\label{sec:teilprojekt}
		Um im Netzwerk Sensordaten auswerten zu können, muss untersucht
		werden, wie Sensoren und Aktoren auf dem im Ansatz beschriebenen
		Sensorknoten angesteuert und an die CoAP-Schnittstelle
		weitergeleitet werden können.
	\subsubsection{Integration bestehender Sensornetze}
		Wie in den Zielstellungen definiert, sollen grundsätzlich
		bestehende Sensornetze verwendet werden können.
		Beispielhaft wurden hierbei FS20- und HomeMatic-Geräte in das
		Hausautomatisierungssystem eingebunden.

\subsection{Dokumentabgrenzung}
	Dieses Dokument soll sich lediglich mit dem Teilprojekt der
	Sensoransteuerung beschäftigen.
	
	Dabei soll dargestellt werden, welche Möglichkeiten zur
	Ansteuerung der Peripherie existieren.	Des Weiteren sollen die
	Möglichkeiten unter Contiki untersucht werden und die Auswirkungen
	auf das Kommunikationsverhalten des Sensorknotens dargestellt werden.

	Der Einsatz von CoAP als Kommunikationsprotokoll wird dabei nicht
	detailliert untersucht. Es wird jedoch als Rahmenbedingung für den
	Einsatzzweck herangezogen.

	Durch den Einsatz der batteriebetriebenen ATmega128RFA1-Mikrocontroller
	als Sensorknoten müssen insbesondere die knappen Ressourcen Speicher und
	Energie beachtet werden.

	\section{Physikalische Ansteuerung}


%===============================================================================
	Die Ansteuerung der Sensoren kann in zwei grundsätzliche Klassen
	eingeteilt werden:
	\begin{LaTeXdescription}
	\item[\normalfont\itshape direkt]
		Der Sensor liefert eine durch den Mikrocontroller messbare
		elektrische Größe.  So wird bei einem Schalter durch einen
		Flankenwechsel ein Interrupt ausgelöst oder ein regelbarer
		Widerstand liefert unterschiedliche Spannungswerte, die
		durch einen Analog-Digital-Wandler abgetastet werden können.
	\item[\normalfont\itshape interfacebasiert]
		Die Peripherie wird über eine standardisierte Schnittstelle,
		wie UART oder \ItC, angesprochen.
	\end{LaTeXdescription}
	Die Ansteuerungen von Aktoren lässt sich analog klassifizieren.  Hierbei
	reagiert der Aktor auf eine anliegende Spannung oder Flanke bzw. ihm
	wird seine auszuführende Aktion über einen Befehl mitgeteilt.

	Diese Klassen sollen im Folgenden näher beleuchtet werden.


%===============================================================================
\subsection{Direkte Ansteuerung}
\label{sec:anst:direkt}
	Die direkte Ansteuerung von Sensoren ist verschiedentlich mit Problemen
	behaftet.
%-------------------------------------------------------------------------------
\subsubsection{Spannungswerte}
	Der Analog-Digital-Wandler leidet unter vorhersehbaren und unvorhersehbaren
	Messfehlern.

	Unvorhersehbare Messfehler ergeben sich durch elektromagnetische
	Störungen, die durch interne und externe Schaltkreise entstehen.
	\autocite[][S.\,422]{atmega128rfa1} Die Unvorhersehbarkeit besagt nicht,
	dass diese Fehler nur selten auftreten -- vielmehr treten sie sogar
	häufig auf.  Lediglich die konkrete Auswirkung des Fehlers ist
	nicht vorhersehbar.

	Diese Fehler können verringert werden, indem nicht benötigte Komponenten
	abgeschalten werden und  der \enquote{ADC Noise Reduction Mode}
	\autocite[][S.\,159]{atmega128rfa1} genutzt wird.  Üblich ist es auch
	mehrere (\zB 16 oder 128) Messungen direkt hintereinander durchzuführen
	und einen Mittelwert zu bilden.  Weitere Möglichkeiten sind hier
	auch in \autocite[][S.\,422]{atmega128rfa1} beschrieben.

	Dem stehen die vorhersehbaren Fehler entgegen.  Hier ist zum einen ein
	festgestellter systematischer Messfehler des ATmega128RFA1 zu nennen.
	So weichte der Messwert im unteren Wertebereich wenig und mit steigendem
	Wertebereich weiter ab. (siehe \autoref{fig:ad:messfehler} und \autoref{tbl:ad:messfehler})
	Der Fehler wurde noch nicht näher bestimmt und sollte weiter untersucht
	werden, ob dies von Modell zu Modell oder sogar von Gerät zu Gerät
	abweicht.  Letzteres wäre katastrophal, da sich dann für jedes einzelne
	Gerät die Notwendigkeit einer Kalibrierung ergeben würde.
	%\begin{figure}[!t]
\centering
\begin{circuitikz}[european resistors,node distance=0.3cm, european voltages]
\coordinate (vcc) at (0,0);
\coordinate (gnd) at (6,0);
\coordinate (r1s) at (0,-1);
\coordinate (between) at (3,-1);
\coordinate (r2e) at (6,-1);

\node [above of=vcc] {VCC};
\node [above of=gnd] {GND};
\draw (vcc)
	to [short, *-] (r1s)
	to [vR, l={\(R_1\)}] (between)
	to [R, l={\(R_2 \mathop{=} \unita{1{,}97}{\kilo\ohm}\)}] (r2e)
	to [short, -*] (gnd);

%\draw (vcc) to [open, v^>={\(U_{\text{ges}\)}] (gnd);
\draw (between)++(0,-.3cm) to [open, v_>={\(U_2\)}] ++(3cm,0);
\draw (vcc) to [open, v^>={\(U_\text{ges} \mathop{=} \unita{3{,}18}{\volt}\)}] (gnd);
\end{circuitikz}
\caption{AD-Beschaltung}
\label{fig:ad:beschaltung}
\end{figure}
%
	\begin{table}
\centering
\caption{Messwerte bei einer Referenzspannung von \unita{1{,}6}{\volt}}
\label{tbl:ad:messfehler}
\begin{tabular}{R{1}{3}rr}
\toprule
	\multicolumn{1}{c}{\thead{\(U_\text{anliegend}\) in \volt}}
	& \thead{erwartet}
	& \thead{gemessen}
	\tabularnewline
\midrule
	0,050	& 8	& 7	\tabularnewline
	0,091	& 15	& 13	\tabularnewline
	0,136	& 22	& 20	\tabularnewline
	0,169	& 27	& 26	\tabularnewline
	0,242	& 39	& 37	\tabularnewline
	0,314	& 50	& 48	\tabularnewline
	0,346	& 55	& 53	\tabularnewline
	0,516	& 82	& 80	\tabularnewline
	0,596	& 95	& 92	\tabularnewline
	0,799	& 127	& 124	\tabularnewline
	0,884	& 141	& 137	\tabularnewline
	1,020	& 163	& 159	\tabularnewline
	1,170	& 187	& 183	\tabularnewline
	1,268	& 202	& 198	\tabularnewline
	1,296	& 207	& 202	\tabularnewline
	1,353	& 216	& 211	\tabularnewline
\bottomrule
\end{tabular}
\end{table}
%
	\begin{figure}[!t]
\centering
\begin{tikzpicture}
%Achsen:
%\draw[->] (0,0) -- coordinate (xachse) (6,0);
%\draw[->] (0,0) -- coordinate (yachse) (0,6);

%\draw plot[scale=1/128, mark=*, mark options={fill=white}] file{dat/ad_messwerte_calc.dat};

\datavisualization [scientific axes,
	x axis={min value=0,
		max value=1.6,
		length=5cm,
		ticks={step=0.4, minor steps between steps=1},
		label={\(U_\text{anliegend}\) in \volt}},
	y axis={min value=0,
		max value=6,
		length=3cm,
		ticks={step=2, minor steps between steps=1},
		label={erwartet \(-\) gemessen}},
	all axes={grid},
	visualize as scatter]
	data {
		x,	y
		0.0505454252,	1
		0.0910023242,	2
		0.1360095528,	2
		0.1695885219,	1
		0.2426258714,	2
		0.3141725176,	2
		0.3468770764,	2
		0.516029654,	2
		0.5960608944,	3
		0.7990561224,	3
		0.8848305085,	4
		1.0202931596,	4
		1.170953271,	4
		1.2681376518,	4
		1.2967501552,	5
		1.3536300778,	5
		};
\end{tikzpicture}
\caption{Differenz zwischen Erwartungs- und Messwerten nach \protect\autoref{tbl:ad:messfehler}}
\label{fig:ad:messfehler}
\end{figure}
%

	Die Konvertierung der Spannung erfolgt durch sukzessive Approximation
	\autocite[][S.\,412]{atmega128rfa1} anhand einer Referenzspannung.
	Ergibt sich die zu messende Spannung aus der Betriebsspannung, so ist
	beim Batteriebetrieb des Sensorknotens ein Absinken der Betriebsspannung
	im Laufe der Betriebszeit zu beachten.

	Soll ein \emph{Aktor} mittels analoger Spannungswerte angesteuert werden,
	so kann dies durch Pulsweitenmodulation erreicht werden.  Auf dem
	ATmega128RFA1 können mit den Timern 0 und 2 Spannungswerte zwischen
	Ground und Vcc moduliert werden.

%-------------------------------------------------------------------------------
\subsubsection{Interrupts}
	Mittels Interrupts kann auf sich verändernde anliegende Signale
	an einem Eingangspin reagiert werden.  Der ATmega128RFA1 kann dabei
	auf steigende, fallend oder beide Flanken reagieren.  Zusätzlich kann er
	auch Interrupts auslösen, solange an dem Pin ein Low-Signal anliegt.
	\autocite[][S.\,220]{atmega128rfa1}

	Problematisch ist dabei, dass der Programmierer selber die Signale
	entprellen muss.  Außerdem wurde ein eigenartiges Verhalten
	der Interrupts 0 und 1 festgestellt.  Jeder Interrupt funktioniert für
	sich.
	Werden jedoch beide gleichzeitig verwendet, so kommt
	es zu folgender Situation:  Wird hardwareseitig der Interrupt
	0 ausgelöst, so wird einmal die Interrupt-Service-Routine
	des Interrupts 0 und zweimal die Interrupt-Service-Routine des
	Interrupts 1 ausgelöst.  Ebenso trat dieses Phänomen andersherum
	auf -- bei einem hardwareseitigen Auslösen des Interrupts 1
	wurde einmal die Interrupt-Service-Routine des Interrupts 1 und
	zweimal die Interrupt-Service-Routine des Interrupts 0 ausgelöst.
	Die Ursache für dieses Verhalten muss noch weiter untersucht werden.
	Dabei sollte auch geklärt werden, ob dies für die anderen Interrupts
	ebenso zutrifft.

	Um \emph{Aktoren} über Flanken oder High-/Low-Pegel zu steuern, können
	die Pins als Ausgang deklariert werden und dann bei Bedarf an- und
	ausgeschalten werden.


%===============================================================================
\subsection{Interfacebasierte Ansteuerung}
	Im Gegensatz zur in \autoref{sec:anst:direkt} beschriebenen direkten
	Ansteuerung, werden bei der interfacebasierten Ansteuerung die
	eigentlichen Messwerte von einem externen Schaltkreis bestimmt und
	dem Mikrocontroller nur noch über ein Interface zur Verfügung gestellt.
	Die Abstraktion zum physikalischen Problem der Messung ist also höher.

	Der ATmega128RFA1 bietet für die Schnittstellen UART, \ItC und SPI
	eine direkte Hardwareunterstützung.  Andere Schnittstellen müssen
	selber mit ihren Timings programmiert werden.  Über UART können nur
	zwei Geräte miteinander Daten austauschen. \ItC und SPI ermöglichen es,
	auf einem Anschluss (nacheinander) mit verschiedenen angeschlossenen
	Sensoren und Aktoren zu kommunizieren.

	Bei den Protokollen kann man zwischen Binär- und Textprotokollen
	unterscheiden.
%-------------------------------------------------------------------------------
\subsubsection{Binärprotokolle}
	Binärprotokolle sind relativ kompakt.  Der Master schickt einen
	Befehl\,/\,eine Anforderung an den Sensor bzw. Aktor, der dann direkt
	antwortet.


%-------------------------------------------------------------------------------
\subsubsection{Textprotokolle}
	Textprotokolle sind vielseitiger als Binärprotokolle.	Dafür
	benötigen die Befehle und Antworten mehr Bytes bei der Übertragung.
	Die Übertragung dauert also länger und verbraucht damit mehr Energie.
	Zusätzlich muss die Nachricht (umständlich) geparst und Zahlen von
	ASCII-Code in Integerwerte (und umgekehrt) überführt werden, um sie
	weiter verarbeiten zu können.	Auch dieses Parsen erhöht sowohl
	den Energiebedarf als auch den Speicherbedarf für Zeichenklassen-
	und Zustandsübergangstabellen.  Wenn außerdem die Länge der
	einzelnen Nachrichten nicht explizit durch das Protokoll begrenzt
	ist, muss dynamisch Speicher zum Puffern der eingehenden Nachricht
	reserviert werden, was sich ungünstig auf speicherarme Systeme wie
	Mikrocontroller auswirkt.


	\section{Konzepte der Ansteuerung in Contiki-OS}

	Zielplattformen von Contiki-OS sind eingebettete,
	batteriebetriebene Systeme, die über Funk (oder andere Netzwerke)
	miteinander kommunizieren. \autocite{contiki, dunkels04contiki}

	Contiki bietet eine Reihe von Modulen, die es für den Einsatz
	in einem drahtlosen Sensornetz prädestiniert.
	Für die Ansteuerung der Sensoren ist das Sensorinterface sehr
	interessant, weshalb es näher beleuchtet wird.

	Da im Sensorinterface die Sensoren physikalisch angesteuert werden
	müssen und im Forschungsprojekt viele \ItC-Sensoren verwendet wurden,
	wird ebenfalls ein \ItC-Interface und seine Eigenheiten betrachtet.

	Bevor allerdings das Sensor- bzw. \ItC-Interface erläutert wird,
	soll die Struktur von Contiki-OS \autocite{contikidoc} kurz vorgestellt werden,
	um zu verstehen, wo sich welches Interface bzw. welche Implementation
	befindet.

\subsection{Struktur von Contiki-OS}
	Da Contiki viele verschiedene Plattformen unterstützen möchte,
	wird generischer Code von plattformabhängigem Code (strikt) getrennt.
	Dabei gliedert sich Contiki in folgende Abschnitte:
	\begin{LaTeXdescription}
	\item[\normalfont\itshape core]
		Der Kern von Contiki arbeitet \emph{plattformunabhängig}.
		Der Netzwerk-Stack ist beispielsweise bis auf die Ansteuerung
		des Funksensors komplett enthalten. Das Interface für die
		Ansteuerung des Funksensors befindet sich aber wiederum
		im Kern.
	\item[\normalfont\itshape cpu]
		In diesem Abschnitt befindet sich \emph{CPU-abhängiger Code}.
		Es wird zwischen AVR, ARM und weiteren CPUs unterschieden.
		Hier ist \iA die interfacebasierte physikalische Ansteuerung
		implementiert, da Funktionalitäten des Mikrocontrollers genutzt
		werden. Zugehörige Schnittstellen können sich im Kern befinden.
	\item[\normalfont\itshape platform]
		An dieser Stelle werden \emph{plattformspezifische
		Eigenschaften}
		umgesetzt. Vor allem befindet sich die Main-Methoden in diesem
		Bereich. Eine Plattform kann auch für einen bestimmten
		Mikrocontroller wie den ATmega128RFA1 stehen.
	\item[\normalfont\itshape apps]
		Wenn Funktionalitäten nicht im Kern benötigt werden und sie
		dennoch plattformunabhängig sind, so kommen diese hier hin.
		Die CoAP/REST-Implementierung befindet sich beispielsweise
		hier, da es sich nur um eine Anwendungsschicht im
		Netzwerk-Stack handelt.
	\item[\normalfont\itshape examples / project]
		Der letzte Bereich organisiert ein spezielles Projekt.
		An dieser Stelle wird festgelegt, welche Plattform verwendet
		wird und welche Prozesse automatisch gestartet werden.
		Ein spezielles Projekt könnte ein Sensorknoten sein, welcher
		Sensoren zum Auslesen der Raumtemperatur sowie der
		Luftfeuchtigkeit bietet und diese Informationen über Funk
		(6LoWPAN) verteilen kann.
	\end{LaTeXdescription}


\subsection{I2C-Interface}
	Im Forschungsprojekt wurden vor allem Sensoren verwendet, die
	eine interfacebasierte Ansteuerung über \ItC erlauben.

	\ItC\footnote{\ItC wird im AVR Reference Manual \autocite{atmega128rfa1}
		auch Two-Wire Interface (TWI) genannt.}
	ist eine serielle Schnittstelle, die den Anschluss mehrerer
	\ItC-Devices ermöglicht. So konnte sowohl ein Luftfeuchtigkeits-,
	ein Druck-, ein Beschleunigungs- sowie ein Lichtsensor an einem Bus
	angeschlossen werden. Am Mikrocontroller werden für \ItC nur zwei Pins
	(\ItC-SCL, \ItC-SDA) benötigt. Der Mikrocontroller fungiert hierbei als
	Master und leitet die Übertragung der Sensordaten ein.

	In Contiki wurde kein \ItC-Interface gefunden. Da \ItC keine generische
	Implementation aufgrund der Verwendung von Mikrocontroller-Komponenten
	erlaubt, wurde ein eigenes \ItC-Interface in den Kern von Contiki
	integriert. Eine zugehörige Implementation erfolgte dann im
	CPU-abhängigen Teil von Contiki.

	Dieses \ItC-Interface erlaubt die Verwendung des \ItC-Busses,
	ohne detaillierte Kenntnisse zum Ablauf der Kommunikation zu haben.
	Allerdings ist der Multi-Master-Betrieb nicht berücksichtigt worden,
	da dieser für die Ansteuerung von Sensoren nicht benötigt wird.
	Das Interface ist in \autoref{lst:i2c-interface} zu sehen und zeigt
	die drei benötigten Funktionen zum Ansteuern eines Sensors
	sowie die möglichen Übertragungsfehler.

	% I2C-Interface -- Listing
	\lstinputlisting[style=float,language=c,%
		label=lst:i2c-interface,%
		caption={Das \ItC-Interface (Ausschnitt)},%
	]{listing/i2c-interface.h}

	Die AVR-Implementation des \ItC-Interfaces benutzt den vorgesehenen
	\ItC-Interrupt, um die Übertragung zügig abarbeiten zu können.
	An welchem Pin sich der \ItC-Bus befindet, wird durch die AVR-Libc
	bestimmt.

	Die AVR-Implementation sieht nur eine synchrone Übertragung vor,
	das heißt, dass das Betriebssystem so lange warten muss, bis die Übertragung
	abgeschlossen oder ein Fehler aufgetreten ist. Durch die Verwendung
	des \ItC-Interrupts wird die Übertragung sogar bevorzugt abgearbeitet
	bzw. kann \idR nicht durch andere Interrupts unterbrochen werden.

	\ItC-Sensoren können in Bezug auf das Übertragungsverhalten
	in zwei Klassen eingeteilt werden:
	Die ersten verwenden einen
	Hold-Mode\footnote{Der Sensor \emph{hält} den \ItC-Bus während der
		Messung des Sensorwertes, so dass die Kommunikation nicht
		abgeschlossen oder eine andere nicht begonnen werden kann.
		Der Feuchtigkeitssensor SHT21 unterstützt sowohl den Hold-
		wie auch den No-Hold-Mode.}
	und die anderen einen No-Hold-Mode.
	
	\begin{LaTeXdescription}
	\item[\normalfont\itshape Hold-Mode]
		Dieses synchrone Übertragungsverhalten hat bei \ItC-Sensoren,
		die im so genannten Hold-Mode arbeiten, den Nachteil, dass
		während des Messens des Sensorwertes der \ItC-Bus blockiert ist
		und damit auch das Betriebssystem warten muss, bis die
		Übertragung abgeschlossen ist.
		
		Insgesamt wird mehr Energie
		verbraucht. Auf der anderen Seite werden keine Mechanismen
		gebraucht, die anzeigen, dass eine Übertragung bereits
		stattfindet.

	\item[\normalfont\itshape No-Hold-Mode]
		Die zweite Möglichkeit der Übertragung ist der No-Hold-Mode.
		Bei diesem wird dem Sensor mitgeteilt, dass ein Sensorwert
		gemessen werden soll. Der Mikrocontroller muss dann beim Sensor
		nach einer gewissen Messdauer den Sensorwert abfragen (pollen),
		sofern er bereits vorliegt.
		
		Viele Sensoren bieten für den No-Hold-Mode auch eine
		Interrupt-Leitung an, die anzeigt, wann der neue Sensorwert
		ausgemessen ist und somit abgefragt werden kann.

		Auch hier ist das synchrone Übertragungsverhalten sinnvoll, da
		der jeweilige Prozess die Übertragung einleitet und damit auch
		sofort das Ergebnis der Übertragung vorliegen hat.  Nach einer
		bestimmten Zeit oder beim Eintreten eines externen Interrupts
		wird dann der Sensorwert abgerufen.
	\end{LaTeXdescription}

	Abschließend ist zu sagen, dass es ratsam ist,
	zur Ansteuerung der \ItC-Sensoren ein Sensorinterface zu benutzen.
	Mit diesem lassen sich dann alle konfigurierbaren Optionen einstellen,
	wie den Hold- oder No-Hold-Mode.

\subsection{Das generische Sensorinterface}

	Contiki besitzt ein \emph{generisches} Sensorinterface, um Sensoren
	ansteuern zu können. Damit ist eine Abstraktion geschaffen, die
	die Entwickler der Sensortreiber dazu bewegen soll, ein einheitliches
	Schema zu nutzen.
	Entscheidend bei diesem Interface sind drei Funktionen:
	\begin{itemize}
	\item 	Die erste Funktion (\lstinline=configure=)
		konfiguriert den Sensor;
	\item 	die zweite (\lstinline=status=)
		gibt Informationen zum Sensor zurück,
		beispielsweise die konfigurierten Einstellungen
		oder den Zustand des Sensors;
	\item 	die dritte (\lstinline=value=)
		kann die Sensorwerte auslesen,
		beispielsweise den gemessenen Temperaturwert oder die
		Lichtintensität.
	\end{itemize}
	Zusammen mit einen Namen für den Sensor bilden diese vier Bestandteile
	den Sensortreiber.

	Eine frühe Version des Interfaces erforderte weitere Funktionen
	vom Sensortreiber: Hierunter war eine Initialisierungs-, 
	Aktivierungs- und Deaktivierungsfunktion. Diese Funktionen können aber
	durch die Konfiguration des Sensors übernommen werden,
	weshalb sie aus dem Interface entfernt wurden.

\subsubsection{Das Konfigurieren}
	Die Funktion \lstinline=configure= dient dem Konfigurieren.
	Sie besitzt zwei Parameter, wovon der erste den Konfigurationstyp
	und der zweite den Konfigurationswert angibt.
	Ein Rückgabewert kann anzeigen, ob die Konfiguration erfolgreich war.

	Was bei einem Sensor konfiguriert werden kann und was sinnvoll
	zu konfigurieren ist, muss jeder Sensortreiber für sich entscheiden.
	Es gibt lediglich die Begrenzung, dass die beiden Parameter sowie 
	der Rückgabewert Integers sind (und damit abhängig vom Mikrocontroller
	durchaus unterschiedlich groß sein können).

	Das Sensorinterface definiert globale Konfigurationstypen,
	wovon der wichtigste \lstinline=SENSORS_ACTIVATE= ist.
	Dieser sollte das Aktivieren und Deaktivieren des Sensortreibers
	und sinnvoller Weise auch des Sensors bewirken.
	Beispielsweise kann der Sensor mit Hilfe dieses
	Konfigurationsparameters in den Sleep-Mode versetzt werden.

\subsubsection{Der Auslesen der Status}
	Die Funktion \lstinline=status= dient dem Auslesen der Status
	eines Sensortreibers. Diese Funktion gibt den Status
	als Rückgabewert zurück. Der gewünschte Status wird wie bei der
	Konfiguration per Parameter übergeben. Es ist daher nicht undenkbar,
	dass die beiden Parameter, Konfigurationstyp und Statustyp,
	dieselben Wertigkeiten behandeln.

	Auch hier ist es sehr zu empfehlen,
	bei Übergabe von \lstinline=SENSORS_ACTIVATE= den Zustand des
	Sensortreibers/Sensors zurückzugeben, wobei \enquote{0} bedeutet, dass
	der Sensor aus ist. Eine \enquote{1} gibt die Aktivität des Sensors an.

\subsubsection{Das Auslesen von Sensorinformationen}
	Die Funktion \lstinline=value= bewirkt, dass Sensorinformationen
	ausgelesen werden. Der übergebene Parameter kennzeichnet
	den auszulesenden Sensorwert, beispielsweise einen Temperaturwert.

	Da der Rückgabetyp vom Typ Integer\footnote{Ein Integer ist auf dem
		ATmega128RFA1 nur \unita{8}{\bit} breit. Auf anderen Architekturen
		kann ein Integer breiter sein.}
	ist, ist es nicht möglich jeden beliebigen Sensorwert zurückzugeben.
	Es gibt Sensorwerte, die einen größeren Wertebereich benötigten.
	Ein Beispiel sind Temperaturwerte:
	Da die Typen Float oder Double auf Mikrocontrollern so oft wie möglich
	vermieden werden sollten, wird die Temperatur \unita{22.53}{\celsius}
	typischerweise als ganze Zahl (\zB \(2253\)) übergeben -- was aber den
	Wertebereich einer Integerzahl (0--255) auf dem ATmega128RFA1
	überschreitet.

	Hinzu kommt, dass ein fehlerhaftes Verhalten
	ebenfalls im Rückgabewert eingebunden werden muss.
	Ein durchaus oft auftretender Fehler bei batteriebetriebenen
	Netzknoten resultiert aus einer zu niedrigen Betriebsspannung:
	Der Mikrocontroller ATmega128RFA1 kann noch mit \unita{1.8}{\volt}
	arbeiten, aber einige Sensoren wie der SHT21 stellen bei \unita{2.1}{\volt}
	den Betrieb ein, so dass ein Auslesen des Sensorwerts zwangsweise
	fehlschlagen wird.

	Aus diesem Grund empfiehlt es sich, dass der Rückgabetyp kennzeichnet,
	ob ein Fehler aufgetreten ist oder nicht. Bei bereits vorhandene
	Sensorimplementationen wurde \enquote{0} für fehlerhaftes
	sowie \enquote{1} für fehlerfreies Verhalten des Sensors verwendet.
	Prinzipiell ist dieses (bisher) nicht festgelegt, so dass auch eine
	\enquote{0} für fehlerfreies Verhalten stehen könnte und ein
	von Null verschiedener Wert für den Fehlertyp.

	Eigene Tests haben gezeigt, dass es sich empfiehlt, den Sensorwert im
	Sensortreiber zwischenzuspeichern.  Dieses kann auch in einem
	RAW-Format passieren, so dass erst das Auslesen des
	zwischengespeicherten Sensorwerts diesen in das gewünschte Format
	konvertiert.

	Ein weiterer Vorteil ist es, dass der Sensortreiber alle
	Sensorwerte intern in einer beliebigen Datenstruktur ablegen kann, so
	dass auch historische Werte -- ggf. mit Zeitstempel -- abrufbar sind.
	Der Sensortreiber stellt dann die benötigten Funktionen zum Auslesen
	der Sensorwerte bereit.

	Zur Zeit ist keine Abstraktion zum Übergeben des Sensorwertes vom
	Sensorinterface zum Aufrufer vorgesehen.
	Es wäre also interessant, ob solch eine Abstraktion in Bezug zur
	Hausautomatisierung existiert und wie das vorhandene, generische
	Sensorinterface erweitert werden kann, um die Anforderungen
	der Ansteuerung auf Anwendungsebene zu erfüllen.

	\section{Ansteuerung auf Anwendungsebene}
\label{sec:coap}

	In einem Hausautomatisierungssystem sollen Sensorinformationen
	sowie Steuerbefehle für Aktoren übertragen werden.
	Daher ist es erforderlich, auf den Netzknoten eine Anwendungsschicht
	zu besitzen, die diese Informationen bzw. Befehle
	versenden und empfangen kann.

	An dieser Stelle geht es also nicht um die Umsetzung der genauen Befehle an
	die Sensoren und Aktoren oder die Programmierabstraktion,
	sondern um eine sinnvolle Abstraktion auf Netzwerkebene,
	so dass andere Netzknoten alle nötigen Informationen zur
	Weiterverarbeitung vorliegen haben.

	Wichtige Informationen neben dem aktuellen Sensorwert
	können Folgende sein:
	\begin{itemize}
	\item 	der Name des Sensors,
	\item 	der Typ des Sensors (\zB Temperatur, Lichtintensität)
	\item 	den möglichen Wertebereich,
	\item 	Ungenauigkeiten.
	\end{itemize}

	Ein Temperatursensor liefert einen Temperaturwert im Bereich
	von \unita{-40}{\celsius} bis \unita{+100}{\celsius}.
	Dem Regelverarbeitungssystem genügt es unter Umständen,
	dass nur (starke) Temperaturschwankungen aufgezeigt werden,
	\zB durch das Öffnen eines Fensters an kalten Tagen.
	Damit ist es überflüssig, dass jede Sekunde der Sensorwert
	durch die Regelverarbeitung gepollt\footnote{\enquote{Pollen} bezieht sich
		in diesem Abschnitt auf die Kommunikation zwischen
		Regelverarbeitungsserver und Sensorknoten, nicht aber auf ein
		Pollen zwischen Sensorknoten und Sensor.}
	wird, da ein Fenster
	an kalten Tagen nur selten geöffnet wird.

	Das Pollen der Sensorwerte, also der Temperaturwerte,
	bewirkt folglich ein Überfluten des Sensornetzes
	mit \iA nicht benötigten Informationen.

	Hierfür müssen Mechanismen zur Ansteuerung auf Anwendungsebene
	bereitgestellt werden, die die Netzlast reduzieren und somit Energie
	für die Übertragung und Auswertung der Nachrichten eingespart werden
	kann.

	Daneben gibt es auch Situationen, in denen eine Änderung der
	Sensorinformation zeitkritisch verarbeitet werden muss,
	das heißt, es kann nicht abgewartet werden, bis die Regelverarbeitung
	das nächste Mal den Sensorwert pollt.
	Hierzu muss der Regelverarbeitungsserver selbstständig
	benachrichtigt werden, damit eine schnelle Abarbeitung erfolgen kann.

\subsection{CoAP}
	Ein Anwendungsprotokoll, dass die Anforderungen der Ansteuerung
	auf Anwendungsebene erfüllen kann, ist CoAP. Es ist ein asynchrones
	Protokoll, weshalb UDP als Transfer Protokoll ausreicht.  Da CoAP sich
	noch in Entwicklung befindet, ist keine Kompatibilität zu vorherigen
	Draft-Versionen sichergestellt.

	Zum Zeitpunkt des Forschungsprojekts befand sich CoAP bei Draft-12 bzw.
	-13. Die implementierten Draft-Versionen von CoAP waren meist auf
	unterschiedlichen Ständen.  So bot Contiki 2.5 bei Projektbeginn nur
	Draft-03. Nach Veröffentlichung von Contiki 2.6 war auch Draft-07
	implementiert.
	\autocite{kovatsch11low-power}
	
	Seitens des Hausautomatisierungsservers sah es ähnlich aus: Die
	CoAP-Implementation auf dem Hausautomatisierungsserver (CoAPy) wurde
	der Contiki-Version angepasst, um eine Kommunikation herzustellen.

	Das Anwendungsprotokoll CoAP bietet verschiedene Vorteile
	für die Hausautomatisierung im Allgemeinen und für die Ansteuerung
	der Sensoren und Aktoren im Konkreten. Diese Vorteile werden
	in den folgenden Abschnitten näher erläutert.

	Weitere Informationen zu CoAP sind der Projektgruppe
	\enquote{Steuerung des Systems} sowie im Speziellen zu Contiki
	der Gruppe \enquote{Sensornetz-Kommunikation} zu entnehmen.

\subsection{Kommunikationsbeispiele}
	Mit dem Discovery Mechanismus von CoAP ist es möglich, die
	Ressourcen des Sensorknotens zu erfahren.
	Ein Sensorknoten mit dem Sensor SHT21 kann sowohl die Luftfeuchtigkeit
	wie auch die Temperatur messen.
	Damit können für diesen Sensorknoten zwei
	REST-Ressourcen angelegt werden: \lstinline=sensor/sht21/temperature=
	und \lstinline=sensor/sht21/humidity=.
	Der Sensorknoten könnte aber auch noch andere Ressourcen bieten,
	wie eine Einstellungsoption, die es erlaubt,
	den Radio Duty Cycle zu verändern oder Einstellungen am SHT21
	vorzunehmen.

	Es können zwei grundsätzliche Verfahren unterschieden werden,
	um auf ein Request, der an den Sensorknoten (CoAP-Server) gerichtet
	ist, zu antworten:
	\begin{LaTeXdescription}
	\item[\normalfont\itshape Piggy-backed Response]
		Bei einem Request wird zuerst der
		Sensorwert ausgelesen und dann zusammen mit dem ACK
		als Response übertragen werden. Hierbei wird der
		Netzwerktraffic reduziert, erfordert aber eine schnelles
		Auslesen des Sensorwertes.
	\item[\normalfont\itshape Separate Response]
		Die andere Möglichkeit ist es, sobald ein Request eingetroffen
		ist, den Empfang mit einem ACK zu bestätigen.
		Danach wird der Sensorwert erfasst und als Response zum
		CoAP-Client geschickt, der dann den Empfang mit einem ACK
		bestätigt.
		\label{sec:coap:separateresponse}
	\end{LaTeXdescription}

	Den Zustand einer LED sollte folglich nur mit einem
	\emph{Piggy-backed Response} beantwortet werden, um den Netzwerktraffic
	gering zu halten.

	Für das Auslesen des Feuchtigkeitssensors SHT21 wird je
	nach \ItC-Bus-Geschwindigkeit etwa \unita{50}{\micro\second} benötigt.
	Hier kann man je nach Anwendungsfall entscheiden, welche
	Response-Methode besser ist.

	Für den Lichtsensor TSL2561 eignet sich im Prinzip nur die zweite
	Variante, denn dieser benötigt durchaus
	\unita{400}{\milli\second} zum Messen der Lichtintensität.
	Es gäbe zwar die Möglichkeit eine kürzere Messdauer einzustellen,
	allerdings kann dadurch der Messfehler sich erhöhen.
	Zur Beantwortung des Requests der aktuellen Lichtintensität
	ist ein separates Response wichtig, denn der CoAP-Client
	-- \idR ist dies der Regelverarbeitungsserver --
	wird im Falle eines Piggy-backed-Response mehrere Requests losschicken,
	da er davon ausgehen muss, dass die Nachricht nicht beim CoAP-Server
	angekommen ist.

	Bei zeitkritischen Ereignissen -- wie das Öffnen eines Fensters oder
	der Wohnungstür -- möchte der CoAP-Client nicht ständig den
	Sensorknoten pollen müssen. Daher bietet CoAP auch die Möglichkeit,
	eine Ressource zu überwachen, das Observe-Feature.
	Bei diesem Feature teilt ein CoAP-Client dem Sensorknoten mit, dass
	der Client selbstständig informiert werden soll,
	sobald sich der Zustand einer Ressource ändert.

	In Contiki sind in der REST-Implementierung für dieses Observe-Feature
	zwei Ressource-Typen zur Verfügung gestellt worden:
	Der erste Typ ist eine periodische Ressource, die folglich periodisch
	den Zustand der Ressource prüft und \idR nach Überschreiten eines
	Schwellwertes eine Response-Nachricht an alle eingetragenen
	CoAP-Clients verschickt. Damit kann das Sensornetz entlastet werden, da
	nicht jeder Sensorwert übertragen wird.
	Der andere Typ ist eine eventbasierte Ressource, die aufgrund des
	Eintretens eines vorher definierten Events eine Response-Nachricht
	ausgibt.  Das Event kann beispielsweise das Drücken eines Buttons sein
	oder aber ein PUT-Request an eine andere Ressource.

	Bezogen auf den Türkontaktsensor kann nun eine Eventressource angelegt
	werden: Beim Öffnen der Tür wird dann der Sensorknoten per Interrupt
	aus seiner Schlafphase erwachen und sofort eine CoAP-Nachricht
	an alle eingetragenen CoAP-Clients verschicken, um anzuzeigen,
	dass die Tür geöffnet wurde.


\subsection{CoAP Schnittstelle}
	Da ein paar Kommunikationsbeispiele genannt wurden, soll
	nun die CoAP Schnittstelle in Contiki beispielhaft vorgestellt werden.

	Die eventbasierte sowie die periodische Ressource werden hierbei nicht
	als Quellcode vorgestellt, da die Möglichkeiten hierbei sehr vielfältig
	sein können und dieses Observe-Feature noch nicht vollständig im
	Projekt untersucht werden konnte.

	Der REST Server kann in Contiki als
	Prozess\footnote{Prozesse sind in Contiki Protothreads
		\autocite{dunkels05using, dunkels06protothreads, dunkels07simplifying}}
	gestartet werden.
	Die Initialisierungsphase des Servers,
	welche in \autoref{lst:restserverprocess} gezeigt ist,
	erfolgt in drei Schritten:
	\begin{enumerate}
	\item 	Zuerst müssen alle Komponenten in den Ressourcen initialisiert werden,
		sofern dies noch nicht geschehen ist.
	\item 	Danach wird die REST Engine initialisiert und
	\item 	zuletzt werden alle benötigten Ressourcen
		aktiviert.
	\end{enumerate}

	% CoAP REST Server -- Listing
	\lstinputlisting[style=float,language=c,%
		label=lst:restserverprocess,%
		caption={CoAP REST Serverprozess},%
	]{listing/coap-server-process.c}

	Damit Ressourcen aktiviert werden können, müssen diese zuvor
	auch angelegt werden. In Contiki geschieht dies mit dem Makro
	\lstinline=RESOURCE=. Der erste Parameter definiert den Namen
	der Ressource, welcher zugleich auch im Handler vorkommt.
	Der zweite Parameter kennzeichnet, welche \emph{REST-Method-Types}
	von der Ressource unterstützt werden. Aus HTTP bekannte Method-Types
	sind
	\lstinline=METHOD_GET=,
	\lstinline=METHOD_POST=,
	\lstinline=METHOD_PUT=,
	\lstinline=METHOD_DELETE=.
	Auf eine Beschreibung wird an dieser Stelle verzichtet und auf die
	Dokumentation von CoAP \autocite{kovatsch11low-power} verwiesen.
	Der dritte Parameter gibt den URI-Pfad an und der letzte
	die Attribute. Es besteht die Möglichkeit, eine Ressource
	auch als Subressource zu definieren.

	In \autoref{lst:rest_res_text} ist eine Text-Ressource gezeigt.
	Dieser Ressource kann ein CoAP-Client eine Wunschlänge
	per Query-Variable übergeben.

	% CoAP REST Resource -- Text -- Listing
	\lstinputlisting[style=float,language=c,%
		label=lst:rest_res_text,%
		caption={Text Ressource},%
	]{listing/coap-server-res-text.c}

	Bei einer einfachen Sensor-Ressource genügt die GET-Methode, so dass
	ein Client die Ressource lediglich auslesen kann. Die Piggy-backed
	Variante ist in \autoref{lst:rest_res_sensor} gezeigt.

	% CoAP REST Resource -- Sensor -- Listing
	\lstinputlisting[style=float,language=c,%
		label=lst:rest_res_sensor,%
		caption={Sensor Ressource},%
	]{listing/coap-server-res-sensor.c}

	Eine Erweiterung des in \autoref{lst:rest_res_sensor} gezeigten
	Beispiels könnte die Einführung einer Query-Variable sein,
	welche anzeigt, ob der Temperaturwert in Grad Celsius
	oder Grad Fahrenheit übertragen werden soll. Hierzu sind keine zwei
	Ressourcen notwendig.

	Soll die Observe Funktion von CoAP genutzt werden, dann stehen hierzu
	zwei weitere Ressource-Typen bereit, die die normale Ressource
	erweitern: Die periodische sowie die eventbasierte Ressource.
	Beide Ressourcetypen haben einen zusätzlichen Handler.
	Ein CoAP-Client kann dieser Ressource überwachen und wird damit
	in eine Liste aufgenommen.

	In diesem Abschnitt wurde lediglich ein Einblick in die
	CoAP-Schnittstelle gegeben. Die Implementation der CoAP-Funktionalität
	ist in Contiki zweigeteilt: Die Ressourcen und die REST Engine
	werden von einem \emph{Abstraction Layer} bereitgestellt, der sich
	in Contiki \emph{Erbium} nennt und sich im Apps-Ordner befindet.
	Dieser stellt die \emph{RESTful Web services} zur Verfügung.

	Zur Kompilierzeit wird bestimmt, welche CoAP-Implementation genutzt
	wird. Die Implementationen sind ebenfalls im Apps-Ordner von Contiki
	zu finden: Es ist die Draft-Version 03 (er-coap-03)
	und 07 (er-coap-07) implementiert.

	Sowohl Abstraction Layer wie auch die beiden Draft-Implementationen
	stammen vom selben Autor, Matthias Kovatsch. Ziel der Implementation
	ist ein Low-Power-CoAP \autocite{kovatsch11low-power} für Contiki
	bereitzustellen.

\subsection{Zusammenhang zwischen CoAP und RDC}
	Mit dem im vorherigen Absatz genannten Beispiel des Türkontaktsensors
	kann man einen Zusammenhang zwischen CoAP und dem Radio Duty Cycle (RDC)
	erkennen:

	Angenommen, die Tür wird im Normalfall nur alle drei Monate einmal
	geöffnet und der Sensorknoten arbeitet zuverlässig, so dass ein Ausfall
	ausgeschlossen werden kann. Dann ist es nicht von Vorteil, dass der
	RDC besagt, dass der Knoten alle \unita{8}{\hertz} aufwacht, um
	empfangsbereit zu sein. Die hierfür benötigte Energie könnte durchaus
	eingespart werden, so dass sich die Lebenszeit des Sensorknotens
	drastisch erhöht. Der Nachteil ist natürlich, dass der Sensorknoten
	während einer langen Zeit nicht erreichbar ist. Wenn man nur die
	Zustände \enquote{offen} und \enquote{geschlossen} überträgt, könnte
	der Fall eintreten, dass aufgrund eines Ausfalls oder anderer Gründe
	der letzte Zustand nicht bekannt ist. Das nächste Update erfolgt
	allerdings nicht in nächster Zeit.
	
	Hierfür bietet CoAP eine Lösung an: Man kann CoAP-Proxies verwenden,
	die die Sensorwerte cachen und somit im Namen der schlafenden
	Sensorknoten antworten können.

	Wenn diese Idee weiterverfolgt wird, so erscheint es naheliegend,
	auch einen unregelmäßigen RDC zu verwenden, wie im Falle
	des Türkontaktsensors. Normalerweise wird in einem Sensornetz davon
	ausgegangen, dass alle Netzknoten den gleichen RDC
	besitzen. Eine vernünftige Zusammenarbeit von CoAP, dem Routing
	Protokoll sowie dem RDC Protokoll kann bewirken,
	dass verschiedene RDCs im gleichen Sensornetz
	zum Einsatz kommen. Diese Theorie gilt es noch zu prüfen.


	\section{Ausblick}
	Die durchgeführten exemplarischen Implementationen haben geholfen,
	einen Überblick über die Möglichkeiten zur Anbindung von Sensoren
	bzw. Aktoren unter Contiki zu entwickeln.  Dabei sind allerdings
	verschiedene Probleme aufgetreten, die in nachfolgenden Arbeiten
	untersucht und gelöst werden müssen:
	\begin{itemize}
	\item	Die Messungenauigkeiten des Analog-Digital-Wandlers des
		ATmega128RFA1 sowie das ungewöhnliche Verhalten der
		Interrupts 0 und 1, beides in \autoref{sec:anst:direkt}
		beschrieben, müssen untersucht und gelöst werden.
	\item	Die CoAP-Schnittstelle Contikis muss dahingehend untersucht
		werden, wie die in \autoref{sec:coap} angesprochenen
		eventbasierten und periodischen Ressourcen konkret umgesetzt
		werden können.  Auch die Verwendung des Separate Response
		unter Contiki muss getestet werden.
	\item	Die Stromsparmechanismen von Contiki
		\autocite{ritter05experimental, dunkels07softwarebased}
		müssen genutzt werden.
		Dazu sind unter anderem die Schlafmodi des ATmega128RFA1 zu
		implementieren und ein Konzept zu entwickeln, wie die Prozesse
		dem Betriebssystem mitteilen können, ob ein Schlafmodus
		zulässig ist oder nicht.
	\end{itemize}


	% if have a single appendix:
%\appendix[Proof of the Zonklar Equations]
% or
%\appendix  % for no appendix heading
% do not use \section anymore after \appendix, only \section*
% is possibly needed

% use appendices with more than one appendix
% then use \section to start each appendix
% you must declare a \section before using any
% \subsection or using \label (\appendices by itself
% starts a section numbered zero.)
%

%\appendices
\appendix[Vorlesung Sensornetze]

%===========================================================
% ....
%===========================================================

%\section{Vorlesung Sensornetze}

	In der Vorlesung \emph{Sensornetze}
	wurden Themen behandelt, die gewinnbringend in das Forschungsprojekt
	einfließen konnten.
	Folgende Themen waren für das Forschungsprojekt besonders interessant:
	\begin{itemize}
	\item 	Contiki-OS: Implementationsaspekte und Netzwerk-Stack
	\item 	RPL (Routing Protocol Layer)
	\item 	CoAP
	\item 	HomeMatic
	\item 	KNX
	\end{itemize}
	Zu diesen Themen wurden Vorträge gehalten und es wurden auch
	Papers geschrieben, die über Prof. Vogt mit Zustimmung der jeweiligen
	Authoren eingesehen werden können.


%===========================================================
% ....
%===========================================================

% you can choose not to have a title for an appendix
% if you want by leaving the argument blank
%\section{}

%	Appendix two text goes here.

%===========================================================
% Acknowledgements
%===========================================================

% use section* for acknowledgement
\ifCLASSOPTIONcompsoc
  % The Computer Society usually uses the plural form
  %\section*{Acknowledgments}
  \section*{Danksagungen}
\else
  % regular IEEE prefers the singular form
  %\section*{Acknowledgment}
  \section*{Danksagung}
\fi

	Die Autoren wollen Prof. Dr.-Ing. Jörg Vogt für das Anbieten und Betreuen des
	Forschungsprojektes \emph{Sensornetze} sowie den anderen Teilnehmern
	für ihre Mitarbeit danken.
	Es ist ein sehr interessantes Forschungsgebiet und wir wünschen
	den noch verbleibenden sowie den neu hinzukommenden Teilnehmern
	viel Spaß bei der Fortführung dieses Forschungsprojektes.
	Für Fragen stehen wir gerne zur Verfügung.



	%%%%%%%%%%%%%%%%%%%%%%%%%%%%%%%%%%%%%%%%%%%%%%%%%%%%
	%\backmatter
	% Literaturverzeichnis: biblatex
	%\phantomsection
	\printbibliography

	%\pdfbookmark[1]{&\approx& ~Literaturverzeichnis}{toc}
	
	% Literaturverzeichnis: bibtex
	%\bibliographystyle{alphadin}
	%\bibliography{literature}
	%%%%%%%%%%%%%%%%%%%%%%%%%%%%%%%%%%%%%%%%%%%%%%%%%%%%
\end{document}
