\section{Physikalische Ansteuerung}


%===============================================================================
	Die Ansteuerung der Sensoren kann in zwei grundsätzliche Klassen
	eingeteilt werden:
	\begin{LaTeXdescription}
	\item[\normalfont\itshape direkt]
		Der Sensor liefert eine durch den Mikrocontroller messbare
		elektrische Größe.  So wird bei einem Schalter durch einen
		Flankenwechsel ein Interrupt ausgelöst oder ein regelbarer
		Widerstand liefert unterschiedliche Spannungswerte, die
		durch einen Analog-Digital-Wandler abgetastet werden können.
	\item[\normalfont\itshape interfacebasiert]
		Die Peripherie wird über eine standardisierte Schnittstelle,
		wie UART oder \ItC, angesprochen.
	\end{LaTeXdescription}
	Die Ansteuerungen von Aktoren lässt sich analog klassifizieren.  Hierbei
	reagiert der Aktor auf eine anliegende Spannung oder Flanke bzw. ihm
	wird seine auszuführende Aktion über einen Befehl mitgeteilt.

	Diese Klassen sollen im Folgenden näher beleuchtet werden.


%===============================================================================
\subsection{Direkte Ansteuerung}
\label{sec:anst:direkt}
	Die direkte Ansteuerung von Sensoren ist verschiedentlich mit Problemen
	behaftet.
%-------------------------------------------------------------------------------
\subsubsection{Spannungswerte}
	Der Analog-Digital-Wandler leidet unter vorhersehbaren und unvorhersehbaren
	Messfehlern.

	Unvorhersehbare Messfehler ergeben sich durch elektromagnetische
	Störungen, die durch interne und externe Schaltkreise entstehen.
	\autocite[][S.\,422]{atmega128rfa1} Die Unvorhersehbarkeit besagt nicht,
	dass diese Fehler nur selten auftreten -- vielmehr treten sie sogar
	häufig auf.  Lediglich die konkrete Auswirkung des Fehlers ist
	nicht vorhersehbar.

	Diese Fehler können verringert werden, indem nicht benötigte Komponenten
	abgeschalten werden und  der \enquote{ADC Noise Reduction Mode}
	\autocite[][S.\,159]{atmega128rfa1} genutzt wird.  Üblich ist es auch
	mehrere (\zB 16 oder 128) Messungen direkt hintereinander durchzuführen
	und einen Mittelwert zu bilden.  Weitere Möglichkeiten sind hier
	auch in \autocite[][S.\,422]{atmega128rfa1} beschrieben.

	Dem stehen die vorhersehbaren Fehler entgegen.  Hier ist zum einen ein
	festgestellter systematischer Messfehler des ATmega128RFA1 zu nennen.
	So weichte der Messwert im unteren Wertebereich wenig und mit steigendem
	Wertebereich weiter ab. (siehe \autoref{fig:ad:messfehler} und \autoref{tbl:ad:messfehler})
	Der Fehler wurde noch nicht näher bestimmt und sollte weiter untersucht
	werden, ob dies von Modell zu Modell oder sogar von Gerät zu Gerät
	abweicht.  Letzteres wäre katastrophal, da sich dann für jedes einzelne
	Gerät die Notwendigkeit einer Kalibrierung ergeben würde.
	%\begin{figure}[!t]
\centering
\begin{circuitikz}[european resistors,node distance=0.3cm, european voltages]
\coordinate (vcc) at (0,0);
\coordinate (gnd) at (6,0);
\coordinate (r1s) at (0,-1);
\coordinate (between) at (3,-1);
\coordinate (r2e) at (6,-1);

\node [above of=vcc] {VCC};
\node [above of=gnd] {GND};
\draw (vcc)
	to [short, *-] (r1s)
	to [vR, l={\(R_1\)}] (between)
	to [R, l={\(R_2 \mathop{=} \unita{1{,}97}{\kilo\ohm}\)}] (r2e)
	to [short, -*] (gnd);

%\draw (vcc) to [open, v^>={\(U_{\text{ges}\)}] (gnd);
\draw (between)++(0,-.3cm) to [open, v_>={\(U_2\)}] ++(3cm,0);
\draw (vcc) to [open, v^>={\(U_\text{ges} \mathop{=} \unita{3{,}18}{\volt}\)}] (gnd);
\end{circuitikz}
\caption{AD-Beschaltung}
\label{fig:ad:beschaltung}
\end{figure}
%
	\begin{table}
\centering
\caption{Messwerte bei einer Referenzspannung von \unita{1{,}6}{\volt}}
\label{tbl:ad:messfehler}
\begin{tabular}{R{1}{3}rr}
\toprule
	\multicolumn{1}{c}{\thead{\(U_\text{anliegend}\) in \volt}}
	& \thead{erwartet}
	& \thead{gemessen}
	\tabularnewline
\midrule
	0,050	& 8	& 7	\tabularnewline
	0,091	& 15	& 13	\tabularnewline
	0,136	& 22	& 20	\tabularnewline
	0,169	& 27	& 26	\tabularnewline
	0,242	& 39	& 37	\tabularnewline
	0,314	& 50	& 48	\tabularnewline
	0,346	& 55	& 53	\tabularnewline
	0,516	& 82	& 80	\tabularnewline
	0,596	& 95	& 92	\tabularnewline
	0,799	& 127	& 124	\tabularnewline
	0,884	& 141	& 137	\tabularnewline
	1,020	& 163	& 159	\tabularnewline
	1,170	& 187	& 183	\tabularnewline
	1,268	& 202	& 198	\tabularnewline
	1,296	& 207	& 202	\tabularnewline
	1,353	& 216	& 211	\tabularnewline
\bottomrule
\end{tabular}
\end{table}
%
	\begin{figure}[!t]
\centering
\begin{tikzpicture}
%Achsen:
%\draw[->] (0,0) -- coordinate (xachse) (6,0);
%\draw[->] (0,0) -- coordinate (yachse) (0,6);

%\draw plot[scale=1/128, mark=*, mark options={fill=white}] file{dat/ad_messwerte_calc.dat};

\datavisualization [scientific axes,
	x axis={min value=0,
		max value=1.6,
		length=5cm,
		ticks={step=0.4, minor steps between steps=1},
		label={\(U_\text{anliegend}\) in \volt}},
	y axis={min value=0,
		max value=6,
		length=3cm,
		ticks={step=2, minor steps between steps=1},
		label={erwartet \(-\) gemessen}},
	all axes={grid},
	visualize as scatter]
	data {
		x,	y
		0.0505454252,	1
		0.0910023242,	2
		0.1360095528,	2
		0.1695885219,	1
		0.2426258714,	2
		0.3141725176,	2
		0.3468770764,	2
		0.516029654,	2
		0.5960608944,	3
		0.7990561224,	3
		0.8848305085,	4
		1.0202931596,	4
		1.170953271,	4
		1.2681376518,	4
		1.2967501552,	5
		1.3536300778,	5
		};
\end{tikzpicture}
\caption{Differenz zwischen Erwartungs- und Messwerten nach \protect\autoref{tbl:ad:messfehler}}
\label{fig:ad:messfehler}
\end{figure}
%

	Die Konvertierung der Spannung erfolgt durch sukzessive Approximation
	\autocite[][S.\,412]{atmega128rfa1} anhand einer Referenzspannung.
	Ergibt sich die zu messende Spannung aus der Betriebsspannung, so ist
	beim Batteriebetrieb des Sensorknotens ein Absinken der Betriebsspannung
	im Laufe der Betriebszeit zu beachten.

	Soll ein \emph{Aktor} mittels analoger Spannungswerte angesteuert werden,
	so kann dies durch Pulsweitenmodulation erreicht werden.  Auf dem
	ATmega128RFA1 können mit den Timern 0 und 2 Spannungswerte zwischen
	Ground und Vcc moduliert werden.

%-------------------------------------------------------------------------------
\subsubsection{Interrupts}
	Mittels Interrupts kann auf sich verändernde anliegende Signale
	an einem Eingangspin reagiert werden.  Der ATmega128RFA1 kann dabei
	auf steigende, fallend oder beide Flanken reagieren.  Zusätzlich kann er
	auch Interrupts auslösen, solange an dem Pin ein Low-Signal anliegt.
	\autocite[][S.\,220]{atmega128rfa1}

	Problematisch ist dabei, dass der Programmierer selber die Signale
	entprellen muss.  Außerdem wurde ein eigenartiges Verhalten
	der Interrupts 0 und 1 festgestellt.  Jeder Interrupt funktioniert für
	sich.
	Werden jedoch beide gleichzeitig verwendet, so kommt
	es zu folgender Situation:  Wird hardwareseitig der Interrupt
	0 ausgelöst, so wird einmal die Interrupt-Service-Routine
	des Interrupts 0 und zweimal die Interrupt-Service-Routine des
	Interrupts 1 ausgelöst.  Ebenso trat dieses Phänomen andersherum
	auf -- bei einem hardwareseitigen Auslösen des Interrupts 1
	wurde einmal die Interrupt-Service-Routine des Interrupts 1 und
	zweimal die Interrupt-Service-Routine des Interrupts 0 ausgelöst.
	Die Ursache für dieses Verhalten muss noch weiter untersucht werden.
	Dabei sollte auch geklärt werden, ob dies für die anderen Interrupts
	ebenso zutrifft.

	Um \emph{Aktoren} über Flanken oder High-/Low-Pegel zu steuern, können
	die Pins als Ausgang deklariert werden und dann bei Bedarf an- und
	ausgeschalten werden.


%===============================================================================
\subsection{Interfacebasierte Ansteuerung}
	Im Gegensatz zur in \autoref{sec:anst:direkt} beschriebenen direkten
	Ansteuerung, werden bei der interfacebasierten Ansteuerung die
	eigentlichen Messwerte von einem externen Schaltkreis bestimmt und
	dem Mikrocontroller nur noch über ein Interface zur Verfügung gestellt.
	Die Abstraktion zum physikalischen Problem der Messung ist also höher.

	Der ATmega128RFA1 bietet für die Schnittstellen UART, \ItC und SPI
	eine direkte Hardwareunterstützung.  Andere Schnittstellen müssen
	selber mit ihren Timings programmiert werden.  Über UART können nur
	zwei Geräte miteinander Daten austauschen. \ItC und SPI ermöglichen es,
	auf einem Anschluss (nacheinander) mit verschiedenen angeschlossenen
	Sensoren und Aktoren zu kommunizieren.

	Bei den Protokollen kann man zwischen Binär- und Textprotokollen
	unterscheiden.
%-------------------------------------------------------------------------------
\subsubsection{Binärprotokolle}
	Binärprotokolle sind relativ kompakt.  Der Master schickt einen
	Befehl\,/\,eine Anforderung an den Sensor bzw. Aktor, der dann direkt
	antwortet.


%-------------------------------------------------------------------------------
\subsubsection{Textprotokolle}
	Textprotokolle sind vielseitiger als Binärprotokolle.	Dafür
	benötigen die Befehle und Antworten mehr Bytes bei der Übertragung.
	Die Übertragung dauert also länger und verbraucht damit mehr Energie.
	Zusätzlich muss die Nachricht (umständlich) geparst und Zahlen von
	ASCII-Code in Integerwerte (und umgekehrt) überführt werden, um sie
	weiter verarbeiten zu können.	Auch dieses Parsen erhöht sowohl
	den Energiebedarf als auch den Speicherbedarf für Zeichenklassen-
	und Zustandsübergangstabellen.  Wenn außerdem die Länge der
	einzelnen Nachrichten nicht explizit durch das Protokoll begrenzt
	ist, muss dynamisch Speicher zum Puffern der eingehenden Nachricht
	reserviert werden, was sich ungünstig auf speicherarme Systeme wie
	Mikrocontroller auswirkt.

