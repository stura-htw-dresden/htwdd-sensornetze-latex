\section{Ansteuerung auf Anwendungsebene}
\label{sec:coap}

	In einem Hausautomatisierungssystem sollen Sensorinformationen
	sowie Steuerbefehle für Aktoren übertragen werden.
	Daher ist es erforderlich, auf den Netzknoten eine Anwendungsschicht
	zu besitzen, die diese Informationen bzw. Befehle
	versenden und empfangen kann.

	An dieser Stelle geht es also nicht um die Umsetzung der genauen Befehle an
	die Sensoren und Aktoren oder die Programmierabstraktion,
	sondern um eine sinnvolle Abstraktion auf Netzwerkebene,
	so dass andere Netzknoten alle nötigen Informationen zur
	Weiterverarbeitung vorliegen haben.

	Wichtige Informationen neben dem aktuellen Sensorwert
	können Folgende sein:
	\begin{itemize}
	\item 	der Name des Sensors,
	\item 	der Typ des Sensors (\zB Temperatur, Lichtintensität)
	\item 	den möglichen Wertebereich,
	\item 	Ungenauigkeiten.
	\end{itemize}

	Ein Temperatursensor liefert einen Temperaturwert im Bereich
	von \unita{-40}{\celsius} bis \unita{+100}{\celsius}.
	Dem Regelverarbeitungssystem genügt es unter Umständen,
	dass nur (starke) Temperaturschwankungen aufgezeigt werden,
	\zB durch das Öffnen eines Fensters an kalten Tagen.
	Damit ist es überflüssig, dass jede Sekunde der Sensorwert
	durch die Regelverarbeitung gepollt\footnote{\enquote{Pollen} bezieht sich
		in diesem Abschnitt auf die Kommunikation zwischen
		Regelverarbeitungsserver und Sensorknoten, nicht aber auf ein
		Pollen zwischen Sensorknoten und Sensor.}
	wird, da ein Fenster
	an kalten Tagen nur selten geöffnet wird.

	Das Pollen der Sensorwerte, also der Temperaturwerte,
	bewirkt folglich ein Überfluten des Sensornetzes
	mit \iA nicht benötigten Informationen.

	Hierfür müssen Mechanismen zur Ansteuerung auf Anwendungsebene
	bereitgestellt werden, die die Netzlast reduzieren und somit Energie
	für die Übertragung und Auswertung der Nachrichten eingespart werden
	kann.

	Daneben gibt es auch Situationen, in denen eine Änderung der
	Sensorinformation zeitkritisch verarbeitet werden muss,
	das heißt, es kann nicht abgewartet werden, bis die Regelverarbeitung
	das nächste Mal den Sensorwert pollt.
	Hierzu muss der Regelverarbeitungsserver selbstständig
	benachrichtigt werden, damit eine schnelle Abarbeitung erfolgen kann.

\subsection{CoAP}
	Ein Anwendungsprotokoll, dass die Anforderungen der Ansteuerung
	auf Anwendungsebene erfüllen kann, ist CoAP. Es ist ein asynchrones
	Protokoll, weshalb UDP als Transfer Protokoll ausreicht.  Da CoAP sich
	noch in Entwicklung befindet, ist keine Kompatibilität zu vorherigen
	Draft-Versionen sichergestellt.

	Zum Zeitpunkt des Forschungsprojekts befand sich CoAP bei Draft-12 bzw.
	-13. Die implementierten Draft-Versionen von CoAP waren meist auf
	unterschiedlichen Ständen.  So bot Contiki 2.5 bei Projektbeginn nur
	Draft-03. Nach Veröffentlichung von Contiki 2.6 war auch Draft-07
	implementiert.
	\autocite{kovatsch11low-power}
	
	Seitens des Hausautomatisierungsservers sah es ähnlich aus: Die
	CoAP-Implementation auf dem Hausautomatisierungsserver (CoAPy) wurde
	der Contiki-Version angepasst, um eine Kommunikation herzustellen.

	Das Anwendungsprotokoll CoAP bietet verschiedene Vorteile
	für die Hausautomatisierung im Allgemeinen und für die Ansteuerung
	der Sensoren und Aktoren im Konkreten. Diese Vorteile werden
	in den folgenden Abschnitten näher erläutert.

	Weitere Informationen zu CoAP sind der Projektgruppe
	\enquote{Steuerung des Systems} sowie im Speziellen zu Contiki
	der Gruppe \enquote{Sensornetz-Kommunikation} zu entnehmen.

\subsection{Kommunikationsbeispiele}
	Mit dem Discovery Mechanismus von CoAP ist es möglich, die
	Ressourcen des Sensorknotens zu erfahren.
	Ein Sensorknoten mit dem Sensor SHT21 kann sowohl die Luftfeuchtigkeit
	wie auch die Temperatur messen.
	Damit können für diesen Sensorknoten zwei
	REST-Ressourcen angelegt werden: \lstinline=sensor/sht21/temperature=
	und \lstinline=sensor/sht21/humidity=.
	Der Sensorknoten könnte aber auch noch andere Ressourcen bieten,
	wie eine Einstellungsoption, die es erlaubt,
	den Radio Duty Cycle zu verändern oder Einstellungen am SHT21
	vorzunehmen.

	Es können zwei grundsätzliche Verfahren unterschieden werden,
	um auf ein Request, der an den Sensorknoten (CoAP-Server) gerichtet
	ist, zu antworten:
	\begin{LaTeXdescription}
	\item[\normalfont\itshape Piggy-backed Response]
		Bei einem Request wird zuerst der
		Sensorwert ausgelesen und dann zusammen mit dem ACK
		als Response übertragen werden. Hierbei wird der
		Netzwerktraffic reduziert, erfordert aber eine schnelles
		Auslesen des Sensorwertes.
	\item[\normalfont\itshape Separate Response]
		Die andere Möglichkeit ist es, sobald ein Request eingetroffen
		ist, den Empfang mit einem ACK zu bestätigen.
		Danach wird der Sensorwert erfasst und als Response zum
		CoAP-Client geschickt, der dann den Empfang mit einem ACK
		bestätigt.
		\label{sec:coap:separateresponse}
	\end{LaTeXdescription}

	Den Zustand einer LED sollte folglich nur mit einem
	\emph{Piggy-backed Response} beantwortet werden, um den Netzwerktraffic
	gering zu halten.

	Für das Auslesen des Feuchtigkeitssensors SHT21 wird je
	nach \ItC-Bus-Geschwindigkeit etwa \unita{50}{\micro\second} benötigt.
	Hier kann man je nach Anwendungsfall entscheiden, welche
	Response-Methode besser ist.

	Für den Lichtsensor TSL2561 eignet sich im Prinzip nur die zweite
	Variante, denn dieser benötigt durchaus
	\unita{400}{\milli\second} zum Messen der Lichtintensität.
	Es gäbe zwar die Möglichkeit eine kürzere Messdauer einzustellen,
	allerdings kann dadurch der Messfehler sich erhöhen.
	Zur Beantwortung des Requests der aktuellen Lichtintensität
	ist ein separates Response wichtig, denn der CoAP-Client
	-- \idR ist dies der Regelverarbeitungsserver --
	wird im Falle eines Piggy-backed-Response mehrere Requests losschicken,
	da er davon ausgehen muss, dass die Nachricht nicht beim CoAP-Server
	angekommen ist.

	Bei zeitkritischen Ereignissen -- wie das Öffnen eines Fensters oder
	der Wohnungstür -- möchte der CoAP-Client nicht ständig den
	Sensorknoten pollen müssen. Daher bietet CoAP auch die Möglichkeit,
	eine Ressource zu überwachen, das Observe-Feature.
	Bei diesem Feature teilt ein CoAP-Client dem Sensorknoten mit, dass
	der Client selbstständig informiert werden soll,
	sobald sich der Zustand einer Ressource ändert.

	In Contiki sind in der REST-Implementierung für dieses Observe-Feature
	zwei Ressource-Typen zur Verfügung gestellt worden:
	Der erste Typ ist eine periodische Ressource, die folglich periodisch
	den Zustand der Ressource prüft und \idR nach Überschreiten eines
	Schwellwertes eine Response-Nachricht an alle eingetragenen
	CoAP-Clients verschickt. Damit kann das Sensornetz entlastet werden, da
	nicht jeder Sensorwert übertragen wird.
	Der andere Typ ist eine eventbasierte Ressource, die aufgrund des
	Eintretens eines vorher definierten Events eine Response-Nachricht
	ausgibt.  Das Event kann beispielsweise das Drücken eines Buttons sein
	oder aber ein PUT-Request an eine andere Ressource.

	Bezogen auf den Türkontaktsensor kann nun eine Eventressource angelegt
	werden: Beim Öffnen der Tür wird dann der Sensorknoten per Interrupt
	aus seiner Schlafphase erwachen und sofort eine CoAP-Nachricht
	an alle eingetragenen CoAP-Clients verschicken, um anzuzeigen,
	dass die Tür geöffnet wurde.


\subsection{CoAP Schnittstelle}
	Da ein paar Kommunikationsbeispiele genannt wurden, soll
	nun die CoAP Schnittstelle in Contiki beispielhaft vorgestellt werden.

	Die eventbasierte sowie die periodische Ressource werden hierbei nicht
	als Quellcode vorgestellt, da die Möglichkeiten hierbei sehr vielfältig
	sein können und dieses Observe-Feature noch nicht vollständig im
	Projekt untersucht werden konnte.

	Der REST Server kann in Contiki als
	Prozess\footnote{Prozesse sind in Contiki Protothreads
		\autocite{dunkels05using, dunkels06protothreads, dunkels07simplifying}}
	gestartet werden.
	Die Initialisierungsphase des Servers,
	welche in \autoref{lst:restserverprocess} gezeigt ist,
	erfolgt in drei Schritten:
	\begin{enumerate}
	\item 	Zuerst müssen alle Komponenten in den Ressourcen initialisiert werden,
		sofern dies noch nicht geschehen ist.
	\item 	Danach wird die REST Engine initialisiert und
	\item 	zuletzt werden alle benötigten Ressourcen
		aktiviert.
	\end{enumerate}

	% CoAP REST Server -- Listing
	\lstinputlisting[style=float,language=c,%
		label=lst:restserverprocess,%
		caption={CoAP REST Serverprozess},%
	]{listing/coap-server-process.c}

	Damit Ressourcen aktiviert werden können, müssen diese zuvor
	auch angelegt werden. In Contiki geschieht dies mit dem Makro
	\lstinline=RESOURCE=. Der erste Parameter definiert den Namen
	der Ressource, welcher zugleich auch im Handler vorkommt.
	Der zweite Parameter kennzeichnet, welche \emph{REST-Method-Types}
	von der Ressource unterstützt werden. Aus HTTP bekannte Method-Types
	sind
	\lstinline=METHOD_GET=,
	\lstinline=METHOD_POST=,
	\lstinline=METHOD_PUT=,
	\lstinline=METHOD_DELETE=.
	Auf eine Beschreibung wird an dieser Stelle verzichtet und auf die
	Dokumentation von CoAP \autocite{kovatsch11low-power} verwiesen.
	Der dritte Parameter gibt den URI-Pfad an und der letzte
	die Attribute. Es besteht die Möglichkeit, eine Ressource
	auch als Subressource zu definieren.

	In \autoref{lst:rest_res_text} ist eine Text-Ressource gezeigt.
	Dieser Ressource kann ein CoAP-Client eine Wunschlänge
	per Query-Variable übergeben.

	% CoAP REST Resource -- Text -- Listing
	\lstinputlisting[style=float,language=c,%
		label=lst:rest_res_text,%
		caption={Text Ressource},%
	]{listing/coap-server-res-text.c}

	Bei einer einfachen Sensor-Ressource genügt die GET-Methode, so dass
	ein Client die Ressource lediglich auslesen kann. Die Piggy-backed
	Variante ist in \autoref{lst:rest_res_sensor} gezeigt.

	% CoAP REST Resource -- Sensor -- Listing
	\lstinputlisting[style=float,language=c,%
		label=lst:rest_res_sensor,%
		caption={Sensor Ressource},%
	]{listing/coap-server-res-sensor.c}

	Eine Erweiterung des in \autoref{lst:rest_res_sensor} gezeigten
	Beispiels könnte die Einführung einer Query-Variable sein,
	welche anzeigt, ob der Temperaturwert in Grad Celsius
	oder Grad Fahrenheit übertragen werden soll. Hierzu sind keine zwei
	Ressourcen notwendig.

	Soll die Observe Funktion von CoAP genutzt werden, dann stehen hierzu
	zwei weitere Ressource-Typen bereit, die die normale Ressource
	erweitern: Die periodische sowie die eventbasierte Ressource.
	Beide Ressourcetypen haben einen zusätzlichen Handler.
	Ein CoAP-Client kann dieser Ressource überwachen und wird damit
	in eine Liste aufgenommen.

	In diesem Abschnitt wurde lediglich ein Einblick in die
	CoAP-Schnittstelle gegeben. Die Implementation der CoAP-Funktionalität
	ist in Contiki zweigeteilt: Die Ressourcen und die REST Engine
	werden von einem \emph{Abstraction Layer} bereitgestellt, der sich
	in Contiki \emph{Erbium} nennt und sich im Apps-Ordner befindet.
	Dieser stellt die \emph{RESTful Web services} zur Verfügung.

	Zur Kompilierzeit wird bestimmt, welche CoAP-Implementation genutzt
	wird. Die Implementationen sind ebenfalls im Apps-Ordner von Contiki
	zu finden: Es ist die Draft-Version 03 (er-coap-03)
	und 07 (er-coap-07) implementiert.

	Sowohl Abstraction Layer wie auch die beiden Draft-Implementationen
	stammen vom selben Autor, Matthias Kovatsch. Ziel der Implementation
	ist ein Low-Power-CoAP \autocite{kovatsch11low-power} für Contiki
	bereitzustellen.

\subsection{Zusammenhang zwischen CoAP und RDC}
	Mit dem im vorherigen Absatz genannten Beispiel des Türkontaktsensors
	kann man einen Zusammenhang zwischen CoAP und dem Radio Duty Cycle (RDC)
	erkennen:

	Angenommen, die Tür wird im Normalfall nur alle drei Monate einmal
	geöffnet und der Sensorknoten arbeitet zuverlässig, so dass ein Ausfall
	ausgeschlossen werden kann. Dann ist es nicht von Vorteil, dass der
	RDC besagt, dass der Knoten alle \unita{8}{\hertz} aufwacht, um
	empfangsbereit zu sein. Die hierfür benötigte Energie könnte durchaus
	eingespart werden, so dass sich die Lebenszeit des Sensorknotens
	drastisch erhöht. Der Nachteil ist natürlich, dass der Sensorknoten
	während einer langen Zeit nicht erreichbar ist. Wenn man nur die
	Zustände \enquote{offen} und \enquote{geschlossen} überträgt, könnte
	der Fall eintreten, dass aufgrund eines Ausfalls oder anderer Gründe
	der letzte Zustand nicht bekannt ist. Das nächste Update erfolgt
	allerdings nicht in nächster Zeit.
	
	Hierfür bietet CoAP eine Lösung an: Man kann CoAP-Proxies verwenden,
	die die Sensorwerte cachen und somit im Namen der schlafenden
	Sensorknoten antworten können.

	Wenn diese Idee weiterverfolgt wird, so erscheint es naheliegend,
	auch einen unregelmäßigen RDC zu verwenden, wie im Falle
	des Türkontaktsensors. Normalerweise wird in einem Sensornetz davon
	ausgegangen, dass alle Netzknoten den gleichen RDC
	besitzen. Eine vernünftige Zusammenarbeit von CoAP, dem Routing
	Protokoll sowie dem RDC Protokoll kann bewirken,
	dass verschiedene RDCs im gleichen Sensornetz
	zum Einsatz kommen. Diese Theorie gilt es noch zu prüfen.
