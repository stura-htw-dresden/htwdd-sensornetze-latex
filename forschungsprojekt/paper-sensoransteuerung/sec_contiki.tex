\section{Konzepte der Ansteuerung in Contiki-OS}

	Zielplattformen von Contiki-OS sind eingebettete,
	batteriebetriebene Systeme, die über Funk (oder andere Netzwerke)
	miteinander kommunizieren. \autocite{contiki, dunkels04contiki}

	Contiki bietet eine Reihe von Modulen, die es für den Einsatz
	in einem drahtlosen Sensornetz prädestiniert.
	Für die Ansteuerung der Sensoren ist das Sensorinterface sehr
	interessant, weshalb es näher beleuchtet wird.

	Da im Sensorinterface die Sensoren physikalisch angesteuert werden
	müssen und im Forschungsprojekt viele \ItC-Sensoren verwendet wurden,
	wird ebenfalls ein \ItC-Interface und seine Eigenheiten betrachtet.

	Bevor allerdings das Sensor- bzw. \ItC-Interface erläutert wird,
	soll die Struktur von Contiki-OS \autocite{contikidoc} kurz vorgestellt werden,
	um zu verstehen, wo sich welches Interface bzw. welche Implementation
	befindet.

\subsection{Struktur von Contiki-OS}
	Da Contiki viele verschiedene Plattformen unterstützen möchte,
	wird generischer Code von plattformabhängigem Code (strikt) getrennt.
	Dabei gliedert sich Contiki in folgende Abschnitte:
	\begin{LaTeXdescription}
	\item[\normalfont\itshape core]
		Der Kern von Contiki arbeitet \emph{plattformunabhängig}.
		Der Netzwerk-Stack ist beispielsweise bis auf die Ansteuerung
		des Funksensors komplett enthalten. Das Interface für die
		Ansteuerung des Funksensors befindet sich aber wiederum
		im Kern.
	\item[\normalfont\itshape cpu]
		In diesem Abschnitt befindet sich \emph{CPU-abhängiger Code}.
		Es wird zwischen AVR, ARM und weiteren CPUs unterschieden.
		Hier ist \iA die interfacebasierte physikalische Ansteuerung
		implementiert, da Funktionalitäten des Mikrocontrollers genutzt
		werden. Zugehörige Schnittstellen können sich im Kern befinden.
	\item[\normalfont\itshape platform]
		An dieser Stelle werden \emph{plattformspezifische
		Eigenschaften}
		umgesetzt. Vor allem befindet sich die Main-Methoden in diesem
		Bereich. Eine Plattform kann auch für einen bestimmten
		Mikrocontroller wie den ATmega128RFA1 stehen.
	\item[\normalfont\itshape apps]
		Wenn Funktionalitäten nicht im Kern benötigt werden und sie
		dennoch plattformunabhängig sind, so kommen diese hier hin.
		Die CoAP/REST-Implementierung befindet sich beispielsweise
		hier, da es sich nur um eine Anwendungsschicht im
		Netzwerk-Stack handelt.
	\item[\normalfont\itshape examples / project]
		Der letzte Bereich organisiert ein spezielles Projekt.
		An dieser Stelle wird festgelegt, welche Plattform verwendet
		wird und welche Prozesse automatisch gestartet werden.
		Ein spezielles Projekt könnte ein Sensorknoten sein, welcher
		Sensoren zum Auslesen der Raumtemperatur sowie der
		Luftfeuchtigkeit bietet und diese Informationen über Funk
		(6LoWPAN) verteilen kann.
	\end{LaTeXdescription}


\subsection{I2C-Interface}
	Im Forschungsprojekt wurden vor allem Sensoren verwendet, die
	eine interfacebasierte Ansteuerung über \ItC erlauben.

	\ItC\footnote{\ItC wird im AVR Reference Manual \autocite{atmega128rfa1}
		auch Two-Wire Interface (TWI) genannt.}
	ist eine serielle Schnittstelle, die den Anschluss mehrerer
	\ItC-Devices ermöglicht. So konnte sowohl ein Luftfeuchtigkeits-,
	ein Druck-, ein Beschleunigungs- sowie ein Lichtsensor an einem Bus
	angeschlossen werden. Am Mikrocontroller werden für \ItC nur zwei Pins
	(\ItC-SCL, \ItC-SDA) benötigt. Der Mikrocontroller fungiert hierbei als
	Master und leitet die Übertragung der Sensordaten ein.

	In Contiki wurde kein \ItC-Interface gefunden. Da \ItC keine generische
	Implementation aufgrund der Verwendung von Mikrocontroller-Komponenten
	erlaubt, wurde ein eigenes \ItC-Interface in den Kern von Contiki
	integriert. Eine zugehörige Implementation erfolgte dann im
	CPU-abhängigen Teil von Contiki.

	Dieses \ItC-Interface erlaubt die Verwendung des \ItC-Busses,
	ohne detaillierte Kenntnisse zum Ablauf der Kommunikation zu haben.
	Allerdings ist der Multi-Master-Betrieb nicht berücksichtigt worden,
	da dieser für die Ansteuerung von Sensoren nicht benötigt wird.
	Das Interface ist in \autoref{lst:i2c-interface} zu sehen und zeigt
	die drei benötigten Funktionen zum Ansteuern eines Sensors
	sowie die möglichen Übertragungsfehler.

	% I2C-Interface -- Listing
	\lstinputlisting[style=float,language=c,%
		label=lst:i2c-interface,%
		caption={Das \ItC-Interface (Ausschnitt)},%
	]{listing/i2c-interface.h}

	Die AVR-Implementation des \ItC-Interfaces benutzt den vorgesehenen
	\ItC-Interrupt, um die Übertragung zügig abarbeiten zu können.
	An welchem Pin sich der \ItC-Bus befindet, wird durch die AVR-Libc
	bestimmt.

	Die AVR-Implementation sieht nur eine synchrone Übertragung vor,
	das heißt, dass das Betriebssystem so lange warten muss, bis die Übertragung
	abgeschlossen oder ein Fehler aufgetreten ist. Durch die Verwendung
	des \ItC-Interrupts wird die Übertragung sogar bevorzugt abgearbeitet
	bzw. kann \idR nicht durch andere Interrupts unterbrochen werden.

	\ItC-Sensoren können in Bezug auf das Übertragungsverhalten
	in zwei Klassen eingeteilt werden:
	Die ersten verwenden einen
	Hold-Mode\footnote{Der Sensor \emph{hält} den \ItC-Bus während der
		Messung des Sensorwertes, so dass die Kommunikation nicht
		abgeschlossen oder eine andere nicht begonnen werden kann.
		Der Feuchtigkeitssensor SHT21 unterstützt sowohl den Hold-
		wie auch den No-Hold-Mode.}
	und die anderen einen No-Hold-Mode.
	
	\begin{LaTeXdescription}
	\item[\normalfont\itshape Hold-Mode]
		Dieses synchrone Übertragungsverhalten hat bei \ItC-Sensoren,
		die im so genannten Hold-Mode arbeiten, den Nachteil, dass
		während des Messens des Sensorwertes der \ItC-Bus blockiert ist
		und damit auch das Betriebssystem warten muss, bis die
		Übertragung abgeschlossen ist.
		
		Insgesamt wird mehr Energie
		verbraucht. Auf der anderen Seite werden keine Mechanismen
		gebraucht, die anzeigen, dass eine Übertragung bereits
		stattfindet.

	\item[\normalfont\itshape No-Hold-Mode]
		Die zweite Möglichkeit der Übertragung ist der No-Hold-Mode.
		Bei diesem wird dem Sensor mitgeteilt, dass ein Sensorwert
		gemessen werden soll. Der Mikrocontroller muss dann beim Sensor
		nach einer gewissen Messdauer den Sensorwert abfragen (pollen),
		sofern er bereits vorliegt.
		
		Viele Sensoren bieten für den No-Hold-Mode auch eine
		Interrupt-Leitung an, die anzeigt, wann der neue Sensorwert
		ausgemessen ist und somit abgefragt werden kann.

		Auch hier ist das synchrone Übertragungsverhalten sinnvoll, da
		der jeweilige Prozess die Übertragung einleitet und damit auch
		sofort das Ergebnis der Übertragung vorliegen hat.  Nach einer
		bestimmten Zeit oder beim Eintreten eines externen Interrupts
		wird dann der Sensorwert abgerufen.
	\end{LaTeXdescription}

	Abschließend ist zu sagen, dass es ratsam ist,
	zur Ansteuerung der \ItC-Sensoren ein Sensorinterface zu benutzen.
	Mit diesem lassen sich dann alle konfigurierbaren Optionen einstellen,
	wie den Hold- oder No-Hold-Mode.

\subsection{Das generische Sensorinterface}

	Contiki besitzt ein \emph{generisches} Sensorinterface, um Sensoren
	ansteuern zu können. Damit ist eine Abstraktion geschaffen, die
	die Entwickler der Sensortreiber dazu bewegen soll, ein einheitliches
	Schema zu nutzen.
	Entscheidend bei diesem Interface sind drei Funktionen:
	\begin{itemize}
	\item 	Die erste Funktion (\lstinline=configure=)
		konfiguriert den Sensor;
	\item 	die zweite (\lstinline=status=)
		gibt Informationen zum Sensor zurück,
		beispielsweise die konfigurierten Einstellungen
		oder den Zustand des Sensors;
	\item 	die dritte (\lstinline=value=)
		kann die Sensorwerte auslesen,
		beispielsweise den gemessenen Temperaturwert oder die
		Lichtintensität.
	\end{itemize}
	Zusammen mit einen Namen für den Sensor bilden diese vier Bestandteile
	den Sensortreiber.

	Eine frühe Version des Interfaces erforderte weitere Funktionen
	vom Sensortreiber: Hierunter war eine Initialisierungs-, 
	Aktivierungs- und Deaktivierungsfunktion. Diese Funktionen können aber
	durch die Konfiguration des Sensors übernommen werden,
	weshalb sie aus dem Interface entfernt wurden.

\subsubsection{Das Konfigurieren}
	Die Funktion \lstinline=configure= dient dem Konfigurieren.
	Sie besitzt zwei Parameter, wovon der erste den Konfigurationstyp
	und der zweite den Konfigurationswert angibt.
	Ein Rückgabewert kann anzeigen, ob die Konfiguration erfolgreich war.

	Was bei einem Sensor konfiguriert werden kann und was sinnvoll
	zu konfigurieren ist, muss jeder Sensortreiber für sich entscheiden.
	Es gibt lediglich die Begrenzung, dass die beiden Parameter sowie 
	der Rückgabewert Integers sind (und damit abhängig vom Mikrocontroller
	durchaus unterschiedlich groß sein können).

	Das Sensorinterface definiert globale Konfigurationstypen,
	wovon der wichtigste \lstinline=SENSORS_ACTIVATE= ist.
	Dieser sollte das Aktivieren und Deaktivieren des Sensortreibers
	und sinnvoller Weise auch des Sensors bewirken.
	Beispielsweise kann der Sensor mit Hilfe dieses
	Konfigurationsparameters in den Sleep-Mode versetzt werden.

\subsubsection{Der Auslesen der Status}
	Die Funktion \lstinline=status= dient dem Auslesen der Status
	eines Sensortreibers. Diese Funktion gibt den Status
	als Rückgabewert zurück. Der gewünschte Status wird wie bei der
	Konfiguration per Parameter übergeben. Es ist daher nicht undenkbar,
	dass die beiden Parameter, Konfigurationstyp und Statustyp,
	dieselben Wertigkeiten behandeln.

	Auch hier ist es sehr zu empfehlen,
	bei Übergabe von \lstinline=SENSORS_ACTIVATE= den Zustand des
	Sensortreibers/Sensors zurückzugeben, wobei \enquote{0} bedeutet, dass
	der Sensor aus ist. Eine \enquote{1} gibt die Aktivität des Sensors an.

\subsubsection{Das Auslesen von Sensorinformationen}
	Die Funktion \lstinline=value= bewirkt, dass Sensorinformationen
	ausgelesen werden. Der übergebene Parameter kennzeichnet
	den auszulesenden Sensorwert, beispielsweise einen Temperaturwert.

	Da der Rückgabetyp vom Typ Integer\footnote{Ein Integer ist auf dem
		ATmega128RFA1 nur \unita{8}{\bit} breit. Auf anderen Architekturen
		kann ein Integer breiter sein.}
	ist, ist es nicht möglich jeden beliebigen Sensorwert zurückzugeben.
	Es gibt Sensorwerte, die einen größeren Wertebereich benötigten.
	Ein Beispiel sind Temperaturwerte:
	Da die Typen Float oder Double auf Mikrocontrollern so oft wie möglich
	vermieden werden sollten, wird die Temperatur \unita{22.53}{\celsius}
	typischerweise als ganze Zahl (\zB \(2253\)) übergeben -- was aber den
	Wertebereich einer Integerzahl (0--255) auf dem ATmega128RFA1
	überschreitet.

	Hinzu kommt, dass ein fehlerhaftes Verhalten
	ebenfalls im Rückgabewert eingebunden werden muss.
	Ein durchaus oft auftretender Fehler bei batteriebetriebenen
	Netzknoten resultiert aus einer zu niedrigen Betriebsspannung:
	Der Mikrocontroller ATmega128RFA1 kann noch mit \unita{1.8}{\volt}
	arbeiten, aber einige Sensoren wie der SHT21 stellen bei \unita{2.1}{\volt}
	den Betrieb ein, so dass ein Auslesen des Sensorwerts zwangsweise
	fehlschlagen wird.

	Aus diesem Grund empfiehlt es sich, dass der Rückgabetyp kennzeichnet,
	ob ein Fehler aufgetreten ist oder nicht. Bei bereits vorhandene
	Sensorimplementationen wurde \enquote{0} für fehlerhaftes
	sowie \enquote{1} für fehlerfreies Verhalten des Sensors verwendet.
	Prinzipiell ist dieses (bisher) nicht festgelegt, so dass auch eine
	\enquote{0} für fehlerfreies Verhalten stehen könnte und ein
	von Null verschiedener Wert für den Fehlertyp.

	Eigene Tests haben gezeigt, dass es sich empfiehlt, den Sensorwert im
	Sensortreiber zwischenzuspeichern.  Dieses kann auch in einem
	RAW-Format passieren, so dass erst das Auslesen des
	zwischengespeicherten Sensorwerts diesen in das gewünschte Format
	konvertiert.

	Ein weiterer Vorteil ist es, dass der Sensortreiber alle
	Sensorwerte intern in einer beliebigen Datenstruktur ablegen kann, so
	dass auch historische Werte -- ggf. mit Zeitstempel -- abrufbar sind.
	Der Sensortreiber stellt dann die benötigten Funktionen zum Auslesen
	der Sensorwerte bereit.

	Zur Zeit ist keine Abstraktion zum Übergeben des Sensorwertes vom
	Sensorinterface zum Aufrufer vorgesehen.
	Es wäre also interessant, ob solch eine Abstraktion in Bezug zur
	Hausautomatisierung existiert und wie das vorhandene, generische
	Sensorinterface erweitert werden kann, um die Anforderungen
	der Ansteuerung auf Anwendungsebene zu erfüllen.
