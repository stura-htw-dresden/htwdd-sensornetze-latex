\section{Ausblick}
	Die durchgeführten exemplarischen Implementationen haben geholfen,
	einen Überblick über die Möglichkeiten zur Anbindung von Sensoren
	bzw. Aktoren unter Contiki zu entwickeln.  Dabei sind allerdings
	verschiedene Probleme aufgetreten, die in nachfolgenden Arbeiten
	untersucht und gelöst werden müssen:
	\begin{itemize}
	\item	Die Messungenauigkeiten des Analog-Digital-Wandlers des
		ATmega128RFA1 sowie das ungewöhnliche Verhalten der
		Interrupts 0 und 1, beides in \autoref{sec:anst:direkt}
		beschrieben, müssen untersucht und gelöst werden.
	\item	Die CoAP-Schnittstelle Contikis muss dahingehend untersucht
		werden, wie die in \autoref{sec:coap} angesprochenen
		eventbasierten und periodischen Ressourcen konkret umgesetzt
		werden können.  Auch die Verwendung des Separate Response
		unter Contiki muss getestet werden.
	\item	Die Stromsparmechanismen von Contiki
		\autocite{ritter05experimental, dunkels07softwarebased}
		müssen genutzt werden.
		Dazu sind unter anderem die Schlafmodi des ATmega128RFA1 zu
		implementieren und ein Konzept zu entwickeln, wie die Prozesse
		dem Betriebssystem mitteilen können, ob ein Schlafmodus
		zulässig ist oder nicht.
	\end{itemize}
