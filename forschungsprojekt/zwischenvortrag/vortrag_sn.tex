%%%%%%%%%%%%%%%%%%%%%%%%%%%%%%%%%%%%%%%%%%%%%%%%%%%%%%%%%%%%
%% filename:	vortrag.tex
%% template:	Mon, 07 May 2012 13:01:14 +0200
%% author:	Hermann Lorenz
%% date:	20. Nov 2012 10:00
%%%%%%%%%%%%%%%%%%%%%%%%%%%%%%%%%%%%%%%%%%%%%%%%%%%%%%%%%%%%
\documentclass[%
	ngerman%
	]{beamer}

%\setbeameroption{show only notes} 	% note{} -- Anmerkungsfolien


%%%%%%%%%%%%%%%%%%%%%%%%%%%%%%%%%%%%%%%%%%%%%%%%%%%%%%%%%%%%
%% Lokalisierung %%%%%%%%%%%%%%%%%%%%%%%%%%%%%%%%%%%%%%%%%%%
%%%%%%%%%%%%%%%%%%%%%%%%%%%%%%%%%%%%%%%%%%%%%%%%%%%%%%%%%%%%
\usepackage[utf8]{inputenc}	% Umlaute direkt eingeben
\usepackage[T1]{fontenc}	% Wörter mit Umlaute umbrechen
\usepackage[ngerman]{babel}	% deutsche Bezeichner
\usepackage[babel,german=guillemets]{csquotes}	% \enquote{}
\usepackage{libertine}
\usepackage[scaled=0.83]{beramono}


%%%%%%%%%%%%%%%%%%%%%%%%%%%%%%%%%%%%%%%%%%%%%%%%%%%%%%%%%%%%
%% Bilder %%%%%%%%%%%%%%%%%%%%%%%%%%%%%%%%%%%%%%%%%%%%%%%%%%
%%%%%%%%%%%%%%%%%%%%%%%%%%%%%%%%%%%%%%%%%%%%%%%%%%%%%%%%%%%%
\usepackage{graphicx}	% \includegraphics{bild.pdf}
\usepackage{tikz}

%-----------------------------------------------------------
% TikZ library: arrows, positioning

\usetikzlibrary{arrows,positioning}
\tikzset{
    %Define standard arrow tip
    >=stealth',
    %Define style for boxes
    punkt/.style={
           rectangle,
           rounded corners,
           draw=black, very thick,
           text width=6.5em,
           minimum height=2em,
           text centered},
    % Define arrow style
    pil/.style={
           ->,
           thick,
           shorten <=2pt,
           shorten >=2pt,}
}
\usetikzlibrary{fit}

%%%%%%%%%%%%%%%%%%%%%%%%%%%%%%%%%%%%%%%%%%%%%%%%%%%%%%%%%%%%
%% pdf-links %%%%%%%%%%%%%%%%%%%%%%%%%%%%%%%%%%%%%%%%%%%%%%%
%%%%%%%%%%%%%%%%%%%%%%%%%%%%%%%%%%%%%%%%%%%%%%%%%%%%%%%%%%%%
\usepackage[ngerman]{varioref}	% \vpageref{}

%%%%%%%%%%%%%%%%%%%%%%%%%%%%%%%%%%%%%%%%%%%%%%%%%%%%%%%%%%%%
%% eigene Macros %%%%%%%%%%%%%%%%%%%%%%%%%%%%%%%%%%%%%%%%%%%
%%%%%%%%%%%%%%%%%%%%%%%%%%%%%%%%%%%%%%%%%%%%%%%%%%%%%%%%%%%%
\usepackage{xspace}

\newcommand{\zB}{z.\,B.\xspace}
\newcommand{\conclusion}{\(\to\)\xspace}

\newcommand{\todo}[1]{\textcolor{red}{TODO: #1}}
\newcommand{\prove}[1]{\textcolor{red}{TODO/PROVE(#1)}}
%\newcommand{\todo}[1]{}
%\newcommand{\prove}[1]{}

\definecolor{lgray}{gray}{.3}
\newcommand{\notall}[1]{\textcolor{lgray}{#1}}
\newcommand{\eg}[1]{\textcolor{lgray}{#1}}

\newcommand{\noteparagraph}[1]{\smallskip \textbf{#1}\,\,}

\setbeamertemplate{frametitle}{
	\hspace{-1.5em}
	\insertframetitle\\
	\hspace{-.5em}\scriptsize\insertframesubtitle\hfill\insertpart\\[-.9em]
	\rule{\textwidth}{.1pt}
}
\setbeamertemplate{frametitle}{%
	\renewcommand{\arraystretch}{0.5}
	\begin{tabular}{@{}l}
		\hspace{-1.5em}
		\insertframetitle \tabularnewline
		\hspace{-.5em}
		\scriptsize\insertframesubtitle \tabularnewline
	\end{tabular}
	\hfill%
	{\scriptsize\insertpart}\\[-.5em]
	\rule{\textwidth}{.1pt}
}
\setbeamertemplate{footline}{
	\usebeamercolor[fg]{structure}
	\hspace*{.5cm}\raisebox{3pt}{
	\begin{tikzpicture}
		\draw [draw opacity=0.0] (-2pt,0) -- (.85\textwidth + 2pt,0) -- (.85\textwidth + 2pt,-5pt) -- (-2pt,-5pt) -- cycle;
		\draw (0,0) -- (.85\textwidth,0);
		\ifnum\inserttotalframenumber>1
		\if \insertpartstartpage \insertpartendpage
		\else
		\draw [fill,xshift=.85\textwidth / (\insertpartendpage - \insertpartstartpage) * (\insertpagenumber - \insertpartstartpage)] (0,0) -- (2pt,-5pt) -- (-2pt,-5pt) -- cycle;
		\fi
		\fi
	\end{tikzpicture}
	}
	\hfill\raisebox{3pt}{\insertframenumber/\inserttotalframenumber\hspace{3pt}}
}

\definecolor{hllgreen}{rgb}{.2,.7,.2}
\definecolor{hllgreenbg}{rgb}{.9,1,.9}
\definecolor{hllblue}{rgb}{.2,.2,.7}
\definecolor{hllbluebg}{rgb}{.9,.9,1}
\definecolor{hllorange}{rgb}{1,0.482,0}
\definecolor{hllorangebg}{rgb}{1,0.782,.4}
\setbeamertemplate{blocks}[rounded]
\setbeamercolor{structure}{fg=hllblue}
\setbeamercolor{normal text}{fg=black}
\setbeamercolor{alerted text}{fg=hllorange}
\setbeamercolor{block title alerted}{fg=black,bg=hllorange}
\setbeamercolor{block body alerted}{fg=black,bg=hllorangebg}
\setbeamercolor{block title}{fg=white,bg=hllblue}
\setbeamercolor{block body example}{fg=black,bg=hllgreenbg}
\setbeamercolor{block title example}{fg=white,bg=hllgreen}
\setbeamercolor{block body}{fg=black,bg=hllbluebg}

\newcommand{\prosymbol}{%
	\raisebox{-.1\baselineskip}{%
		\begin{tikzpicture}%
			\draw [line width=.1\baselineskip] (-.25\baselineskip,0) -- (.25\baselineskip,0);%
			\draw [line width=.1\baselineskip] (0,-.25\baselineskip) -- (0,.25\baselineskip);%
		\end{tikzpicture}%
	}%
	}
\newcommand{\contrasymbol}{%
	\raisebox{.15\baselineskip}{%
		\begin{tikzpicture}%
			\draw [line width=.1\baselineskip] (-.25\baselineskip,0) -- (.25\baselineskip,0);%
		\end{tikzpicture}%
	}%
	}
\newcommand{\proconsymbol}{%
	\raisebox{-.1\baselineskip}{%
		\begin{tikzpicture}%
			\draw [line width=.05\baselineskip] (-.125\baselineskip,0) -- (-.125\baselineskip,.25\baselineskip);%
			\draw [line width=.05\baselineskip] (-.25\baselineskip,.125\baselineskip) -- (0,.125\baselineskip);%
			\draw [line width=.05\baselineskip] (0,-.125\baselineskip) -- (.25\baselineskip,-.125\baselineskip);%
			\draw [line width=.02\baselineskip] (-.25\baselineskip,-.25\baselineskip) -- (.25\baselineskip,.25\baselineskip);%
		\end{tikzpicture}%
	}%
	}
\definecolor{procolor}{rgb}{0,.8,0}
\definecolor{contracolor}{rgb}{.8,0,0}
\definecolor{proconcolor}{rgb}{0,0,.6}
%         |
%         |
%         |
%   ------+------
%         |
%         |
%         |
%
% h = .5\baselineskip
%
%
%   -------------
% h = .1\baselineskip
%
%
\newenvironment{proconlist}%
	{%
		\begin{list}{?}{}%
		\newcommand{\pro}{\item[\textcolor{procolor}{\prosymbol}]}%
		\newcommand{\contra}{\item[\textcolor{contracolor}{\contrasymbol}]}%
		\newcommand{\procon}{\item[\textcolor{proconcolor}{\proconsymbol}]}%
	}{%
		\end{list}%
	}


\usepackage{listings}
\lstset{basicstyle=\ttfamily\scriptsize,backgroundcolor=\color[rgb]{.9,.9,.9}}
\usepackage{dirtree}

\usepackage{tabularx}
\usepackage{multirow}
\usepackage{booktabs}
\usepackage{bbding}
\usepackage[squaren]{SIunits}

\AtBeginSection{%
	\begin{frame}{Übersicht}%
		\tableofcontents[currentsection]%
	\end{frame}%
}

%-------------------------------------------------------------------------------
% Navigationsleiste ausblenden
\setbeamertemplate{navigation symbols}{}

%-------------------------------------------------------------------------------
% Diesen Abschnitt über \begin{document} lassen, damit die PDF-Informationen
% korrekt gesetzt werden.
\title{Forschungsprojekt Sensornetze}
\date{10.\,Dezember~2012}
\author{Angelos~Drossos \and Hermann~Lorenz
	\and Ulrich~Meckel \and Martin~Doenicke
	\and Thomas~Bettermann \and Robert~Krampe 
	\and Marcus~Kupke \and Markus~Fischer
	\and Enrico~Uhlig}
\institute{Hochschule für Technik und Wirtschaft Dresden\\%
			Master Angewandte Informationstechnologien\\%
			Forschungsprojekt Sensornetze\\%
			Prof. Dr. J. Vogt%
}
%-------------------------------------------------------------------------------

\begin{document}
%-------------------------------------------------------------------------------
\begin{frame}[plain]
	\maketitle
\end{frame}

%-------------------------------------------------------------------------------
\section{Einführung}

\begin{frame}{Einführung}{Forschungsprojekt Sensornetze}
		\begin{block}<1->{Was wollen wir machen?}
			\begin{itemize}
			\item<1-> 	Aufstellen eines Heimautomatisierungsservers
			\item<1-> 	Steuerung von Sensorknoten (aktive und passive)
			\item<2-> 	Konzipierung eines Sensornetzes (6LoWPAN)
			\item<2-> 	aber auch Nutzung proprietärer Sensornetze
					\eg{(FS20, Homatic)}
			\item<3-> 	Entwicklung eines \enquote{neuen} Open-Source-Standards
			\item<3-> 	Einbindung vorhandener Technologien
			\end{itemize}
		\end{block}
		\begin{block}<4->{Warum wollen wir dies machen?}
			\begin{itemize}
			\item 	Vermeidung mehrerer Technologien in einem Haushalt
					\eg{(erschwert oder verhindert die Steuerung des Haushalts)}
			\item 	Abhängigkeit von bestimmten Herstellern verringern 
					\eg{(Homatic)}
			\item 	Hersteller können sich spezialisieren 
					\eg{(Server, Sensorknoten)}
			\item 	Ingenieurbüros können sich leichter beteiligen
			\end{itemize}
		\end{block}
\end{frame}


\section{Heimautomatisierungsserver}

\begin{frame}{Heimautomatisierungsserver}{Systemaufbau}
	\begin{figure}
	\centering
	\includegraphics[width=\linewidth]{Systemaufbau}
	%\linebreak
	%\emph{}
	\end{figure}
\end{frame}

\begin{frame}{Heimautomatisierungsserver}{Steuerungsserver und Gateway-Server}
	\begin{block}{Steuerungsserver}
		\begin{itemize}
		\item 	Pairing neuer Geräte
		\item 	Pflegen der angemeldeten Geräte und dessen Sensoren/Aktoren
				(in einer Datenbank)
		\item 	Regelkreis bestimmt Automatisierungsprozess (Regelwerk)
		\end{itemize}
	\end{block}
	\begin{block}{Gateway-Server}
		\begin{itemize}
		\item 	Schnittstelle zwischen CoAP und proprietären Sensornetzen
				\eg(FS20, HomeMatic)
		\item 	Nutzen der Datenbank zur Kategorisierung
		\item 	Caching von Sensorwerten
		\end{itemize}
	\end{block}
\end{frame}

%\subsection{Regelwerk}

%\begin{frame}{Heimautomatisierungsserver}{Regelwerk}
%	\begin{block}{Automatisierung per Regelwerk}
%		\begin{itemize}
%		\item 	\todo{Stand der Technik, vorhandene Lösungen!}
%		\end{itemize}
%	\end{block}

%	\begin{block}{XML als Datenformat}
%		\begin{itemize}
%		\item 	weit verbreitetes Datenformat
%		\item 	Parser in vielen Programmiersprachen vorhanden
%		\item 	XML Schema Validation
%		\item 	in Blockdarstellungen überführbar (für Nutzer)
%		\end{itemize}
%	\end{block}
%\end{frame}

\begin{frame}{Heimautomatisierungsserver}{Anfrage an das Sensornetz}
	\begin{block}{direkte CoAP-Anfragen}
		\begin{itemize}
		\item 	in das 6LoWPAN-Netz ohne Probleme
		\item 	in das FS20/HomeMatic-Netz über Gateway
		\end{itemize}
	\end{block}
	\begin{block}{HTTP-Anfragen auf dem Server}
		\begin{itemize}
		\item 	Mapping von HTTP auf CoAP
		\end{itemize}
	\end{block}
	\begin{block}{Webserver}
		\begin{itemize}
		\item 	Statistiken
		\item 	Erstellung von Regeln
		\item 	Übersicht über alle Sensoren und Aktoren
		\end{itemize}
	\end{block}
\end{frame}


\subsection[REST]{REST Programmierparadigma im Sensornetz}


\begin{frame}{\insertsubsection}{}
% Warum ist das REST Programmierparadigma sinnvoll im Sensornetz
		\begin{block}<+->{Adressierbarkeit}
			\begin{itemize}
        		\item jeder REST-Dienst eines Servers hat eindeutige Adresse (URI)
        		\item Sensoren und Aktoren haben feste Adresse % unter der sie erreichbar sind
    			\end{itemize}
		\end{block}
		\begin{block}<+->{Zustandslosigkeit}
    			\begin{itemize}
        		\item zustandslose Anfragen
        		\item jeder Request ist in sich geschlossen
        		\item Request enthält alle Informationen für die Verarbeitung
    			\end{itemize}
		\end{block}
		\begin{block}<+->{Operationen}
    			\begin{itemize}
        		\item REST-Server beschreiben ihre Dienste auf Anfrage
        		\item standardisierte Anfragemöglichkeiten\newline
				(GET / PUT / POST / DELETE)
        		\item Sensoren/Aktoren können mit GET abgefragt werden
        		\item Aktoren können mit PUT geschaltet werden
    			\end{itemize}
		\end{block}
\end{frame}



% CoAP : http://datatracker.ietf.org/doc/draft-ietf-core-coap/
\begin{frame}{Constrained Application Protocol (CoAP)}{}
        \begin{proconlist}
            \pro REST Paradigma, aber \enquote{leichtgewichtiger} als HTTP
            \pro einfaches HTTP-Mapping möglich, laut RFC-draft vorgesehen
            \pro optionale Zuverlässigkeitsmechanismen (Confirmable / Non-Confirmable)
            \pro Interoperabilität
            \pro entspricht dem Ansatz \emph{Internet der Dinge}
            \pro geringer Overhead gegenüber
HTTP\footnote{\url{http://www.comnets.uni-bremen.de/itg/itgfg521/aktuelles/fg-workshop-29092011/ITG_HH_thomas_poetsch.pdf}}
            \procon noch in Entwicklung (wird ständig verbessert, jedoch ggf. veraltete Implementierungen)
            \contra größerer Overhead als Eigenentwicklung\newline
			(jedoch mangelnde Interoperabilität)
        \end{proconlist}
\end{frame}


\section{Proprietäre Systeme}

\begin{frame}{CUL}
	\begin{columns}
		\begin{column}[c]{0.45\textwidth}
			\begin{block}{CUL}
			\begin{itemize}
			\item 	RF-Device in Form eines USB-Dongles \linebreak
					(externe Antenne)
			\item 	Open-Source-Software (culfw) unterstützt verschiedene Protokolle
			\item 	verschiedene Modi: Slow-RF und \enquote{AskSin}
			\item 	\alert{kein Mischbetrieb zwischen Slow-RF und AskSin}
			\end{itemize}
			\end{block}
		\end{column}
		\begin{column}[c]{0.45\textwidth}
			\includegraphics[width=\linewidth]{Cul}
			\begin{block}{Protokolle}
				\begin{itemize}
				\item 	Slow-RF: \linebreak
						\textbf{FS20}, \textbf{FHT}, EM, \linebreak
						S300, HMS, \ldots
				\item 	AskSin: \linebreak
						\textbf{HomeMatic}
				\end{itemize}
			\end{block}
		\end{column}
	\end{columns}
\end{frame}

\subsection{FS20}

\begin{frame}{FS20}

\begin{block}{Wirtschaftliche Lage}
\begin{itemize}
\item 	Verfügbarkeit: eine Vielzahl an Hardware
\item 	Preisniveau: moderat
\end{itemize}
\end{block}

\begin{block}{Eigenschaften}
\begin{itemize}
\item 	Nachrichtenaustausch: unverschlüsselt, keine Bestätigung
\item 	Pairing: simpel \eg{(Pairing-Modus aktivieren, Setzen von IDs via Kommandos)}
\item 	Unterscheidung zwischen Hauscode und Devicecode
\item 	Kommunikation: CUL \eg{(Firmware culfw)}
\end{itemize}
\end{block}
\end{frame}

\begin{frame}{FS20}{Stand}

\begin{block}<+->{Funksteckdose (FS20 ST-2)}
\begin{itemize}
\item 	Aktorknoten
\item 	reine FS20-Komponente
\item 	schaltbar \eg{(aus Python heraus)}
\end{itemize}
\end{block}

\begin{block}<+->{Funk-Tür-Fensterkontakt (FHT 80TF-2)}
\begin{itemize}
\item 	Sensorknoten
\item 	keine reine FS20-Komponente (FHT-Protokoll)
\item 	CUL unterstützt nicht nur FS20,
		sondern auch FHT und viele weitere Protokolle
\item 	Nachrichten werden empfangen und können ausgewertet werden
\end{itemize}
\end{block}
\uncover<3->{\alert{$\Rightarrow$ für den Prototypen ausreichend}}
\end{frame}

\subsection{HomeMatic}

\begin{frame}{HomeMatic}{Vergleich zu FS20}
		\centering
	\begin{tabular}{lcc}
	\toprule
	           & \textit{FS20} & \textit{HomeMatic} \tabularnewline
	\midrule
	Preisniveau & moderat & teuer \tabularnewline
	Gerätevielfalt & groß & klein \tabularnewline
	Frequenzband & \multicolumn{2}{c}{868 MHz} \tabularnewline
	Authentifizierung & keine & AES \tabularnewline
	Übertragung & unverschlüsselt & XOR-verschlüsselt  \tabularnewline
	Empfangsbestätigung & nein & ja  \tabularnewline
	\multirow{2}{*}{Sonstiges} & \multirow{2}{*}{Funktionsgruppen} & drahtgebundene \tabularnewline
	                           &                                   & Komponenten\tabularnewline
	\bottomrule
	\end{tabular}
\end{frame}

\begin{frame}{HomeMatic}{Stand / Vorgehen}
\begin{block}{Vorgehen}
\begin{itemize}
\item 	Analyse des Quellcodes von FHEM zum Verständnis des HomeMatic
		zugrunde liegenden Protokolls
\item 	Validierung des Protokolls durch Mitschnitte per CUL
\end{itemize}
\end{block}

\begin{block}{Stand}
\begin{itemize}
\item 	Funk-Zwischenstecker-Schaltaktor 1fach: Empfang und Auswertung des Paketes erfolgriech
\item 	Funk-Handsender 4 Tasten: Empfang und Auswertung des Paketes erfolgriech
\item 	Steuern eines Aktors ist noch offen
\end{itemize}
\end{block}
\end{frame}


\section{6LoWPAN}

\subsection{Routing (RPL)}

\begin{frame}{Routing in 6LoWPAN}{RPL}
		\begin{block}{}
			\begin{itemize}
			\item 	dynamisches Routing-Protokoll
					für energiearme und verlustbehaftete Netzwerke
			\item 	Route-Over-Protokoll
			\item 	Protokoll auf Basis eines Distanzvektor-Algorithmus:
					\begin{itemize}
					\item 	DODAG (Destination Oriented Directed Acyclic Graph)
					\item 	jeder Knoten besitzt einen Rang
					\item 	spezielle Nachrichten zum Austausch der
							Routing-Informationen \eg{(DIO, DIS, DAO)}
					\item 	Zielfunktion berechnet Rang
							und wählt bevorzugten Elternknoten
					\end{itemize}
			\item 	weitere Besonderheiten:
					\begin{itemize}
					\item 	Local und Global-Repair (z.B. bei Ausfall eines Knotens)
					\item 	Verwendung des Trickle Timer Algorithmus
					\end{itemize}
			\item 	Implementation in Contiki: ContikiRPL
					\begin{itemize}
					\item 	keine im Standard definierten Sicherheitsmechanismen implementiert
					\end{itemize}
			\end{itemize}
		\end{block}
\end{frame}

\begin{frame}{Routing in 6LoWPAN}{RPL}
	\begin{figure}
	\centering
	\includegraphics[width=0.5\linewidth]{Dodag}
	\linebreak
	\emph{Ein Beispielnetz}
	\end{figure}
\end{frame}
\subsection{Sensorknoten}

\begin{frame}{Sensorknoten letztes Semester}
	\begin{itemize}
	\item 	Sensor Terminal Board
			mit rcb128rfa1 Modul
	\item 	AVR ATmega128rfa1 Microcontroller
	\item 	Ansteuerung der Sensoren
			ohne Betriebssystem:
			\begin{itemize}
			\item 	Feuchtigkeitssensor
			\item 	Drucksensor
			\item 	Geschwindigkeitssensor
			\item 	Ansteuerung per I2C-Bus
			\end{itemize}
	\item 	Einarbeitung in Contiki:
			\begin{itemize}
			\item 	Wie können die Sensoren sinnvoll implementiert werden?
			\item 	Beachtung der Trennung von
					Core, CPU und Platform
			\item 	Nutzung bereits implementierter Schnittstellen
			\end{itemize}
	\end{itemize}
\end{frame}

\begin{frame}{Sensorknoten dieses Semester}
	\begin{itemize}
	\item 	deRFmega128-Board (Batteriebetrieb)
	\item 	AVR ATmega128rfa1 Microcontroller
	\item 	Verifizierung der I2C-Schnittstelle in Contiki durch neuen Sensor (Lichtsensor)
	\item 	Ansteuerung eines Aktors (Heizungsthermostat):
			\begin{itemize}
			\item 	zwei Boards, die per UART miteinander kommunizieren
			\item 	Software des Heizungsthermostat
					durch Bachelor-Projekt
					bereitgestellt
			\item 	das deRFmega128-Board stellt die Funk-Kommunikation bereit
			\end{itemize}
	\end{itemize}
\end{frame}


%\section[Fazit]{Schlussbemerkungen}
%-------------------------------------------------------------------------------
\begin{frame}{Schlussbemerkungen}
	\begin{itemize}
	\item	die \emph{Implementierung von} Contiki ist sehr verwoben
	\item	die \emph{Programmierung für} Contiki ist recht elegant
	\item	der Dokumentation fehlen Einführungen für das Systemverständnis
	\item 	die Netzwerk-Anbindung ist gut ausgebaut
	\item 	die Funktionalität im Net Stack sind klar getrennt
	\item 	die praktischen Einsatzmöglichkeiten des Dynamic Module Loading
			ist in Sensornetzen beschränkt
\end{itemize}
\end{frame}
%-------------------------------------------------------------------------------
\begin{frame}{Ausblick}
	\begin{itemize}
	\item 	Net Stack: Security--layer (ContikiSec)
	\item 	Sleep Modi einbauen
	\item 	Debug-System (bisher nur als Macro pro Datei)
	\item 	Echtzeit-Eigenschaften
	\end{itemize}
\end{frame}
%-------------------------------------------------------------------------------
\begin{frame}{Ausgewählte Quellen}
%	\begin{enumerate}
%	\item 	Farooq, O. M. \& Kunz, T. (2011):
%	\end{enumerate}
	\begin{thebibliography}{contiki12}
	\bibitem[contiki12]{OSforWSN:2011}
		Farooq, O. M. \& Kunz, T.:
		\emph{Operating Systems for Wireless Sensor Networks: A Survey},
		\url{www.mdpi.com/1424-8220/11/6/5900/pdf},
		2011
	\bibitem[contiki12]{DynReProgSensors:2008}
		Strübe, Jan Moritz:
		\emph{Dynamische Re-Programmierung von Sensorknoten zur Laufzeit},
		\url{strübe.de/wp-content/uploads/2008/08/da.pdf},
		Diplomarbeit,
		% Friedrich-Alexander-Universität Erlangen-Nurnberg,
		2008
	\bibitem[contiki12]{dunkels06:2006}
		Dunkels, Adam et al.:
		\emph{Run-Time Dynamic Linking for Reprogramming Wireless Sensor Networks}
		\url{dunkels.com/adam/dunkels06runtime.pdf},
		2006
	\end{thebibliography}
\end{frame}
%-----------------------------------------------------------------------------11

%-------------------------------------------------------------------------------

\end{document}
