\chapter{Beispielprojekt}
%===============================================================================
%===============================================================================
	In diesem Abschnitt soll kurz erklärt werden, welche Schritte
	unternommen werden müssen, um den ATmega128RFA1 über einen AVR-Dragon
	mit Contiki zu bespielen.

	Dabei wird der ATmega128RFA1 mittels JTAG an den AVR-Dragon angebunden.
	Dieser widerum wird per USB an den PC angeschlossen.
	(\siehe{fig:app:bsp:schaltung})
	\begin{figure}
	\centering
	\begin{tikzpicture}[
		>=stealth',
		Geraet/.style={
			top color=white,
			bottom color=blue!50!black!20,
			draw=blue!50!black!50,
			minimum height=.8cm,
			rectangle,
			rounded corners=1mm
			},
		]
	\path (0,0) node [Geraet] (pc) {PC}
		(pc.east)
		++(2,0) node [anchor=west, Geraet] (dragon) {AVR-Dragon}
		(dragon.east)
		++(2,0) node [anchor=west, Geraet] (atmega) {ATmega128RFA1}
		;
	
	\draw [-] (pc) edge node [font=\tiny,above] {USB} (dragon);
	\draw [-] (dragon) edge node [font=\tiny,above] {JTAG} (atmega);
	\end{tikzpicture}
	\caption{Verkabelung für das Beispielprojekt}
	\label{fig:app:bsp:schaltung}
\end{figure}


	Neben Standardprogrammen, wie einem Editor, sollten noch der
	AVR-Compiler, der Programmer AVR-Dude und Git installiert werden.  Für
	Debian heißen die Pakete \lstinline=avr-gcc=, \lstinline=avrdude=
	und \lstinline=git=.

	Nun müssen die folgenden Schritte durchgeführt werden:
	\begin{enumerate}
	\item	Quellcode von Contiki besorgen:
		\lstinputlisting[style=block]{lst/app_bsp_gitclone}

		Damit entsteht im aktuellen Verzeichnis ein Ordner
		\enquote{contiki}.  In diesem müssen ein paar Anpassungen
		vorgenommen werden.

	\item	Makefile.avr anpassen:
		
		In der Datei \lstinline=contiki/cpu/avr/Makefile.avr=
		müssen die zwei Regeln
		\lstinputlisting[style=block]{lst/app_bsp_makeavr}
		durch
		\lstinputlisting[style=block,emph={sudo}]{lst/app_bsp_makeavr_after}
		ersetzt werden.

		Dadurch bekommt der AVR-Dude die nötigen Rechte, mit dem
		AVR-Dragon zu kommunizieren.  Möchte man dies nicht tun,
		muss man sich auf dem jeweiligen Device Lese- und Schreibrechte
		geben.
	
	\item	Des Weiteren müssen im Makefile für den ATmega128RFA1 noch zwei
		weitere Änderungen vorgenommen werden.

		Datei: \lstinline=contiki/platform/avr-atmegarfa1/Makefile.avr-atmegarfa1=
		\begin{itemize}
		\item	Ersetze
			\lstinputlisting[style=block]{lst/app_bsp_makerfa1_programmer}
			durch
			\lstinputlisting[style=block]{lst/app_bsp_makerfa1_programmer_after}
			um als Programmer den Dragon einzustellen.
		\item	Ersetze
			\lstinputlisting[style=block]{lst/app_bsp_makerfa1_port}
			durch
			\lstinputlisting[style=block]{lst/app_bsp_makerfa1_port_after}
			damit der USB-Port des Dragon automatisch gefunden wird.
		\end{itemize}
	
	\item	Projektverzeichnis anlegen:
		\lstinputlisting[style=block]{lst/app_bsp_mkdir}
	
	\item	Im Projektverzeichnis Prozessdatei (\lstinline=prozess.c=) mit
		dem Inhalt aus \autoref{lst:bsp:prozess} erstellen.
		\lstinputlisting[%
			style=float,%
			label=lst:bsp:prozess,%
			caption={Beispielprojekt prozess.c},%
			language=c]%
			{lst/app_bsp_prozess.c}

	\item	Im Projektverzeichnis ein Makefile mit dem Inhalt aus
		\autoref{lst:bsp:makefile} erstellen.
		\lstinputlisting[%
			style=float,%
			label=lst:bsp:makefile,%
			caption={Beispielprojekt Makefile},%
			language=make]%
			{lst/app_bsp_makefile}
	
	\item	Nun muss nur noch mit \lstinline=make program= das Projekt
		kompiliert und auf den Mikrocontroller übetragen werden.
		Dabei muss das Passwort für den sudo-Befehl des AVR-Dude
		eingegeben werden.
	\end{enumerate}
