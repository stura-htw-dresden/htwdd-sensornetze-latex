\begin{figure}
\centering
\begin{tikzpicture}[
	>=stealth',
	NS Schicht/.style={
		top color=white,
		bottom color=blue!50!black!20,
		draw=blue!50!black!50,
		minimum width=3.5cm,
		rectangle,
		rounded corners=1mm
		},
	OSI Schicht/.style={
		top color=white,
		bottom color=orange!50!black!20,
		draw=orange!50!black!50,
		minimum width=8mm,
		rectangle,
		rounded corners=1mm
		},
	]

\node [NS Schicht] (radio) {Radio Driver};
\node [NS Schicht] (framer) [above of=radio] {Framer};
\node [NS Schicht] (rdc) [above of=framer,yshift=.2cm] {RDC Driver};
\node [NS Schicht] (mac) [above of=rdc] {MAC Driver};
\node [NS Schicht] (nw) [above of=mac,yshift=.2cm] {Network Driver};

\node (s0h) at (framer) [xshift=-7cm] {};
\node (s1h) at (mac) [xshift=-7cm] {};
\node (s2h) at (nw) [xshift=-7cm] {};

% Boxen für die OSI-Schichten
\begin{pgfonlayer}{background}
	\node [OSI Schicht] (s0) [fit=(framer) (radio) (s0h)] {};
	\node [OSI Schicht] (s1) [fit=(mac) (rdc) (s1h)] {};
	\node [OSI Schicht] (s2) [fit=(nw) (s2h)] {};
\end{pgfonlayer}

% Label für die OSI-Schichten
\node [anchor=north west] at (s0.north west) {1 Bitübertragung (Physical)};
\node [anchor=north west] at (s1.north west) {2 Sicherung (Data Link)};
\node [anchor=north west] at (s2.north west) {3 Vermittlung (Network)};

% Pfeile rechts

\draw [->] ([xshift=1cm]radio.east) to node [sloped,above] {eingehende Nachricht} ([xshift=1cm]nw.east);
\draw [<-] ([xshift=2cm]radio.east) to node [sloped,above] {ausgehende Nachricht} ([xshift=2cm]nw.east);

\end{tikzpicture}
\captionbelow{Contiki NetStack Konzept}
\label{fig:ContikiNetStackKonzept}
\end{figure}
