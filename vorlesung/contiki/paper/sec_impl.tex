\chapter{Implementationsaspekte}
\label{sec:impl}
%===============================================================================
%===============================================================================



\section{Protothreads}
%===============================================================================
	Betriebssysteme im Embedded-Bereich werden meistens
	\emph{ereignisgesteuert} oder \emph{threadbasiert} implementiert.

	\begin{description}
	\item[threadbasiert]
		Ein Programm wird linear durchlaufen.  Dabei kann es zwar durch
		einen Scheduler unterbrochen werden bzw. selbständig die CPU
		abgeben, der Kontrollfluss allerdings bleibt linear.

		Der Vorteil besteht darin, dass es eine intuitive Programmierung
		ermöglicht.

		Für jeden Thread muss dabei ein eigener Stack im Hauptspeicher
		angelegt werden.  Der Großteil des Stacks wird nie benutzt, muss
		aber im Hauptspeicher reserviert vorgehalten werden.
		Da der Hauptspeicher bei eingebetteten System aber sehr klein
		ist -- beim ATmega128RFA1 \zB nur \unita{16}{\kilo\byte} --
		sind so nur wenige Threads gleichzeitig möglich.

		Dem kann theoretisch mit einem größeren Hauptspeicher
		entgegengewirkt werden, allerdings geht mit einem größeren
		Hauptspeicher auch ein erhöhter Energiebedarf einher.  Dies
		steht der Notwendigkeit entgegen, den Mikrocontroller so
		energieeffizient wie möglich zu betreiben.

	\item[ereignisgesteuert]
		Bei ereignisgesteuerten Systemen entfallen die Stacks für
		einzelne Threads -- es existiert nur ein globaler Stack.

		Durch die ereignisgesteuerte Programmierung sind allerdings
		keine linearen Programmabläufe direkt möglich.  Sie müssen
		mittels einer Zustandsmaschine umgesetzt werden.  Das ist
		zum einen fehleranfällig und zum anderen vergrößert
		sich dadurch unnötig der auszuführende Code, der wieder
		Platz im Speicher benötigt und natürlich durch längere
		Ausführungszeiten wieder mehr Energie benötigt.
	\end{description}

	Adam Dunkels, der Entwickler von Contiki, hat sich als Lösung diese
	Problematik das Konzept der \emph{Protothreads} ausgedacht.  Sie sind in
	Anlehnung an die Coroutinen in C von Simon Tatham entstanden
	und ermöglichen ein kooperatives Multitasking mit nur einem
	gemeinsamen Stack
	\autocite{Contiki:UnderTheHood}.


\subsection{Konzept}
%-------------------------------------------------------------------------------
	Als Konzept wird jeder Prozess zu einer Funktion, die aus einem
	einzigen Switch-Block besteht \parensiehe{lst:pt-idee}.
		\lstinputlisting[%
			style=float,%
			label=lst:pt-idee,%
			caption={Grundkonzept eines Contikiprozesses},%
			language=c]%
			{lst/pt_idee.c}
	Die einzelnen Case-Marken stellen dabei die möglichen Positionen
	eines Kontextwechsels dar.  Wird dem Prozess vom Scheduler die CPU
	zugewiesen, so wird die Prozessfunktion mit der jeweiligen Sprungmarke
	aufgerufen, an der der Prozess fortgesetzt werden soll.

	Bei der Abgabe der CPU wird die Prozessfunktion verlassen und dem
	Scheduler mitgeteilt, an welcher Sprungmarke bei der nächsten
	CPU-Zuteilung die Funktion fortgesetzt werden soll.

	Dabei ist es sehr einfach möglich, die CPU abzugeben, um anderen
	Prozessen die Abarbeitung zu ermöglichen \parensiehe{lst:pt-pause}
		\lstinputlisting[%
			style=float,%
			label=lst:pt-pause,%
			caption={Abgabe der CPU ohne eine Bedingung zur weiteren Abarbeitung},%
			language=c]%
			{lst/pt_pause.c}
	bzw. auf ein bestimmtes Ereignis/eine Bedingung zu warten
	(\siehe{lst:pt-wait}).
		\lstinputlisting[%
			style=float,%
			label=lst:pt-wait,%
			caption={Abgabe der CPU mit einer Bedingung zur weiteren Abarbeitung},%
			language=c]%
			{lst/pt_wait.c}

	Der Switch-Block erlaubt es dabei auch direkt in einen
	Schleifenblock oder in einen If-Else-Block zu springen.
	\autocite{dunkels06protothreads,duff:loops}
	%\todo{Quelle, hat Dunkels referenziert, Dunkels \enquote{Protothreads: Simplifying Event-Driven Programming of Memory-Constrained Embedded Systems} verweist auf Duff Quellennr 8}
	Damit ist ein Kontextwechsel
	an den verschiedensten Stellen \emph{in der Prozessfunktion} möglich.

	Nachteilig ist hier, dass dieser Mechanismus \emph{nur} direkt in den
	Threads einsetzbar ist.  Ein Kontextwechsel in einer
	aufgerufenen Funktion ist nicht möglich.  Ist eine Funktion so
	umfangreich, dass man einen Kontextwechsel während der Laufzeit
	vorsehen möchten, dann kann man die Funktion in einen eigenen Thread
	auslagern.  Dieser Thread wird dann gestartet und der Prozess wartet
	auf die Beendigung des Threads.  Im Thread kann dann aber wieder ein
	Kontextwechsel an den passenden Stellen vorgenommen werden.


\subsection{Umsetzung}
%-------------------------------------------------------------------------------
	Das Konzept der Protothreads kann von jedem Programmierer auch von Hand
	umgesetzt werden.  Der Vorteil der Protothreadimplementierung in
	Contiki besteht darin, dass die Komplexität des Switch-Blocks
	durch Precompilermakros gekapselt wird \parensiehe{lst:pt-process}.
		\lstinputlisting[%
			style=float,%
			label=lst:pt-process,%
			caption={Grundgerüst eines Contiki-Prozesses},%
			emph={PROCESS,PROCESS_THREAD,PROCESS_BEGIN,PROCESS_END,PROCESS_PAUSE,PROCESS_WAIT_UNTIL},%
			language=c]%
			{lst/pt_process.c}
	Zu sehen, was die Makros letztendlich machen ist nur noch schwer
	möglich und möchte man etwas am Prozess-/Threadsystem ändern, so ist
	dies mit einem hohen Aufwand verbunden.  Für den Anwender der
	Protothreads wird die Arbeit aber bedeutend leichter.  Der Code wird
	les- und wartbar.

	Contiki stellt neben der bedingungslosen CPU-Abgabe bzw. einer eigenen
	Bedingung auch weitere Funktionen zum Kontextwechsel bereit.
	Etwa um auf Events \parensiehe{sec:impl:events} des Betriebssystems zu
	warten.  Diese weiteren Kontextwechsel ermöglichen die Mittel des
	Betriebssystems effizient zu nutzen und sind in der Dokumentation
	\parencite[siehe][Contiki processes]{Contiki:Doc}
	erklärt.

	Für die Label der Case-Marken wird von Contiki die aktuelle
	Zeilennummer des Quellcodes eingesetzt.  Diese ist damit in der Datei
	-- und damit in dem Prozess -- einzigartig, sofern der Programmierer
	keine zwei Kontextwechsel in einer Codezeile vorsieht.


\subsection{Bewertung}
%-------------------------------------------------------------------------------
	Die Protothreads ermöglichen relativ viele \emph{kooperative
	Threads} mit einem nur geringen Speicher zu betreiben, da nur ein
	einziger Stack verwendet wird.	Die Kontextwechsel sind mit einem
	nur geringen Aufwand zur Sicherung des Kontextes verbunden.

	Die Verwendung der Protothreadbibliothek ist relativ simpel.
	Eine Überführung von linearen Abläufen in Zustandsmaschinen ist
	nicht notwendig.

	Konzeptbedingt gibt es aber ein paar Nachteile, die bei der
	Programmierung zu beachten sind.
	\begin{itemize}
	\item	In einem Thread können keine Switch-Blöcke verwendet
		werden.
	\item	Variablen eines Threads, die ihren Wert über einen
		Kontextwechsel behalten sollen, müssen als \lstinline=static=
		oder als globale Variable deklariert werden.  Im letzteren
		Fall sollte die Variable \lstinline=volatile= gesetzt werden,
		falls verschiedene Threads und/oder \acp{ISR}
		darauf zugreifen.
	\item	Kontextwechsel müssen vom Programmierer explizit angegeben
		werden -- es findet kein präemptiver Entzug der CPU-Ressource
		statt.
		Bei Algorithmen mit hoher Laufzeit muss daher regelmäßig ein
		Kontextwechsel vorgesehen werden.
		Dafür weiß der Programmierer aber wann das Programm unterbrochen
		werden kann und kann hier Aufwand für Zustandssicherungen
		einsparen.
	\item	Ein Kontextwechsel ist nur in Threads möglich.  Wird ein Teil
		des Programmes in eine Funktion ausgelagert, so ist dort kein
		Kontextwechsel möglich.  Ist eine Auslagerung notwendig,
		so kann man einen weiteren Thread mit der Aufgabe starten, der
		wiederum selbständig Kontextwechsel vorsehen kann.
	\end{itemize}



\section{IPC-Mechanismen}
%===============================================================================
	Contiki bietet eine Reihe von \acs{IPC}-Mechanismen zur Kommunikation und
	Synchronisierung zwischen den einzelnen Prozessen an.


\subsection{Events und Polling}
%-------------------------------------------------------------------------------
\label{sec:impl:events}
	Unter Contiki können Prozessen zwei verschiedene Signale zugestellt
	werden: Events und Polling
	\begin{description}
	\item[Events]
		Events können von einem Prozess zu einem anderen Prozess
		zugestellt werden.  Sie werden dabei in einer Warteschlange
		(Event-Queue) einsortiert und dann der Reihe nach den jeweiligen
		Prozessen zugestellt.
	\item[Polling]
		Neben Events kann bei jedem Prozess ein Polling-Flag gesetzt
		werden.  Bevor der Scheduler die Event-Queue abarbeitet,
		schaut er bei jedem Prozess, ob das Polling-Flag gesetzt ist.
		Ist dieses Flag gesetzt, so wird direkt der Polling-Handler
		ausgeführt.  Erst wenn alle Polling-Flags abgearbeitet sind,
		wird das erste Event aus der Queue ausgeliefert.
	\end{description}

	Das Polling-Flag wird dadurch notwendig, dass \acp{ISR} möglichst
	kurz sein sollen.  In ihnen sollen nur die notwendigsten Maßnahmen
	vorgenommen werden, um \zB ein eingegangenes Signal nicht zu verlieren.
	Dieses Signal wird in der \ac{ISR} in einen Cache geschrieben und
	der Prozess, der das Signal eigentlich verabeiten soll, wird über
	das Polling-Flag benachrichtigt, dass ein neues Signal vorliegt.

	Events sind also dafür da, dass bestimmte Signale nicht verloren gehen.
	Das Polling hingegen sorgt dafür, dass Signale, die dringend bearbeitet
	werden müssen ausreichend schnell zwischengespeichert werden könne, so
	dass bei häufig eingehenden Signalen keine Signale verloren gehen.
	(\siehe{fig:isr_poll_event})
	\begin{figure}
\centering
\begin{tikzpicture}[
	>=stealth',
	Objekt/.style={
		top color=white,
		bottom color=blue!50!black!20,
		draw=blue!50!black!50,
		minimum width=3.5cm,
		minimum height=.8cm,
		rectangle,
		rounded corners=1mm
		},
	Pfeil/.style={
		top color=white,
		bottom color=orange!50!black!20,
		draw=orange!50!black!50,
		minimum width=8mm,
		single arrow
		},
	]

	
	\node [Objekt] (hw) {Hardware};
	\node [Objekt] (isr) [below of=hw,yshift=-.5cm] {ISR};
	\node [Objekt] (handler) [below of=isr,yshift=-.5cm] {Polling-Handler};
	\node [Objekt] (thread) [below of=handler,yshift=-.5cm] {Thread};

	\draw [->] ([yshift=-1mm]hw.east) to [bend left=70] node [right] {Interrupt} ([yshift=1mm]isr.east);
	\draw [->] ([yshift=-1mm]isr.east) to [bend left=70] node [right] {Poll} ([yshift=1mm]handler.east);
	\draw [->] ([yshift=-1mm]handler.east) to [bend left=70] node [right] {Event} ([yshift=1mm]thread.east);
	\path (thread) edge [loop, in=170,out=190, distance=2cm, ->] 
		node [left] {Event} (thread);

	\node [Pfeil,rotate=90,anchor=west,minimum height=27mm] at ([xshift=3cm]thread.east) {zeitkritisch};
\end{tikzpicture}
\captionbelow{Zusammenspiel von Events und Polling}
\label{fig:isr_poll_event}
\end{figure}


	Beispielsweise könnte eine \ac{ISR} ein eingehendes Zeichen auf der
	seriellen Verbindung in einer globalen Variablen zwischenspeichern
	und einem Prozess A per Polling mitteilen, das dort ein neues Zeichen
	vorliegt.  Dadurch, das sich der Polling-Handler des Prozesses A vor
	die Events in der Event-Queue schiebt, kann hier etwas aufwändiger
	das Zeichen in einen größeren Puffer eingefügt werden und dem
	Prozess A das Event zugestellt werden, dass ein weiteres Zeichen im
	Puffer wartet.

	Damit ist mit nur sehr wenigen Maschineninstruktionen die \ac{ISR}
	abgehandelt und die Interrupts sind alle wieder verfügbar.  Während
	jetzt der Polling-Handler (aufwändig) das eingegangene Zeichen in
	einen Cache einfügt und das Event auslöst, kann \zB trotzdem ein
	weiterer Interrupt, \zB diesmal des Analog-Digital-Wandlers, behandelt
	werden.

	Die Trennung von Polling und Events erlaubt damit die Trennung von
	\begin{itemize}
	\item	Entgegennahme eines Signals (\ac{ISR}),
	\item	aufwändiger Zwischenspeicherung eines Signals (Poll) und
	\item	eigentlicher Verarbeitung des Signals (Event).
	\end{itemize}

\subsubsection{Programmierung Polling}
	Ein Pollingsignal wird mittels
	\lstinline=process_poll(struct process * p)= abgesetzt.
	Der Pollinghandler wird in der Definition des Prozessthreads bekannt
	gegeben. (\siehe{lst:pollhandler})
		\lstinputlisting[%
			style=float,%
			label=lst:pollhandler,%
			caption={Installation eines Polling-Handlers},%
			emph={PROCESS,PROCESS_THREAD,PROCESS_BEGIN,PROCESS_END,PROCESS_PAUSE,PROCESS_WAIT_UNTIL,PROCESS_POLLHANDLER},%
			language=c]%
			{lst/pollhandler.c}
	Aufgrund der Makroimplementierung kann man den auszuführenden Code auch
	direkt dem Makro \lstinline=PROCESS_PROLLHANDLER()= übergeben.  Dies ist
	aber aus Gründen der Lesbarkeit eher abzulehnen.

\subsubsection{Programmierung Events}
	Events können synchron oder asynchron übergeben werden.  Die
	\emph{asynchrone} Methode funktioniert, wie oben bereits beschrieben
	und erfolgt mit der Funktion \lstinline=process_post(...)=.

	Die \emph{synchrone} Methode funktioniert dabei etwas anders.  Hierbei
	wird durch den Aufruf von \lstinline=process_post_synch(...)= sofort
	die Kontrolle an den jeweiligen anderen Prozess abgegeben.  Hierbei muss
	aber darauf geachtet werden, dass keine wechselseitigen synchronen
	Aufrufe auftreten bzw. sich ein Ring bildet. (\siehe{fig:event-cycle})
	\begin{figure}
\centering
\begin{tikzpicture}[
	>=stealth',
	prozess/.style={fill,
		top color=white,
		bottom color=blue!50!black!20,
		draw=blue!50!black!50,
		minimum width=8mm,
		circle},
	]
% Prozesse
\node [prozess] (a) at (45:1cm) {A};
\node [prozess] (b) at (285:1cm) {B};
\node [prozess] (c) at (165:1cm) {C};

% Prozessaufrufe
\draw [->] (a)
	edge [bend left]
	node [auto] {\lstinline=process_post_synch(...)=}
	(b);
\draw [->] (b)
	edge [bend left]
	node [auto] {\lstinline=process_post_synch(...)=}
	(c);
\draw [->] (c)
	edge [bend left]
	node [auto] {\lstinline=process_post_synch(...)=}
	node [-,
		shape=cross out,
		minimum width=.5cm,
		minimum height=.5cm,
		very thick,
		draw=red] {}
	(a);
\end{tikzpicture}
\captionbelow{Verbotener Zyklus von synchronen Events}
\label{fig:event-cycle}
\end{figure}


	Die Möglichkeiten für Prozesse auf ein bestimmtes oder ein beliebiges
	Event zu warten sind vielfältig.  Diese -- und die konkreten
	Parameter für das Absetzen eines Events -- können in der Dokumentation
	von Contiki nachgelesen werden
	\parencite[siehe][Contiki processes]{Contiki:Doc}.


\subsection{Semaphore}
%-------------------------------------------------------------------------------
	Für Semaphore stehen die drei Makros
	\begin{itemize}
	\item	\lstinline=PT_SEM_INIT()=,
	\item	\lstinline=PT_SEM_WAIT()= und
	\item	\lstinline=PT_SEM_SIGNAL()=
	\end{itemize}
	zur Verfügung.  Sie greifen jeweils auf ein Struct vom Typ
	\lstinline=struct pt_sem= zu und ermöglichen damit die Synchronisierung
	\emph{zwischen} verschiedenen \emph{Prozessen}
	\parencite[siehe][Protothreads semaphores]{Contiki:Doc}.

	Die Semaphore der Protothreads haben den Nachteil, dass sie nur für
	Threads zur Verfügung stehen.	Sie können also weder in \acp{ISR}
	noch in Event-Handlern eingesetzt werden.  Wird eine Variable nicht nur
	zwischen Threads geteilt, dann muss hier ein alternativer Mechanismus
	gesucht werden.

	Für \acp{ISR} muss verhindert werden, dass \acp{ISR} überhaupt
	ausgeführt werden können, da ein Warten in der \ac{ISR} aufgrund der
	Anforderung an die Kürze einer \ac{ISR} nicht möglich ist.  Die Variable
	muss also an allen anderen Stellen mit \lstinline=cli()= und
	\lstinline=sei()= oder einem ähnlichen Mechanismus gesperrt werden.
	In einem Event-Handler darf ebenfalls kein \lstinline=PT_SEM_WAIT=
	verwendet werden, da dieses intern auf  \lstinline=PT_WAIT=
	zurückgreift, was in einem Handler nicht verwendet werden darf.

	Ein weiteres kleines Manko ist es, dass eine Try-Funktion fehlt, die
	entweder den kritischen Abschnitt betritt oder mit einem Fehlerwert
	sofort zurückkehrt, statt an dieser Stelle zu warten.


\subsection{Protosockets}
%-------------------------------------------------------------------------------
	Contiki bietet zur Kommunikation zwischen Prozessen die Protosockets.
	Sie sind eine leichtgewichtige Implementation, die ein Interface ähnlich
	der BSD-Sockets anbieten.
	(\lstinline=PSOCK_INIT=,
		\lstinline=PSOCK_SEND_STR=,
		\lstinline=PSOCK_READBUF=, \dots)

	Da die Implementierung der Protosockets auf Protothreads zurückgreift,
	sind sie mit den selben Problemen behaftet.  So sind sie \zB an einen
	Protothread geknüpft.  Auf einem Protosocket kann also nie mehr als ein
	Protothread arbeiten
	\parencite[siehe][Protosockets library]{Contiki:Doc}.


\subsection{Speicherverwaltung}
%-------------------------------------------------------------------------------
	Das dynamische Anfordern von Speichersegment ist auf eingebetteten
	System wenig üblich, da der Speicher sowieso knapp ist und das
	dynamische Anfordern zusätzlichen Speicher für die Verwaltung der
	Segmente verbraucht.  Trotzdem bietet Contiki drei verschiedene
	Möglichkeiten, falls es unbedingt benötigt wird.

	\begin{description}
	\item[memb]
		Über das \emph{Memory Block Management} wird zur Compilezeit
		über \lstinline=MEMB= eine feste Anzahl gleichgroßer
		Speichersegmente reserviert.  Diese können dann zur Laufzeit
		über \lstinline=memb_alloc= angefordert und über \lstinline=memb_free= freigegeben werden
		\parencite[siehe][Memory block management functions]{Contiki:Doc}.
	\item[mmem]
		Der \emph{Managed Memory Allocator} erlaubt es zur Laufzeit
		zu entscheiden, wie groß ein angefordert Speicherbereich sein soll (\zB für Strings).
		Um den Speicherbedarf klein zu halten, werden auch nach der
		Anforderung die Speicherbereiche automatisch umsortiert.
		Daher ist hier dann eine indirekte Adressierung des gewünschten
		Speicherbereiches notwendig
		\parencite[siehe][Managed memory allocator]{Contiki:Doc}.
	\item[malloc]
		Zusätzlich stehen in der stdlib.h die bekannten C-Funktionen
			\lstinline=malloc=,
			\lstinline=calloc=,
			\lstinline=realloc= und
			\lstinline=free=
		zur Verfügung.  Die angeforderten Speicherbereiche werden im
		Gegensatz zu memm nicht umsortiert, müssen also nicht indirekt
		adressiert werden.  Der Speicher wird mittels eines Heaps
		verwaltet.  Dadurch kann es zur Fragmentierung und damit zum
		unnötigen Speicherverbrauch kommen.  Deshalb sollte auf diese
		Funktionen möglichst nicht zurückgegriffen werden
		\autocite{ContikiWiki:malloc}.
	\end{description}

	Generell sind memb und memm zu bevorzugen.  Die Entscheidung für einen
	der Mechanismen muss jedoch von Fall zu Fall erfolgen.
	(\siehe{tbl:speicher})
	\begin{table}
\centering
\captionabove{Vergleich der Mechanismen zur Speicherverwaltung}
\label{tbl:speicher}
\begin{tabular}{lccc}
\toprule
	\theadhll{Mechanismus}
	& \theadhll{Blockgrößen}
	& \theadhll{Fragmentierung}
	& \theadhll{Addressierung}
	\tabularnewline
\midrule
memb	& fest		& nein	& direkt	\tabularnewline
memm	& beliebig	& nein	& indirekt	\tabularnewline
malloc	& beliebig	& ja	& direkt	\tabularnewline
\bottomrule
\end{tabular}
\end{table}

