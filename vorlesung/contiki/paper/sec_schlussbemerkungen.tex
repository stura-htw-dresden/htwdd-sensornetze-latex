\chapter{Schlussbemerkungen}
\label{sec:fazit}

	Die Einteilung des Contiki-OS in Module bzw. Schichten lässt eine
	flexible und leichtgewichtige Konfiguration des Betriebssystems zu.

	Beim Kompilieren des (ersten) HelloWorld-Projekts stellt man fest,
	dass der komplette Contiki-Kern integriert wird,
	obwohl nicht alle Komponenten für das geschriebene Programm benötigt
	werden.
	Dies ist darin begründet, dass Contiki es ermöglicht, zur Laufzeit
	weitere Programme nachzuladen.  Würden nicht alle Kernkomponenten
	von vornherein integriert werden, müsste beim Nachladen eines weiteren
	Programmes eventuell zusätzlich der Contiki-Kern ausgetauscht werden.
	Damit würde ein bedeutend höherer Energieaufwand für Funkübertragung
	und Speichervorgang auf dem Endgerät einhergehen.

\medskip

	Durch eine gezielte Konfiguration des Systems ist es möglich, den
	wenigen,
	vorhandenen Speicher optimal zu nutzen.  Eine wesentliche Komponente
	des Betriebssystems sind die leichtgewichtigen Protothreads, durch die
	mehrere Prozesse kooperativ miteinander interagieren können.

	Da die Protothreads sich einen gemeinsamen Stack teilen, ist der
	Speicheroverhead auf eine Minimum reduziert.
	Die eingängige Funktionsweise der Protothreads zieht
	einen nur sehr geringen Einarbeitungsaufwand mit sich.

	Mithilfe von Protothreads und Contiki-eigener IPC-Mechanismen sind
	elegante und verständliche Lösungen vieler typischer und neuer Probleme
	der Sensornetze möglich.

\medskip

	Tiefgreifende Veränderungen des Betriebssystemkerns bedürfen,
	auf Grund der hohen Komplexität des Betriebssystems, eines hohen
	Einarbeitungsaufwandes.

	Die hohe Komplexität ist darin zu begründen, dass Contiki auf sehr
	unterschiedlichen Mikrocontrollerplattformen eingesetzt werden soll.
	Zum einen unterscheiden sich Plattformen untereinander stark und es
	müssen daher einige Schritte zur Abstraktion der Hardware durchgeführt
	werden.
	Zum anderen ist
	es selbst für eine einzelne Plattform nicht trivial, das System
	komplett zu überblicken.

	Ein Beispiel für das nichttriviale Verhalten ist der Energiesparmodus
	des ATmega128RFA1: Ziel ist es, den Mikrocontroller in seiner
	Schlafphase in den Deep-Sleep-Modus zu schalten, denn dort sind die
	Energieeinsparungen erheblich.\footnote{Der AVR ATmega128RFA1 verbraucht
	im Normalbetrieb etwa \(16\)--\unita{18}{\milli\ampere}.
	Im normalen Sleep-Modus sinkt der Verbrauch zwar bereits auf
	\unita{4}{\milli\ampere}, im Deep-Sleep-Modus allerdings noch weiter auf
	\unita{250}{\nano\ampere}.}

	In Contiki lässt es sich aber schwer abschätzen, ob und wann der
	Mikrocontroller in den Deep-Sleep-Modus schalten kann.  Eine
	Voraussetzung ist \iA das Ausschalten des Funksensors. Ob die anderen
	Voraussetzungen erfüllt sind, kann nur derjenige sagen, der den vollen
	Überblick über das Betriebssystem hat. Doch bietet Contiki hierzu
	bereits Lösungen an, um die Energieeffizienz einzelner Module zu
	kontrollieren.

	Bei Projekten für eingebettete Systeme ist davon auszugehen, dass
	die Entwickler das komplette System überschauen, um eine möglichst
	hohe Energieeffizienz gewährleisten zu können.

\medskip

	Der Netzwerk-Stack von Contiki ist sinnvoll gegliedert, so dass sich
	neue Implementierungen leicht nachvollziehbar einbauen lassen.
	Für den Großteil der Internet-Anwendungen reicht die vorhandene
	Implementierung aus, um den jeweiligen Sensorknoten mit dem Internet
	zu verbinden.

	Dennoch besteht im Bereich des Duty-Cycling noch viel Forschungsbedarf,
	so dass in nächster Zeit vermutlich kein Standard zu erwarten ist, der
	sich durchsetzen wird. Dies liegt vor allem daran, dass der Duty-Cycle
	stark von den Anforderungen der jeweiligen Sensornetzanwendung abhängig
	ist.

	Da eine Schicht im Netzwerk-Stack sehr viele Varianten beinhalten kann,
	wird der Einsatz von C-Makros nicht unvermeidbar bleiben, solange das
	vorhandene Netzwerk-Konzept beibehalten wird.  Das bedeutet folglich,
	dass die Entwickler eines Projekts sich über die Varianten, sowie deren
	Kompatibilität zueinander, Gedanken machen müssen.  Dies erfordert eine
	ausführliche Dokumentation der jeweiligen Implementation.

\medskip

	Alles in allem ist Contiki-OS ein Betriebssystem, das ausgereifte
	Konzepte sowie Unterstützung bei der Einhaltung der Anforderungen an
	das eingebettete System bietet.

	Nichtsdestotrotz muss ein Entwickler gute Kenntnisse zu den Interna von
	Contiki-OS haben, um die Sensornetzanforderungen optimal umzusetzen.
	Mit einer guten Dokumentation ist es nicht schwierig, sich diese
	Kenntnisse anzueignen.
