\begin{table}[t]
\begin{threeparttable}[c]
\centering
\caption[Contiki NetStack Beispiel]{Contiki NetStack Beispiel.
	Zu sehen ist der Vergleich zwischen dem 6LoWPAN-Stack und die Implementation in Contiki.}
	\label{tab:ContikiNetStackExample}
\begin{tabular}{rrlrl}
	\toprule
	\multicolumn{3}{c}{\theadhll{6LoWPAN NetStack}}
	& \multicolumn{2}{c}{\theadhll{Contiki NetStack}}
	\tabularnewline
	\cmidrule(r){1-3}
	\cmidrule(l){4-5}
	  \multicolumn{2}{c}{\theadhll{OSI-Schicht}}
	  & \multicolumn{1}{c}{{\theadhll{Protokoll}}}
	  & \multicolumn{1}{c}{{\theadhll{Contiki-Schicht}}}
	  & \multicolumn{1}{c}{{\theadhll{Implementation}}}
	  \tabularnewline
	\midrule
	4	& Transport 	& UDP					& 					& uIP-UDP 				\tabularnewline
	\addlinespace
	3	& Vermittlung 	& IPv6 / ICMPv6			& Network Driver	& SicsLoWPAN (uIPv6) 	\tabularnewline
	-	& Adaptation\tnote{1}	& 6LoWPAN Adaptation	& 					& 					\tabularnewline
	\addlinespace
	2	& Sicherung 	& \ieeeframe{} MAC		& MAC Driver 		& CSMA 				\tabularnewline
		& 				&						& RDC Driver 		& X-MAC\tnote{2} 			\tabularnewline
	\addlinespace
	1	& Bitübertragung & \ieeeframe{} PHY	& Framer 			& Framer\,802.15.4 	\tabularnewline
		& 				 &						& Radio Driver 		& rf\,230\_bb 		\tabularnewline
	\bottomrule
\end{tabular}
\begin{tablenotes}\footnotesize
\item[1] 6LoWPAN fügt in das OSI-Referenzmodell eine neue Schicht ein: der Adaptation Layer.
\item[2] 6LoWPAN definiert keinen genauen \ac{RDC}. In Contiki wird oft ContikiMAC als Default verwendet.
\end{tablenotes}
\end{threeparttable}
\end{table}
