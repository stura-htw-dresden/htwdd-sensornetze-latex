\section{Dynamic Module Loading}

% Titel geändert, so wie er auf der Contiki-OS.org Seite angegeben wurde.
%
% *4. Thema "Support for dynamic loading and replacement of individual
%  programs and services"*
% oder anders: "Downloading code at run-time"
% (ca. 15min - 20min)
%
% - Paper: Contiki - a Lightweight and Flexible Operation System for Tiny
%          Networked Sensors
%          Adam Dunkels, Björn Grönvall, Thiemo Voigt (2004)
% - Paper: Run-Time Dynamic Linking for Reprogramming Wireless Sensor Networks
%          Adam Dunkels, Noclas Finne, Joakim Eriksson, Thiemo Voigt (2006)
%
% - Diplomarbeit: Dynamische Re-Programmierung von Sensorknoten zur Laufzeit
%                 Jan Moritz Strübe (2008)
%   Link: strübe.de/wp-content/uploads/2008/08/da.pdf
%
% - Paper: Dynamic TinyOS: Modular and Transparent Incremental Code-Updates
%          for Sensor Networks
%          Waqaas Munawar, M. H. Alizai, O. Landsiedel, L. Wehrle (2010)
%
% - Das thema ist bei "large scale networks" (100-1000 und mehr Nodes per
%   Netzwerk) extrem wichtig und da wollen wir im Prinzip auch hin
% - Beispiel: Es müssen nicht nur Anwendungen wie eine komplette Shell
%   übertragen werden. Es reicht auch aus, wenn Conigurationen wie der
%   6LoWPAN-Channel geupdatet werden.
% - Anwendungsbeispiele, Theorie dahinter
% - Beispielimplementation (wenn wir es schaffen)
%

%-------------------------------------------------------------------------------
\begin{frame}{Dynamic Module Loading}{Einführung}
	\begin{exampleblock}<+->{Anwendung}
		\begin{itemize}
		\item 	an schwer zugänglichen Orten
				(Daten sammeln, extern auswerten, Software anpassen)
		\item 	Korrektur von Softwarefehlern
		\item 	Schließen von Sicherheitslücken
		\item 	Erweiterung der Funktionalität:
				spezielle Applikationen situationsbedingt (temporär) ausführen
		\end{itemize}
	\end{exampleblock}
	\begin{block}<+->{Problemstellung}
		\begin{itemize}
		\item 	genügend Speicher zum Zwischenspeichern des \enquote{Programms}
		\item 	energieaufwändige Übertragung (Verkürzung der Batterielaufzeit)
		\end{itemize}
	\end{block}
	\begin{alertblock}<+->{Lösung}
		effizientes Verteilen der neuen \enquote{Softwareversion}
	\end{alertblock}
\end{frame}
%....... note .................................................................
%\note{}
%-------------------------------------------------------------------------------
\subsection{Anforderungen und Umsetzungen}
%-------------------------------------------------------------------------------
\begin{frame}{Dynamic Module Loading}{Anforderungen}
	\begin{block}<+->{Anforderungen}
		\begin{itemize}
		\item 	Das ganze System ist austauschbar.
		\item 	Nur ein Modul ist zur Laufzeit nachzuladbar und ausführbar.
		\item 	Wenn ein kleiner Codeabschnitt geändert werden soll, dann soll
				nicht das komplette System neu aufgespielt werden.
		\item 	\alert{niedriger Energieverbrauch}
				sowohl bei der Übertragung als auch Ausführung
		\end{itemize}
	\end{block}
	\begin{block}<+->{Methoden der Umsetzung} % see TinyOS-Paper
		\begin{itemize}
		\item 	Full-Image Replacement
		\item 	Differential Image Replacement
		\item 	Virtual Machines
		\item 	Dynamic Operation Systems
		\end{itemize}
	\end{block}
\end{frame}
%....... note .................................................................
%\note{}
%-------------------------------------------------------------------------------
\begin{frame}{Dynamic Module Loading}{Umsetzungsarten}
	\begin{block}{Methoden der Umsetzung} % see TinyOS-Paper
		\begin{itemize}
		\item 	Full-Image Replacement
		\item 	Differential Image Replacement
		\item 	Virtual Machines
		\item 	Dynamic Operation Systems
		\end{itemize}
	\end{block}
	\begin{block}{Lösungen in bekannten Betriebssystemen}
		\begin{itemize}
		\item 	TinyOS % Incremental Code-Updates (Differential Image Replacement)
		\item 	SOS (Weiterentwicklung 2008 eingestellt) % SOS: lookup table
		\item 	Contiki-OS % native (ELF, Compact ELF), Java VM, CVM
		\end{itemize}
	\end{block}
\end{frame}
%....... note .................................................................
%\note{}
%-------------------------------------------------------------------------------
\subsection{Vom Empfang bis zur Ausführung}
%-------------------------------------------------------------------------------
\begin{frame}{Dynamic Module Loading}{Vom Empfang bis zur Ausführung des Moduls}
	\begin{block}<+->{Schritte vom Empfang bis zur Ausführung}
		\begin{enumerate}
		\item 	Empfangen
		\item 	EEPROM beschreiben
		\item 	\enquote{Link \& Relocation}
		\item 	Flash ROM beschreiben
		\end{enumerate}
		$\Rightarrow$ 	Das Programm ist ausführbar.
	\end{block}
	\begin{block}<+->{Technologien im Contiki}
		\begin{itemize}
		\item 	ELF
		\item 	CELF (compact ELF)
		\item 	Java Virtual Machine
		\item 	CVM (Contiki Virtual Machine)
		\item 	\notall{(und weitere)}
		\end{itemize}
	\end{block}
\end{frame}
%....... note .................................................................
%\note{}
%-------------------------------------------------------------------------------
\begin{frame}{Dynamic Module Loading}{Vom Empfang bis zur Ausführung des Moduls}
	\begin{block}<+->{Notwendige Schritte in den verschiedenen Technologien}
			\centering
			\begin{tabular}{lcccc}
				\toprule
				& \textit{ELF} & \textit{CELF} &  \textit{Java VM} & \textit{CVM}
								\tabularnewline
				\midrule
				Empfangen               & \Checkmark  & \Checkmark & \Checkmark & \Checkmark
				\tabularnewline
				EEPROM beschreiben      & \Checkmark & \Checkmark &            & 
				\tabularnewline
				Link \& Relocation      & \Checkmark & \Checkmark &  & 
				\tabularnewline
				Flash ROM beschreiben & \Checkmark & \Checkmark & \Checkmark & 
				\tabularnewline
				\bottomrule
			\end{tabular}
	\end{block}
	\begin{block}<+->{Energie-Verbrauch [\milli\joule]}
			\centering
			\begin{tabular}{lcccc}
				\toprule
				& \textit{ELF} & \textit{CELF} &  \textit{Java VM} & \textit{CVM}
								\tabularnewline
				\midrule
				Empfangen               & 29 & 12 & 22 & 2
				\tabularnewline
				EEPROM beschreiben      &  2 &  1 &     & 
				\tabularnewline
				Link \& Relocation      &  3 &  3 &   & 
				\tabularnewline
				Flash ROM beschreiben &  1 &  1 & 5 & 
				\tabularnewline
				\bottomrule
			\end{tabular}
	\end{block}
\end{frame}
%....... note .................................................................
%\note{}
%-------------------------------------------------------------------------------
\begin{frame}{Dynamic Module Loading}{Vom Empfang bis zur Ausführung des Moduls}
	\begin{block}<+->{Ausführungszeit (einfaches Programm)}
			\centering
			\begin{tabular}{lccc}
				\toprule
				& \textit{Native} &  \textit{Java VM} & \alert{\textit{CVM}}
								\tabularnewline
				\midrule
				Ausführungszeit  [\milli\second] & 0.479 & 1.79 & 0.845
				\tabularnewline
				Energieverbrauch [\micro\joule] & 0.54 & 2.0 & 0.95
				\tabularnewline
				\midrule
				Energieverbrauch [\micro\joule] & 36 000 & 27 000 & 2 000
				\tabularnewline
				bei der Übertragung       &  &  & 
				\tabularnewline
				\bottomrule
			\end{tabular}
	\end{block}
	\begin{block}<+->{Ausführungszeit (komplexes Programm)}
			\centering
			\begin{tabular}{lccc}
				\toprule
				& \alert{\textit{Native}} &  \textit{Java VM} & \textit{CVM}
								\tabularnewline
				\midrule
				Ausführungszeit  [\milli\second] & 0.67& 65.6 & 58.52
				\tabularnewline
				Energieverbrauch [\micro\joule] & 0.75& 73 & 65
				\tabularnewline
				\bottomrule
			\end{tabular}
	\end{block}
\end{frame}
%....... note .................................................................
%\note{}
%-------------------------------------------------------------------------------
\subsection{Ausblick}
%-------------------------------------------------------------------------------
\begin{frame}{Dynamic Module Loading}{Ausblick}
	\begin{alertblock}<+->{Fazit}
	\begin{itemize}
	\item 	Optimale Lösung kommt auf den Anwendungsfall an.
	\item 	Virtual Machines haben ihren Reiz bei der Übertragung, aber
			Defizite bei der Ausführung
	\item 	Virtual Machine Code kann nicht auf Hardware-nahe Funktionen
			zugreifen: Es sind native Schnittstellen erforderlich.
	\end{itemize}
	\end{alertblock}

	\begin{exampleblock}<+->{Dynamic Module Loading in der Hausautomatisierung}
		\begin{itemize}
		\item 	Application Reconfiguration ist eine mögliche Anwendung
		\item 	Sicherheit ist ein Problem: z.B. Manipulationen
		\item 	für Entwicklungsumgebungen sinnvoll
		\item 	in der Praxis i.d.R. zu hoher Energieverbrauch
		\end{itemize}
	\end{exampleblock}
\end{frame}
%....... note .................................................................
%\note{}
%-------------------------------------------------------------------------------

