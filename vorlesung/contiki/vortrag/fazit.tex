\section[Fazit]{Schlussbemerkungen}
%-------------------------------------------------------------------------------
\begin{frame}{Schlussbemerkungen}
	\begin{itemize}
	\item	die \emph{Implementierung von} Contiki ist sehr verwoben
	\item	die \emph{Programmierung für} Contiki ist recht elegant
	\item	der Dokumentation fehlen Einführungen für das Systemverständnis
	\item 	die Netzwerk-Anbindung ist gut ausgebaut
	\item 	die Funktionalität im Net Stack sind klar getrennt
	\item 	die praktischen Einsatzmöglichkeiten des Dynamic Module Loading
			ist in Sensornetzen beschränkt
\end{itemize}
\end{frame}
%-------------------------------------------------------------------------------
\begin{frame}{Ausblick}
	\begin{itemize}
	\item 	Net Stack: Security--layer (ContikiSec)
	\item 	Sleep Modi einbauen
	\item 	Debug-System (bisher nur als Macro pro Datei)
	\item 	Echtzeit-Eigenschaften
	\end{itemize}
\end{frame}
%-------------------------------------------------------------------------------
\begin{frame}{Ausgewählte Quellen}
%	\begin{enumerate}
%	\item 	Farooq, O. M. \& Kunz, T. (2011):
%	\end{enumerate}
	\begin{thebibliography}{contiki12}
	\bibitem[contiki12]{OSforWSN:2011}
		Farooq, O. M. \& Kunz, T.:
		\emph{Operating Systems for Wireless Sensor Networks: A Survey},
		\url{www.mdpi.com/1424-8220/11/6/5900/pdf},
		2011
	\bibitem[contiki12]{DynReProgSensors:2008}
		Strübe, Jan Moritz:
		\emph{Dynamische Re-Programmierung von Sensorknoten zur Laufzeit},
		\url{strübe.de/wp-content/uploads/2008/08/da.pdf},
		Diplomarbeit,
		% Friedrich-Alexander-Universität Erlangen-Nurnberg,
		2008
	\bibitem[contiki12]{dunkels06:2006}
		Dunkels, Adam et al.:
		\emph{Run-Time Dynamic Linking for Reprogramming Wireless Sensor Networks}
		\url{dunkels.com/adam/dunkels06runtime.pdf},
		2006
	\end{thebibliography}
\end{frame}
%-----------------------------------------------------------------------------11
