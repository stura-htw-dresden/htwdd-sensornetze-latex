\section{MAC-Implementierung}

% *3. MAC-Implementierung* (15min - 20min)
%
% - Implementation erläutern: Wie ist der MAC-Layer implementiert?
%   d.h. komplett software-basiert oder werden Funktionen des jeweiligen
%   Microcontrollers genutzt
%   Wenn hardware-basiert: Gibt es Unterschiede zwischen den
%   CPU-Implementationen? (kurz)
%
% - Bei der Erläuterung Blockschaltbilder und/oder Code zeigen
%   (ggf. weitere Grundlagen mit der Routing-Truppe absprechen)
%
% - Vielleicht noch den Netwerk-Simulator Coja zeigen

% Link: de.wikipedia.org/wiki/Sensornetz
%     --> Medienzugriffsprotokolle
% Link: www.sics.se/contiki/wiki/index.php/Change_MAC_or_Radio_Duty_Cycling_Protocols
%     --> RDC, MAC layers

% ? useful
% Link: https://www.hostedredmine.com/projects/contiki-ns3-integration/wiki/Design_for_Integration_Component
%     --> Contiki Socket Network Device, Contiki Scheduler, Interessante Bilder

%-------------------------------------------------------------------------------
\subsection{Net Stack Layers}
%-------------------------------------------------------------------------------
\begin{frame}{Contiki Net Stack layers}{Der Aufbau}
	\begin{enumerate}
	\item 	Radio (Hardware)
	\item 	Radio Duty Cycling (RDC)
	\item 	Framer
	\item 	Medium Access Control (MAC)
	\item 	Network
	\end{enumerate}
\end{frame}
%....... note .................................................................
\note{
	\noteparagraph{Versch. Varianten des Net-Stacks}
	Wie bereits in der Einführung gesagt wurde,
	bietet Contiki einen kompletten Netzwerk-Stack an.
	Es sind sogar verschiedene Varianten des Netzwerk-Stacks implementiert,
	so dass diejenige Variante gefunden werden kann,
	die den eigenen Ansprüchen am ehesten entspricht.

	\noteparagraph{Konzept transparent gegenüber Anwendungen}
	Mit anderen Worten: Wenn die mir zur Verfügung gestellten Varianten
	nicht zusagen, kann ich einfach eine eigene Variante entwickeln
	und ins System integrieren, ohne dass die Anwendungen auf den höheren
	Schichten umgeschrieben werden müssen.

	\noteparagraph{Oberste Schicht: Network}
	Die oberste Schicht, die wir im Net Stack betrachten, ist
	der \emph{Network--layer}. Dadrüber befindet sich noch die uIP (TCP/IP)
	Applikation. Sie gehört aber nicht mehr zum Net Stack, sondern ist eine
	mögliche Abstraktionsschicht zwischen Anwender und NetStack.
}
\note{
	Der NetStack gliedert sich in folgende Schichten, die wir in den nächsten
	Folien näher beleuten werden.

	\noteparagraph{Net Stack}
	\begin{itemize}
	\item 	Ganz unten befindet sich der \emph{Radio--layer}.
	\item 	Darüber ist dann ein Layer, der sich mit dem \emph{Radio Duty Cycle}
			beschäftigt.
	\item 	Ein \emph{Framer} erzeugt die Frames.
	\item 	Der \emph{MAC--layer} kümmert sich um die Organisation des 
			Versendens und Empfangens, also \emph{wie} versendet wird.
	\item 	Ganz oben ist der \emph{Network--layer} angesiedelt.
	\end{itemize}
}
%-------------------------------------------------------------------------------
\begin{frame}{Contiki Net Stack layers}{Die Konfiguration}
	\uncover<1->{\lstinputlisting[language=C,title={core/net/netstack.h}]{mac-xmpl/netstack.h}}
	\uncover<2->{\lstinputlisting[language=C,title={contiki-conf.h (Platform, Projekt)}]{mac-xmpl/contiki-conf.h}}
\end{frame}
%....... note .................................................................
\note{
	Bevor wir die einzelnen Layer und ihre Aufgaben vorstellen,
	wollen wir noch die Konfiguration des Net Stacks erklären.
	Diese ist nämlich ganz einfach.

	\noteparagraph{netstack.h}
	Im Contiki-Kernel sind die Treiber für den Net Stack durch C-Structs
	implementiert: \emph{Extern}, weil die Definition in den jeweiligen
	Net Stack Source Dateien vorhanden sind und nicht in der \emph{netstack.h};
	Diese Structs werden auf Makros gemappt, die beliebig definierbar sind.

	\noteparagraph{contiki-conf.h}
	Die Platform oder das Projekt legt fest, welche Implementation verwendet wird.
	Hier ist ein Beispiel aus der Platform \emph{avr-atmega128rfa1} zu sehen.
	Das Macro mit \emph{CONF} im Namen wird dann in der \emph{netstack.h}
	auf dasjenige ohne \emph{CONF} gemappt.
	Bei der Konfiguration hat die Projekt-Konfiguration i.d.R. vorrang
	vor der Platform-Konfiguration.

	Auf den folgenden Folien erklären wir EUCH nun die Aufgaben
	und die zur Zeit existierenden Varianten des Net Stacks.
}
%-------------------------------------------------------------------------------
\subsection{Aufgaben und Implementationen}
%-------------------------------------------------------------------------------
\begin{frame}{Radio -- layer}{Aufgaben und Implementationen}
	\begin{block}{Aufgaben}
		\begin{itemize}
			\item 	Kommunikation mit dem Funksensor
			\item 	CPU-abhängige Implementation
			\item 	Bestimmte Funktionalität auf die Hardware umlegen
					(Auto-ACK)
		\end{itemize}
	\end{block}

	\begin{block}{Implementationen (CPU)}
		\begin{itemize}
			\item 	NullRadio
			\item 	rf230 (AVR)
			\item 	cc2430\_rf (cc2430)
			\item 	cc2530\_rf (cc253x)
			\item 	stm32w (stm32w108)
			\item 	Contiki MACA (mc1322x)
		\end{itemize}
	\end{block}
\end{frame}
%....... note .................................................................
\note{
	\noteparagraph{Kommunikation, CPU-abhängig}
	Wir beginnen mit dem Radio--layer, denn dieser kommuniziert direkt
	mit der Hardware, also dem Funksensor. Demnach ist es auch logisch,
	dass es sich hierbei um cpu-abhängigen Code handelt, weshalb sich die
	Implementationen für z.B. avr-Mikrocontroller im Ordner /cpu/avr befinden.

	\noteparagraph{Funksensor-Funktionen}
	Je nach Funksensor können bestimmte Funktionen wie das Senden eines
	automatischen ACK auf die Hardware verlagert werden, d.h. die Software
	muss dieses nur einmalig einstellen.

	\noteparagraph{Implementation: AVR}
	Contiki stellt eine leere Radio-Implementation, NullRadio, bereit.
	Desweiteren exisitieren 5 verschiedene Funksensor-Implementationen.
	Für AVR-Mikrocontroller gibt es eine Implementation mit Namen \emph{rf230},
	welcher intern mit dem ATmega128rfa1 und dem AT86RF230-Funksensor
	umgehen kann (1- und 2-chip-Lösung).
}
\note{
	\noteparagraph{Implementation: ccXXXX}
	Die beliebte \emph{true System-on-chip-Solution} von Texas Intruments ist ebenfalls
	mit 2 Prozessoren vertreten: der \emph{cc2430} und die \emph{cc2530}-Reihe.

	\noteparagraph{Implementation: stm32w108}
	Auch ein 32bit ARM Cortex-M3 Micro-Prozessor, welcher ebenfalls 
	eine\emph{System-on-chip-Solution} ist, ist vertreten.
	
	\noteparagraph{Implementation: mc1322x}
	Freescale Funklösungen sind vertreten.

	\noteparagraph{Zusammenfassend} kann man sagen, dass alle wichtigen
	Microcontroller in Contiki vertreten sind.
}
%-------------------------------------------------------------------------------
\begin{frame}{Network -- layer}{Aufgaben und Implementationen}
	\begin{block}{Aufgaben}
		\begin{itemize}
			\item 	Schnittstelle zu den Applikationen
					(Empfangen und Versenden von Nachrichten)
			\item 	Interagiert mit dem MAC--layer
		\end{itemize}
	\end{block}

	\begin{block}{Implementationen}
		\begin{itemize}
			\item 	RIME
			\item 	SicsLowPAN
		\end{itemize}
	\end{block}
\end{frame}
%....... note .................................................................
\note{
	\noteparagraph{Gegenseite}
	Als Contiki-App-Entwickler möchte man vielleicht nicht alle Details
	der Radio-Implementationen kennen. Wir interessieren uns hauptsächlich
	dafür, wie wir Nachrichten empfangen und versenden können.
	Daher betrachten wir nun erstmal die Gegenseite.

	\noteparagraph{Network--layer}
	Der Network--layer hat zur Aufgabe, eine Schnittstelle zu den Applikationen
	anzubieten, welche möglichst konfortabel ist.
	Im Wesentlichen besteht diese nur aus init, output und input.
	Output und Input sind zwei Callbacks, die vor dem Senden/Empfangen gesetzt
	werden. In der Regel reicht es, dass der TCP/IP-Prozess die Callbacks stellt.
	Man kann aber auch eigene sogenannte \enquote{RIME-Sniffer}
	unabhängig vom TCP/IP-Prozess integrieren.

	\noteparagraph{RIME}
	Der NetStack-Part von RIME basiert ebenfalls auf diesen 3 Funktionen,
	obwohl der komplette RIME-Stack -- ähnlich zu dem uIP TCP/IP-Stack --
	eine Reihe zusätzlicher Funktionen bietet.
}
\note{
	\noteparagraph{RIME (Forts.)}
	Der RIME-Stack bietet eine Reihe von Hierarchisch geordneten
	Wireless Network Protocols, von einem anonymen Broadcast (\enquote{abs})
	über multihop-Funktionalitäten bis hin zum Mesh-Network-Routing.
	\prove{Dieses wird im Vortrag nächste Woche genauer behandelt werden.}

	\noteparagraph{6LoWPAN}
	Die zweite Network--layer--Variante ist 6LowPAN.
	\prove{Da wir 6LowPAN bereits in der Vorlesung hatten, werden wir hier nicht
	weiter darauf eingehen.} Es ist soviel zu sagen, dass es sowohl
	uIP-Funktionalitäten als auch RIME-Funktionalitäten nutzt.
}
%-------------------------------------------------------------------------------
\begin{frame}{Radio Duty Cycling -- layer}{Aufgaben und Implementationen}
	\begin{block}{Aufgaben}
		\begin{itemize}
			\item 	schaltet den Radio Transceiver so oft wie möglich aus
			\item 	sorgt damit für energie-effizientes Arbeiten
		\end{itemize}
	\end{block}

	\begin{block}{Implementationen}
		\begin{itemize}
			\item 	ContikiMAC (Low-Power-Listening, LPL)
			\item 	LPP (Low-Power Probing)
			\item 	X-MAC
			\item 	CX-MAC (Compatiblity X-MAC)
			\item 	SicsLowMAC \notall{(no Radio Cycling)}
				% see: http://permalink.gmane.org/gmane.os.contiki.devel/7432
			\item 	NullRDC \notall{(no Radio Cycling)}
			\item 	NullRDC No Framer \notall{(no Radio Cycling)}
		\end{itemize}
	\end{block}
\end{frame}
%....... note .................................................................
\note{
	Nachdem wir nun den Layer auf der obersten Ebene hatten, den Network--layer,
	und auch denjenigen auf der untersten Ebene, den Radio--layer.
	kommen wir nun zu den Zwischenebenen.

	\noteparagraph{Energie-effizienz}
	Da wir möglichst energie-effizient arbeiten wollen, benötigen wir
	eine Einheit, die genau dieses umsetzt.
	Hier kommt der Radio~Duty~Cycling--layer (kurz RDC) ins Spiel.
	Und deshalb hat der RDC--layer die Aufgabe, 
	den Radio--layer anzuweisen, dass der Radio Transceiver so oft wie möglich
	schläft.

	\noteparagraph{Varianten}
	Daher gibt es gerade an dieser Stelle sehr viele Varianten.
	Manche davon werden mit MAC bezeichnet, weshalb es zu Verwirrungen
	mit dem MAC--layer kommen kann.
}
\note{
	\noteparagraph{ContikiMAC}
	Die Variante, die im Contiki als Default eingestellt ist, ist das
	\emph{ContikiMAC}-Protokoll. Dieses Protokoll wird auch in die
	\emph{Low-Power-Listening}-Protokolle eingeordnet. Der Name sagt bereits,
	dass ein Netzwerknoten seinen Radio Transceiver für kurze Zeit anschaltet,
	um zu lauschen. Sollte kein Funkverkehr vorhanden sein, dann legt sich
	der Radio Transceiver wieder schlafen.
	Die \emph{Channel Check Rate} kann in 2er Hertz-Schritten eingestellt
	werden. Standard ist 8 Hz, also 125ms. Eine Rate größer als 16 Hz ist
	untypisch. Zusätzlich kann eine \emph{Phase Optimization}
	eingeschaltet werden, wodurch \todo{Phase Optim.}.

	\noteparagraph{X-MAC und CX-MAC}
	X-MAC sendet, um einen Empfänger zu erreichen,
	eine langen Strom sich wiederholender Pakete,
	bis der Empfänger aufwacht und antwortet.
}
\note{
	\noteparagraph{LPP}
	Beim Low-Power-Probing wird im Gegensatz zum LPL ein Beacon-Signal
	in regelmäßigen Abständen gesendet. Dieser Beacon zeigt an, dass
	ein Knoten für kurze Zeit bereit zum Empfangen ist.

	\noteparagraph{SicslowMAC und NullRDC}
	6LowMAC und NullRDC sind leichtgewichtige Protokolle und beinhalten
	KEIN \emph{Radio Cycling}! NullRDC--No~Framer verpackt die Daten sogar
	nicht in Frames. Daher ist kein energie-effizientes Arbeiten sichergestellt.
	Die RDC-Schnittstelle beinhaltet allerdings
	eine \emph{on} und \emph{off} Funktion, wodurch höhere Schichten
	diese Aufgabe übernehmen könnten.
}
\note{
	Ram requirements:
	\begin{itemize}
	\item 	ContikiMAC: 984
	\item 	LPP: 473
	\item 	X-MAC: 175
	\item 	CX-MAC: 115
	\item 	SicslowMAC: 25
	\item 	NullRDC: 21
	\end{itemize}
}
%-------------------------------------------------------------------------------
\begin{frame}{Framer -- layer}{Aufgaben und Implementationen}
	\begin{block}{Aufgaben}
		\begin{itemize}
			\item 	Erzeugen von Frames zum Versenden
			\item 	Parsen von Frames beim Empfangen
			\item 	Wird vom RDC--layer aufgerufen und legt die bearbeiteten
					Frames in einen Packet-Buffer
		\end{itemize}
	\end{block}

	\begin{block}{Implementationen}
		\begin{itemize}
			\item 	Framer 802.15.4
			\item 	Framer NullMAC
		\end{itemize}
	\end{block}
\end{frame}
%....... note .................................................................
\note{
	Bevor wir zum MAC--layer kommen, ist der Framer an der Reihe.
	Dieser arbeitet sehr eng mit dem RDC--layer zusammen, denn der
	RDC--layer sagt, ob und wann Frames erzeugt oder geparst werden.

	\noteparagraph{Framer 802.15.4}
	Der Framer erzeugt damit die 802.15.4-Frames.

	\noteparagraph{Framer NullMAC} hat weniger zu tun, nämlich
	schreibt bzw. liest er Empfanger- und Sender-Addresse aus bzw. in den
	Packet-Buffer.
}
%-------------------------------------------------------------------------------
\begin{frame}{MAC -- layer}{Aufgaben und Implementationen}
	\begin{block}{Aufgaben}
		\begin{itemize}
			\item 	empfängt eingehende Packete vom RDC layer
			\item 	benutzt den RDC layer, um Packete zu versenden
			\item 	\notall{sorgt dafür, dass Packete erneut gesendet werden, sofern
					der RDC layer eine Kollision detektiert hat}
		\end{itemize}
	\end{block}

	\begin{block}{Implementationen}
		\begin{itemize}
			\item 	CSMA (Carrier Sense Multiple Access)
			\item 	TDMA (Time Division Multiple Access)
			\item 	CTDMA (Carrier / Time Division Multiple Access)
			\item 	RIME UDP
			\item 	NullMAC
			\item 	SicsLowMAC \notall{(AVR, nutzt nicht RDC-Protokoll)}
		\end{itemize}
	\end{block}
\end{frame}
%....... note .................................................................
\note{
	\noteparagraph{Aufgaben MAC--layer}
	Der MAC--layer organisiert den Zugriff auf das Medium.
	Er nimmt also Packete vom RDC layer entgegen und reicht Packete
	vom Network--layer zum RDC--layer weiter.
	Bei einigen Protokollen werden Packete erneut gesendet, sofern
	Kollisionen erkannt wurden.

	\noteparagraph{CSMA und TDMA}
	sollten bekannte Verfahren aus der Mobilfunk-Technik sein, weshalb wir
	diese nicht extra erklären wollen.

	\noteparagraph{CTDMA}
	ist eine Mischung aus CSMA und TDMA.

	\noteparagraph{SicsLowMAC}
	ist nicht vergleichbar mit der SicslowMAC-Implementation aus dem Kernel.
	Es nutzt nicht die allgemeine NetStack-Implementation, sondern 
	nimmt seine eigene.
}
%-------------------------------------------------------------------------------
\subsection{Nachrichtenaustausch}
%-------------------------------------------------------------------------------
\begin{frame}{Nachrichtenaustausch}{Eintreffende Nachricht}
	% Graphic for TeX using PGF
% Title: /home/mkain/Documents/workspace/latex/contiki-os-htwdd/NetStackReceive.dia
% Creator: Dia v0.97.2
% CreationDate: Sat Nov 24 20:38:09 2012
% For: mkain
% \usepackage{tikz}
% The following commands are not supported in PSTricks at present
% We define them conditionally, so when they are implemented,
% this pgf file will use them.
\ifx\du\undefined
  \newlength{\du}
\fi
\setlength{\du}{15\unitlength}
\begin{tikzpicture}
\pgftransformxscale{1.000000}
\pgftransformyscale{-1.000000}
\definecolor{dialinecolor}{rgb}{0.000000, 0.000000, 0.000000}
\pgfsetstrokecolor{dialinecolor}
\definecolor{dialinecolor}{rgb}{1.000000, 1.000000, 1.000000}
\pgfsetfillcolor{dialinecolor}
\definecolor{dialinecolor}{rgb}{1.000000, 1.000000, 1.000000}
\pgfsetfillcolor{dialinecolor}
\fill (10.000000\du,10.000000\du)--(10.000000\du,11.900000\du)--(14.950000\du,11.900000\du)--(14.950000\du,10.000000\du)--cycle;
\pgfsetlinewidth{0.100000\du}
\pgfsetdash{}{0pt}
\pgfsetdash{}{0pt}
\pgfsetmiterjoin
\definecolor{dialinecolor}{rgb}{0.000000, 0.000000, 0.000000}
\pgfsetstrokecolor{dialinecolor}
\draw (10.000000\du,10.000000\du)--(10.000000\du,11.900000\du)--(14.950000\du,11.900000\du)--(14.950000\du,10.000000\du)--cycle;
% setfont left to latex
\definecolor{dialinecolor}{rgb}{0.000000, 0.000000, 0.000000}
\pgfsetstrokecolor{dialinecolor}
\node at (12.475000\du,11.145000\du){Network};
\definecolor{dialinecolor}{rgb}{1.000000, 1.000000, 1.000000}
\pgfsetfillcolor{dialinecolor}
\fill (10.000000\du,5.335000\du)--(10.000000\du,7.235000\du)--(14.850000\du,7.235000\du)--(14.850000\du,5.335000\du)--cycle;
\pgfsetlinewidth{0.100000\du}
\pgfsetdash{}{0pt}
\pgfsetdash{}{0pt}
\pgfsetmiterjoin
\definecolor{dialinecolor}{rgb}{0.000000, 0.000000, 0.000000}
\pgfsetstrokecolor{dialinecolor}
\draw (10.000000\du,5.335000\du)--(10.000000\du,7.235000\du)--(14.850000\du,7.235000\du)--(14.850000\du,5.335000\du)--cycle;
% setfont left to latex
\definecolor{dialinecolor}{rgb}{0.000000, 0.000000, 0.000000}
\pgfsetstrokecolor{dialinecolor}
\node at (12.425000\du,6.480000\du){RDC};
\definecolor{dialinecolor}{rgb}{1.000000, 1.000000, 1.000000}
\pgfsetfillcolor{dialinecolor}
\fill (10.000000\du,7.685000\du)--(10.000000\du,9.585000\du)--(14.900000\du,9.585000\du)--(14.900000\du,7.685000\du)--cycle;
\pgfsetlinewidth{0.100000\du}
\pgfsetdash{}{0pt}
\pgfsetdash{}{0pt}
\pgfsetmiterjoin
\definecolor{dialinecolor}{rgb}{0.000000, 0.000000, 0.000000}
\pgfsetstrokecolor{dialinecolor}
\draw (10.000000\du,7.685000\du)--(10.000000\du,9.585000\du)--(14.900000\du,9.585000\du)--(14.900000\du,7.685000\du)--cycle;
% setfont left to latex
\definecolor{dialinecolor}{rgb}{0.000000, 0.000000, 0.000000}
\pgfsetstrokecolor{dialinecolor}
\node at (12.450000\du,8.830000\du){MAC};
% setfont left to latex
\definecolor{dialinecolor}{rgb}{0.000000, 0.000000, 0.000000}
\pgfsetstrokecolor{dialinecolor}
\node[anchor=west] at (19.750000\du,7.250000\du){};
% setfont left to latex
\definecolor{dialinecolor}{rgb}{0.000000, 0.000000, 0.000000}
\pgfsetstrokecolor{dialinecolor}
\node[anchor=west] at (15.750000\du,4.650000\du){input};
% setfont left to latex
\definecolor{dialinecolor}{rgb}{0.000000, 0.000000, 0.000000}
\pgfsetstrokecolor{dialinecolor}
\node[anchor=west] at (15.870000\du,7.580000\du){input};
% setfont left to latex
\definecolor{dialinecolor}{rgb}{0.000000, 0.000000, 0.000000}
\pgfsetstrokecolor{dialinecolor}
\node[anchor=west] at (15.920000\du,9.717500\du){input};
% setfont left to latex
\definecolor{dialinecolor}{rgb}{0.000000, 0.000000, 0.000000}
\pgfsetstrokecolor{dialinecolor}
\node[anchor=west] at (14.400000\du,-0.712500\du){Nachricht};
\definecolor{dialinecolor}{rgb}{1.000000, 1.000000, 1.000000}
\pgfsetfillcolor{dialinecolor}
\fill (22.315000\du,5.372500\du)--(22.315000\du,7.272500\du)--(27.200000\du,7.272500\du)--(27.200000\du,5.372500\du)--cycle;
\pgfsetlinewidth{0.100000\du}
\pgfsetdash{}{0pt}
\pgfsetdash{}{0pt}
\pgfsetmiterjoin
\definecolor{dialinecolor}{rgb}{0.000000, 0.000000, 0.000000}
\pgfsetstrokecolor{dialinecolor}
\draw (22.315000\du,5.372500\du)--(22.315000\du,7.272500\du)--(27.200000\du,7.272500\du)--(27.200000\du,5.372500\du)--cycle;
% setfont left to latex
\definecolor{dialinecolor}{rgb}{0.000000, 0.000000, 0.000000}
\pgfsetstrokecolor{dialinecolor}
\node at (24.757500\du,6.517500\du){Framer};
\pgfsetlinewidth{0.100000\du}
\pgfsetdash{}{0pt}
\pgfsetdash{}{0pt}
\pgfsetbuttcap
{
\definecolor{dialinecolor}{rgb}{0.000000, 0.000000, 0.000000}
\pgfsetfillcolor{dialinecolor}
% was here!!!
\pgfsetarrowsend{stealth}
\definecolor{dialinecolor}{rgb}{0.000000, 0.000000, 0.000000}
\pgfsetstrokecolor{dialinecolor}
\draw (14.850000\du,6.285000\du)--(22.315000\du,6.322500\du);
}
% setfont left to latex
\definecolor{dialinecolor}{rgb}{0.000000, 0.000000, 0.000000}
\pgfsetstrokecolor{dialinecolor}
\node[anchor=west] at (17.665000\du,6.017500\du){parse};
\pgfsetlinewidth{0.100000\du}
\pgfsetdash{}{0pt}
\pgfsetdash{}{0pt}
\pgfsetbuttcap
{
\definecolor{dialinecolor}{rgb}{0.000000, 0.000000, 1.000000}
\pgfsetfillcolor{dialinecolor}
% was here!!!
\definecolor{dialinecolor}{rgb}{0.000000, 0.000000, 1.000000}
\pgfsetstrokecolor{dialinecolor}
\draw (22.767361\du,3.143056\du)--(24.757500\du,5.372500\du);
}
\pgfsetlinewidth{0.100000\du}
\pgfsetdash{}{0pt}
\pgfsetdash{}{0pt}
\pgfsetbuttcap
\pgfsetmiterjoin
\pgfsetlinewidth{0.100000\du}
\pgfsetbuttcap
\pgfsetmiterjoin
\pgfsetdash{}{0pt}
\definecolor{dialinecolor}{rgb}{1.000000, 1.000000, 1.000000}
\pgfsetfillcolor{dialinecolor}
\fill (9.965000\du,3.037500\du)--(12.410000\du,3.037500\du)--(11.187500\du,1.020745\du)--cycle;
\definecolor{dialinecolor}{rgb}{0.749020, 0.749020, 0.749020}
\pgfsetstrokecolor{dialinecolor}
\draw (9.965000\du,3.037500\du)--(12.410000\du,3.037500\du)--(11.187500\du,1.020745\du)--cycle;
% setfont left to latex
\definecolor{dialinecolor}{rgb}{0.000000, 0.000000, 0.000000}
\pgfsetstrokecolor{dialinecolor}
\node at (11.187500\du,2.710406\du){ISR};
\pgfsetlinewidth{0.100000\du}
\pgfsetdash{}{0pt}
\pgfsetdash{}{0pt}
\pgfsetbuttcap
{
\definecolor{dialinecolor}{rgb}{0.000000, 0.000000, 1.000000}
\pgfsetfillcolor{dialinecolor}
% was here!!!
\definecolor{dialinecolor}{rgb}{0.000000, 0.000000, 1.000000}
\pgfsetstrokecolor{dialinecolor}
\draw (14.800000\du,3.550000\du)--(19.669444\du,2.516667\du);
}
% setfont left to latex
\definecolor{dialinecolor}{rgb}{0.000000, 0.000000, 0.000000}
\pgfsetstrokecolor{dialinecolor}
\node[anchor=west] at (17.200000\du,12.137500\du){input};
\pgfsetlinewidth{0.100000\du}
\pgfsetdash{}{0pt}
\pgfsetdash{}{0pt}
\pgfsetmiterjoin
\pgfsetbuttcap
{
\definecolor{dialinecolor}{rgb}{0.000000, 0.000000, 1.000000}
\pgfsetfillcolor{dialinecolor}
% was here!!!
\definecolor{dialinecolor}{rgb}{0.000000, 0.000000, 1.000000}
\pgfsetstrokecolor{dialinecolor}
\pgfpathmoveto{\pgfpoint{21.734722\du}{3.143056\du}}
\pgfpathcurveto{\pgfpoint{20.034210\du}{5.208997\du}}{\pgfpoint{20.550000\du}{10.837500\du}}{\pgfpoint{14.950000\du}{10.950000\du}}
\pgfusepath{stroke}
}
\pgfsetlinewidth{0.100000\du}
\pgfsetdash{}{0pt}
\pgfsetdash{}{0pt}
\pgfsetbuttcap
\pgfsetmiterjoin
\pgfsetlinewidth{0.100000\du}
\pgfsetbuttcap
\pgfsetmiterjoin
\pgfsetdash{}{0pt}
\definecolor{dialinecolor}{rgb}{1.000000, 1.000000, 1.000000}
\pgfsetfillcolor{dialinecolor}
\fill (10.000000\du,3.087500\du)--(10.000000\du,4.937500\du)--(14.800000\du,4.937500\du)--(14.800000\du,3.087500\du)--cycle;
\definecolor{dialinecolor}{rgb}{0.000000, 0.000000, 0.000000}
\pgfsetstrokecolor{dialinecolor}
\draw (10.000000\du,3.087500\du)--(10.000000\du,4.937500\du)--(14.800000\du,4.937500\du)--(14.800000\du,3.087500\du)--cycle;
\pgfsetbuttcap
\pgfsetmiterjoin
\pgfsetdash{}{0pt}
\definecolor{dialinecolor}{rgb}{0.000000, 0.000000, 0.000000}
\pgfsetstrokecolor{dialinecolor}
\draw (10.480000\du,3.087500\du)--(10.480000\du,4.937500\du);
\pgfsetbuttcap
\pgfsetmiterjoin
\pgfsetdash{}{0pt}
\definecolor{dialinecolor}{rgb}{0.000000, 0.000000, 0.000000}
\pgfsetstrokecolor{dialinecolor}
\draw (14.320000\du,3.087500\du)--(14.320000\du,4.937500\du);
% setfont left to latex
\definecolor{dialinecolor}{rgb}{0.000000, 0.000000, 0.000000}
\pgfsetstrokecolor{dialinecolor}
\node at (12.400000\du,4.212500\du){Radio};
\pgfsetlinewidth{0.100000\du}
\pgfsetdash{}{0pt}
\pgfsetdash{}{0pt}
\pgfsetbuttcap
\pgfsetmiterjoin
\pgfsetlinewidth{0.100000\du}
\pgfsetbuttcap
\pgfsetmiterjoin
\pgfsetdash{}{0pt}
\definecolor{dialinecolor}{rgb}{1.000000, 1.000000, 1.000000}
\pgfsetfillcolor{dialinecolor}
\fill (19.669444\du,0.637500\du)--(19.669444\du,3.143056\du)--(23.800000\du,3.143056\du)--(23.800000\du,0.637500\du)--cycle;
\definecolor{dialinecolor}{rgb}{0.000000, 0.000000, 1.000000}
\pgfsetstrokecolor{dialinecolor}
\draw (19.669444\du,0.637500\du)--(19.669444\du,3.143056\du)--(23.800000\du,3.143056\du)--(23.800000\du,0.637500\du)--cycle;
\pgfsetbuttcap
\pgfsetmiterjoin
\pgfsetdash{}{0pt}
\definecolor{dialinecolor}{rgb}{0.000000, 0.000000, 1.000000}
\pgfsetstrokecolor{dialinecolor}
\draw (20.082500\du,0.637500\du)--(20.082500\du,3.143056\du);
\pgfsetbuttcap
\pgfsetmiterjoin
\pgfsetdash{}{0pt}
\definecolor{dialinecolor}{rgb}{0.000000, 0.000000, 1.000000}
\pgfsetstrokecolor{dialinecolor}
\draw (19.669444\du,0.888056\du)--(23.800000\du,0.888056\du);
% setfont left to latex
\definecolor{dialinecolor}{rgb}{0.000000, 0.000000, 0.000000}
\pgfsetstrokecolor{dialinecolor}
\node at (21.941250\du,1.815556\du){Packet};
% setfont left to latex
\definecolor{dialinecolor}{rgb}{0.000000, 0.000000, 0.000000}
\pgfsetstrokecolor{dialinecolor}
\node at (21.941250\du,2.615556\du){Buffer};
\pgfsetlinewidth{0.100000\du}
\pgfsetdash{}{0pt}
\pgfsetdash{}{0pt}
\pgfsetmiterjoin
\pgfsetbuttcap
{
\definecolor{dialinecolor}{rgb}{0.000000, 0.000000, 0.000000}
\pgfsetfillcolor{dialinecolor}
% was here!!!
\pgfsetarrowsend{stealth}
\definecolor{dialinecolor}{rgb}{0.000000, 0.000000, 0.000000}
\pgfsetstrokecolor{dialinecolor}
\pgfpathmoveto{\pgfpoint{14.800000\du}{4.475000\du}}
\pgfpathcurveto{\pgfpoint{15.900000\du}{4.537500\du}}{\pgfpoint{15.950000\du}{5.437500\du}}{\pgfpoint{14.850000\du}{5.810000\du}}
\pgfusepath{stroke}
}
\pgfsetlinewidth{0.100000\du}
\pgfsetdash{}{0pt}
\pgfsetdash{}{0pt}
\pgfsetmiterjoin
\pgfsetbuttcap
{
\definecolor{dialinecolor}{rgb}{0.000000, 0.000000, 1.000000}
\pgfsetfillcolor{dialinecolor}
% was here!!!
\definecolor{dialinecolor}{rgb}{0.000000, 0.000000, 1.000000}
\pgfsetstrokecolor{dialinecolor}
\pgfpathmoveto{\pgfpoint{20.702083\du}{3.143056\du}}
\pgfpathcurveto{\pgfpoint{19.150000\du}{4.137500\du}}{\pgfpoint{18.900000\du}{5.337500\du}}{\pgfpoint{14.850000\du}{6.285000\du}}
\pgfusepath{stroke}
}
\pgfsetlinewidth{0.100000\du}
\pgfsetdash{}{0pt}
\pgfsetdash{}{0pt}
\pgfsetmiterjoin
\pgfsetbuttcap
{
\definecolor{dialinecolor}{rgb}{0.000000, 0.000000, 0.000000}
\pgfsetfillcolor{dialinecolor}
% was here!!!
\pgfsetarrowsend{stealth}
\definecolor{dialinecolor}{rgb}{0.000000, 0.000000, 0.000000}
\pgfsetstrokecolor{dialinecolor}
\pgfpathmoveto{\pgfpoint{14.850000\du}{6.760000\du}}
\pgfpathcurveto{\pgfpoint{15.950000\du}{6.822500\du}}{\pgfpoint{16.000000\du}{7.787500\du}}{\pgfpoint{14.900000\du}{8.160000\du}}
\pgfusepath{stroke}
}
\pgfsetlinewidth{0.100000\du}
\pgfsetdash{}{0pt}
\pgfsetdash{}{0pt}
\pgfsetmiterjoin
\pgfsetbuttcap
{
\definecolor{dialinecolor}{rgb}{0.000000, 0.000000, 0.000000}
\pgfsetfillcolor{dialinecolor}
% was here!!!
\pgfsetarrowsend{stealth}
\definecolor{dialinecolor}{rgb}{0.000000, 0.000000, 0.000000}
\pgfsetstrokecolor{dialinecolor}
\pgfpathmoveto{\pgfpoint{14.900000\du}{9.110000\du}}
\pgfpathcurveto{\pgfpoint{16.000000\du}{9.172500\du}}{\pgfpoint{16.050000\du}{10.102500\du}}{\pgfpoint{14.950000\du}{10.475000\du}}
\pgfusepath{stroke}
}
\pgfsetlinewidth{0.100000\du}
\pgfsetdash{}{0pt}
\pgfsetdash{}{0pt}
\pgfsetbuttcap
\pgfsetmiterjoin
\pgfsetlinewidth{0.100000\du}
\pgfsetbuttcap
\pgfsetmiterjoin
\pgfsetdash{}{0pt}
\definecolor{dialinecolor}{rgb}{1.000000, 1.000000, 1.000000}
\pgfsetfillcolor{dialinecolor}
\fill (22.300000\du,10.437500\du)--(22.300000\du,12.437500\du)--(27.200000\du,12.437500\du)--(27.200000\du,10.437500\du)--cycle;
\definecolor{dialinecolor}{rgb}{0.000000, 0.000000, 0.000000}
\pgfsetstrokecolor{dialinecolor}
\draw (22.300000\du,10.437500\du)--(22.300000\du,12.437500\du)--(27.200000\du,12.437500\du)--(27.200000\du,10.437500\du)--cycle;
\pgfsetbuttcap
\pgfsetmiterjoin
\pgfsetdash{}{0pt}
\definecolor{dialinecolor}{rgb}{0.000000, 0.000000, 0.000000}
\pgfsetstrokecolor{dialinecolor}
\draw (22.790000\du,10.437500\du)--(22.790000\du,12.437500\du);
\pgfsetbuttcap
\pgfsetmiterjoin
\pgfsetdash{}{0pt}
\definecolor{dialinecolor}{rgb}{0.000000, 0.000000, 0.000000}
\pgfsetstrokecolor{dialinecolor}
\draw (26.710000\du,10.437500\du)--(26.710000\du,12.437500\du);
% setfont left to latex
\definecolor{dialinecolor}{rgb}{0.000000, 0.000000, 0.000000}
\pgfsetstrokecolor{dialinecolor}
\node at (24.750000\du,11.637500\du){APP};
\pgfsetlinewidth{0.100000\du}
\pgfsetdash{}{0pt}
\pgfsetdash{}{0pt}
\pgfsetbuttcap
{
\definecolor{dialinecolor}{rgb}{0.000000, 0.000000, 0.000000}
\pgfsetfillcolor{dialinecolor}
% was here!!!
\pgfsetarrowsend{stealth}
\definecolor{dialinecolor}{rgb}{0.000000, 0.000000, 0.000000}
\pgfsetstrokecolor{dialinecolor}
\draw (14.950000\du,11.425000\du)--(22.300000\du,11.437500\du);
}
\pgfsetlinewidth{0.100000\du}
\pgfsetdash{}{0pt}
\pgfsetdash{}{0pt}
\pgfsetbuttcap
{
\definecolor{dialinecolor}{rgb}{0.000000, 0.000000, 0.000000}
\pgfsetfillcolor{dialinecolor}
% was here!!!
\pgfsetarrowsend{stealth}
\definecolor{dialinecolor}{rgb}{0.000000, 0.000000, 0.000000}
\pgfsetstrokecolor{dialinecolor}
\draw (14.700000\du,-0.162500\du)--(12.410000\du,3.037500\du);
}
% setfont left to latex
\definecolor{dialinecolor}{rgb}{0.000000, 0.000000, 0.000000}
\pgfsetstrokecolor{dialinecolor}
\node[anchor=west] at (15.250000\du,12.987500\du){Callback / TCP/IP};
\end{tikzpicture}

\end{frame}
%....... note .................................................................
%\note{ }
%-------------------------------------------------------------------------------
\begin{frame}{Nachrichtenaustausch}{Ausgehende Nachricht (TCP/IP)}
	% Graphic for TeX using PGF
% Title: /home/mkain/Documents/workspace/latex/contiki-os-htwdd/figs/NetStackSend.dia
% Creator: Dia v0.97.2
% CreationDate: Sat Nov 24 21:37:43 2012
% For: mkain
% \usepackage{tikz}
% The following commands are not supported in PSTricks at present
% We define them conditionally, so when they are implemented,
% this pgf file will use them.
\ifx\du\undefined
  \newlength{\du}
\fi
\setlength{\du}{15\unitlength}
\begin{tikzpicture}
\pgftransformxscale{1.000000}
\pgftransformyscale{-1.000000}
\definecolor{dialinecolor}{rgb}{0.000000, 0.000000, 0.000000}
\pgfsetstrokecolor{dialinecolor}
\definecolor{dialinecolor}{rgb}{1.000000, 1.000000, 1.000000}
\pgfsetfillcolor{dialinecolor}
\definecolor{dialinecolor}{rgb}{1.000000, 1.000000, 1.000000}
\pgfsetfillcolor{dialinecolor}
\fill (10.000000\du,10.000000\du)--(10.000000\du,11.900000\du)--(14.950000\du,11.900000\du)--(14.950000\du,10.000000\du)--cycle;
\pgfsetlinewidth{0.100000\du}
\pgfsetdash{}{0pt}
\pgfsetdash{}{0pt}
\pgfsetmiterjoin
\definecolor{dialinecolor}{rgb}{0.000000, 0.000000, 0.000000}
\pgfsetstrokecolor{dialinecolor}
\draw (10.000000\du,10.000000\du)--(10.000000\du,11.900000\du)--(14.950000\du,11.900000\du)--(14.950000\du,10.000000\du)--cycle;
% setfont left to latex
\definecolor{dialinecolor}{rgb}{0.000000, 0.000000, 0.000000}
\pgfsetstrokecolor{dialinecolor}
\node at (12.475000\du,11.145000\du){Network};
\definecolor{dialinecolor}{rgb}{1.000000, 1.000000, 1.000000}
\pgfsetfillcolor{dialinecolor}
\fill (10.000000\du,5.335000\du)--(10.000000\du,7.235000\du)--(14.850000\du,7.235000\du)--(14.850000\du,5.335000\du)--cycle;
\pgfsetlinewidth{0.100000\du}
\pgfsetdash{}{0pt}
\pgfsetdash{}{0pt}
\pgfsetmiterjoin
\definecolor{dialinecolor}{rgb}{0.000000, 0.000000, 0.000000}
\pgfsetstrokecolor{dialinecolor}
\draw (10.000000\du,5.335000\du)--(10.000000\du,7.235000\du)--(14.850000\du,7.235000\du)--(14.850000\du,5.335000\du)--cycle;
% setfont left to latex
\definecolor{dialinecolor}{rgb}{0.000000, 0.000000, 0.000000}
\pgfsetstrokecolor{dialinecolor}
\node at (12.425000\du,6.480000\du){RDC};
\definecolor{dialinecolor}{rgb}{1.000000, 1.000000, 1.000000}
\pgfsetfillcolor{dialinecolor}
\fill (10.000000\du,7.685000\du)--(10.000000\du,9.585000\du)--(14.900000\du,9.585000\du)--(14.900000\du,7.685000\du)--cycle;
\pgfsetlinewidth{0.100000\du}
\pgfsetdash{}{0pt}
\pgfsetdash{}{0pt}
\pgfsetmiterjoin
\definecolor{dialinecolor}{rgb}{0.000000, 0.000000, 0.000000}
\pgfsetstrokecolor{dialinecolor}
\draw (10.000000\du,7.685000\du)--(10.000000\du,9.585000\du)--(14.900000\du,9.585000\du)--(14.900000\du,7.685000\du)--cycle;
% setfont left to latex
\definecolor{dialinecolor}{rgb}{0.000000, 0.000000, 0.000000}
\pgfsetstrokecolor{dialinecolor}
\node at (12.450000\du,8.830000\du){MAC};
% setfont left to latex
\definecolor{dialinecolor}{rgb}{0.000000, 0.000000, 0.000000}
\pgfsetstrokecolor{dialinecolor}
\node[anchor=west] at (19.750000\du,7.250000\du){};
% setfont left to latex
\definecolor{dialinecolor}{rgb}{0.000000, 0.000000, 0.000000}
\pgfsetstrokecolor{dialinecolor}
\node[anchor=west] at (16.020000\du,8.917500\du){send};
% setfont left to latex
\definecolor{dialinecolor}{rgb}{0.000000, 0.000000, 0.000000}
\pgfsetstrokecolor{dialinecolor}
\node[anchor=west] at (16.020000\du,9.717500\du){packet};
% setfont left to latex
\definecolor{dialinecolor}{rgb}{0.000000, 0.000000, 0.000000}
\pgfsetstrokecolor{dialinecolor}
\node[anchor=west] at (14.400000\du,-0.712500\du){Nachricht};
\definecolor{dialinecolor}{rgb}{1.000000, 1.000000, 1.000000}
\pgfsetfillcolor{dialinecolor}
\fill (22.315000\du,5.372500\du)--(22.315000\du,7.272500\du)--(27.200000\du,7.272500\du)--(27.200000\du,5.372500\du)--cycle;
\pgfsetlinewidth{0.100000\du}
\pgfsetdash{}{0pt}
\pgfsetdash{}{0pt}
\pgfsetmiterjoin
\definecolor{dialinecolor}{rgb}{0.000000, 0.000000, 0.000000}
\pgfsetstrokecolor{dialinecolor}
\draw (22.315000\du,5.372500\du)--(22.315000\du,7.272500\du)--(27.200000\du,7.272500\du)--(27.200000\du,5.372500\du)--cycle;
% setfont left to latex
\definecolor{dialinecolor}{rgb}{0.000000, 0.000000, 0.000000}
\pgfsetstrokecolor{dialinecolor}
\node at (24.757500\du,6.517500\du){Framer};
\pgfsetlinewidth{0.100000\du}
\pgfsetdash{}{0pt}
\pgfsetdash{}{0pt}
\pgfsetbuttcap
{
\definecolor{dialinecolor}{rgb}{0.000000, 0.000000, 0.000000}
\pgfsetfillcolor{dialinecolor}
% was here!!!
\pgfsetarrowsend{stealth}
\definecolor{dialinecolor}{rgb}{0.000000, 0.000000, 0.000000}
\pgfsetstrokecolor{dialinecolor}
\draw (14.850000\du,6.285000\du)--(22.315000\du,6.322500\du);
}
% setfont left to latex
\definecolor{dialinecolor}{rgb}{0.000000, 0.000000, 0.000000}
\pgfsetstrokecolor{dialinecolor}
\node[anchor=west] at (18.265000\du,6.017500\du){create};
\pgfsetlinewidth{0.100000\du}
\pgfsetdash{}{0pt}
\pgfsetdash{}{0pt}
\pgfsetbuttcap
{
\definecolor{dialinecolor}{rgb}{0.000000, 0.000000, 1.000000}
\pgfsetfillcolor{dialinecolor}
% was here!!!
\definecolor{dialinecolor}{rgb}{0.000000, 0.000000, 1.000000}
\pgfsetstrokecolor{dialinecolor}
\draw (22.767400\du,3.143060\du)--(24.757500\du,5.372500\du);
}
\pgfsetlinewidth{0.100000\du}
\pgfsetdash{}{0pt}
\pgfsetdash{}{0pt}
\pgfsetbuttcap
\pgfsetmiterjoin
\pgfsetlinewidth{0.100000\du}
\pgfsetbuttcap
\pgfsetmiterjoin
\pgfsetdash{}{0pt}
\definecolor{dialinecolor}{rgb}{1.000000, 1.000000, 1.000000}
\pgfsetfillcolor{dialinecolor}
\fill (9.965000\du,3.037495\du)--(12.410000\du,3.037495\du)--(11.187500\du,1.020740\du)--cycle;
\definecolor{dialinecolor}{rgb}{0.749020, 0.749020, 0.749020}
\pgfsetstrokecolor{dialinecolor}
\draw (9.965000\du,3.037495\du)--(12.410000\du,3.037495\du)--(11.187500\du,1.020740\du)--cycle;
% setfont left to latex
\definecolor{dialinecolor}{rgb}{0.000000, 0.000000, 0.000000}
\pgfsetstrokecolor{dialinecolor}
\node at (11.187500\du,2.710401\du){ISR};
\pgfsetlinewidth{0.100000\du}
\pgfsetdash{}{0pt}
\pgfsetdash{}{0pt}
\pgfsetbuttcap
{
\definecolor{dialinecolor}{rgb}{0.000000, 0.000000, 1.000000}
\pgfsetfillcolor{dialinecolor}
% was here!!!
\definecolor{dialinecolor}{rgb}{0.000000, 0.000000, 1.000000}
\pgfsetstrokecolor{dialinecolor}
\draw (14.800000\du,3.550000\du)--(19.669400\du,2.516670\du);
}
\pgfsetlinewidth{0.100000\du}
\pgfsetdash{}{0pt}
\pgfsetdash{}{0pt}
\pgfsetmiterjoin
\pgfsetbuttcap
{
\definecolor{dialinecolor}{rgb}{0.000000, 0.000000, 1.000000}
\pgfsetfillcolor{dialinecolor}
% was here!!!
\definecolor{dialinecolor}{rgb}{0.000000, 0.000000, 1.000000}
\pgfsetstrokecolor{dialinecolor}
\pgfpathmoveto{\pgfpoint{21.734700\du}{3.143060\du}}
\pgfpathcurveto{\pgfpoint{20.650000\du}{5.750010\du}}{\pgfpoint{21.050000\du}{10.700010\du}}{\pgfpoint{14.950000\du}{10.475000\du}}
\pgfusepath{stroke}
}
\pgfsetlinewidth{0.100000\du}
\pgfsetdash{}{0pt}
\pgfsetdash{}{0pt}
\pgfsetbuttcap
\pgfsetmiterjoin
\pgfsetlinewidth{0.100000\du}
\pgfsetbuttcap
\pgfsetmiterjoin
\pgfsetdash{}{0pt}
\definecolor{dialinecolor}{rgb}{1.000000, 1.000000, 1.000000}
\pgfsetfillcolor{dialinecolor}
\fill (10.000000\du,3.087500\du)--(10.000000\du,4.937500\du)--(14.800000\du,4.937500\du)--(14.800000\du,3.087500\du)--cycle;
\definecolor{dialinecolor}{rgb}{0.000000, 0.000000, 0.000000}
\pgfsetstrokecolor{dialinecolor}
\draw (10.000000\du,3.087500\du)--(10.000000\du,4.937500\du)--(14.800000\du,4.937500\du)--(14.800000\du,3.087500\du)--cycle;
\pgfsetbuttcap
\pgfsetmiterjoin
\pgfsetdash{}{0pt}
\definecolor{dialinecolor}{rgb}{0.000000, 0.000000, 0.000000}
\pgfsetstrokecolor{dialinecolor}
\draw (10.480000\du,3.087500\du)--(10.480000\du,4.937500\du);
\pgfsetbuttcap
\pgfsetmiterjoin
\pgfsetdash{}{0pt}
\definecolor{dialinecolor}{rgb}{0.000000, 0.000000, 0.000000}
\pgfsetstrokecolor{dialinecolor}
\draw (14.320000\du,3.087500\du)--(14.320000\du,4.937500\du);
% setfont left to latex
\definecolor{dialinecolor}{rgb}{0.000000, 0.000000, 0.000000}
\pgfsetstrokecolor{dialinecolor}
\node at (12.400000\du,4.212500\du){Radio};
\pgfsetlinewidth{0.100000\du}
\pgfsetdash{}{0pt}
\pgfsetdash{}{0pt}
\pgfsetbuttcap
\pgfsetmiterjoin
\pgfsetlinewidth{0.100000\du}
\pgfsetbuttcap
\pgfsetmiterjoin
\pgfsetdash{}{0pt}
\definecolor{dialinecolor}{rgb}{1.000000, 1.000000, 1.000000}
\pgfsetfillcolor{dialinecolor}
\fill (19.669400\du,0.637500\du)--(19.669400\du,3.143056\du)--(23.799956\du,3.143056\du)--(23.799956\du,0.637500\du)--cycle;
\definecolor{dialinecolor}{rgb}{0.000000, 0.000000, 1.000000}
\pgfsetstrokecolor{dialinecolor}
\draw (19.669400\du,0.637500\du)--(19.669400\du,3.143056\du)--(23.799956\du,3.143056\du)--(23.799956\du,0.637500\du)--cycle;
\pgfsetbuttcap
\pgfsetmiterjoin
\pgfsetdash{}{0pt}
\definecolor{dialinecolor}{rgb}{0.000000, 0.000000, 1.000000}
\pgfsetstrokecolor{dialinecolor}
\draw (20.082456\du,0.637500\du)--(20.082456\du,3.143056\du);
\pgfsetbuttcap
\pgfsetmiterjoin
\pgfsetdash{}{0pt}
\definecolor{dialinecolor}{rgb}{0.000000, 0.000000, 1.000000}
\pgfsetstrokecolor{dialinecolor}
\draw (19.669400\du,0.888056\du)--(23.799956\du,0.888056\du);
% setfont left to latex
\definecolor{dialinecolor}{rgb}{0.000000, 0.000000, 0.000000}
\pgfsetstrokecolor{dialinecolor}
\node at (21.941206\du,1.815556\du){Packet};
% setfont left to latex
\definecolor{dialinecolor}{rgb}{0.000000, 0.000000, 0.000000}
\pgfsetstrokecolor{dialinecolor}
\node at (21.941206\du,2.615556\du){Buffer};
\pgfsetlinewidth{0.100000\du}
\pgfsetdash{}{0pt}
\pgfsetdash{}{0pt}
\pgfsetmiterjoin
\pgfsetbuttcap
{
\definecolor{dialinecolor}{rgb}{0.000000, 0.000000, 1.000000}
\pgfsetfillcolor{dialinecolor}
% was here!!!
\definecolor{dialinecolor}{rgb}{0.000000, 0.000000, 1.000000}
\pgfsetstrokecolor{dialinecolor}
\pgfpathmoveto{\pgfpoint{20.702100\du}{3.143060\du}}
\pgfpathcurveto{\pgfpoint{20.000000\du}{3.950010\du}}{\pgfpoint{19.900000\du}{5.650010\du}}{\pgfpoint{14.850000\du}{6.285000\du}}
\pgfusepath{stroke}
}
\pgfsetlinewidth{0.100000\du}
\pgfsetdash{}{0pt}
\pgfsetdash{}{0pt}
\pgfsetbuttcap
\pgfsetmiterjoin
\pgfsetlinewidth{0.100000\du}
\pgfsetbuttcap
\pgfsetmiterjoin
\pgfsetdash{}{0pt}
\definecolor{dialinecolor}{rgb}{1.000000, 1.000000, 1.000000}
\pgfsetfillcolor{dialinecolor}
\fill (22.300000\du,11.487500\du)--(22.300000\du,13.487500\du)--(27.200000\du,13.487500\du)--(27.200000\du,11.487500\du)--cycle;
\definecolor{dialinecolor}{rgb}{0.000000, 0.000000, 0.000000}
\pgfsetstrokecolor{dialinecolor}
\draw (22.300000\du,11.487500\du)--(22.300000\du,13.487500\du)--(27.200000\du,13.487500\du)--(27.200000\du,11.487500\du)--cycle;
\pgfsetbuttcap
\pgfsetmiterjoin
\pgfsetdash{}{0pt}
\definecolor{dialinecolor}{rgb}{0.000000, 0.000000, 0.000000}
\pgfsetstrokecolor{dialinecolor}
\draw (22.790000\du,11.487500\du)--(22.790000\du,13.487500\du);
\pgfsetbuttcap
\pgfsetmiterjoin
\pgfsetdash{}{0pt}
\definecolor{dialinecolor}{rgb}{0.000000, 0.000000, 0.000000}
\pgfsetstrokecolor{dialinecolor}
\draw (26.710000\du,11.487500\du)--(26.710000\du,13.487500\du);
% setfont left to latex
\definecolor{dialinecolor}{rgb}{0.000000, 0.000000, 0.000000}
\pgfsetstrokecolor{dialinecolor}
\node at (24.750000\du,12.687500\du){APP};
\pgfsetlinewidth{0.100000\du}
\pgfsetdash{}{0pt}
\pgfsetdash{}{0pt}
\pgfsetbuttcap
{
\definecolor{dialinecolor}{rgb}{0.000000, 0.000000, 0.000000}
\pgfsetfillcolor{dialinecolor}
% was here!!!
\pgfsetarrowsend{stealth}
\definecolor{dialinecolor}{rgb}{0.000000, 0.000000, 0.000000}
\pgfsetstrokecolor{dialinecolor}
\draw (24.750000\du,11.487500\du)--(24.765000\du,10.005010\du);
}
\pgfsetlinewidth{0.100000\du}
\pgfsetdash{}{0pt}
\pgfsetdash{}{0pt}
\pgfsetbuttcap
{
\definecolor{dialinecolor}{rgb}{0.000000, 0.000000, 0.000000}
\pgfsetfillcolor{dialinecolor}
% was here!!!
\pgfsetarrowsend{stealth}
\definecolor{dialinecolor}{rgb}{0.000000, 0.000000, 0.000000}
\pgfsetstrokecolor{dialinecolor}
\draw (12.410000\du,3.037495\du)--(14.600000\du,-0.212490\du);
}
\pgfsetlinewidth{0.100000\du}
\pgfsetdash{}{0pt}
\pgfsetdash{}{0pt}
\pgfsetmiterjoin
\pgfsetbuttcap
{
\definecolor{dialinecolor}{rgb}{1.000000, 0.000000, 0.000000}
\pgfsetfillcolor{dialinecolor}
% was here!!!
\pgfsetarrowsend{stealth}
\definecolor{dialinecolor}{rgb}{1.000000, 0.000000, 0.000000}
\pgfsetstrokecolor{dialinecolor}
\pgfpathmoveto{\pgfpoint{10.000000\du}{5.810000\du}}
\pgfpathcurveto{\pgfpoint{6.850000\du}{6.535010\du}}{\pgfpoint{7.050000\du}{9.750010\du}}{\pgfpoint{10.000000\du}{10.475000\du}}
\pgfusepath{stroke}
}
% setfont left to latex
\definecolor{dialinecolor}{rgb}{1.000000, 0.000000, 0.000000}
\pgfsetstrokecolor{dialinecolor}
\node[anchor=west] at (6.715000\du,5.350010\du){packet};
% setfont left to latex
\definecolor{dialinecolor}{rgb}{1.000000, 0.000000, 0.000000}
\pgfsetstrokecolor{dialinecolor}
\node[anchor=west] at (6.715000\du,6.150010\du){sent};
\pgfsetlinewidth{0.100000\du}
\pgfsetdash{}{0pt}
\pgfsetdash{}{0pt}
\pgfsetmiterjoin
\pgfsetbuttcap
{
\definecolor{dialinecolor}{rgb}{0.000000, 0.000000, 0.000000}
\pgfsetfillcolor{dialinecolor}
% was here!!!
\pgfsetarrowsend{stealth}
\definecolor{dialinecolor}{rgb}{0.000000, 0.000000, 0.000000}
\pgfsetstrokecolor{dialinecolor}
\pgfpathmoveto{\pgfpoint{22.315000\du}{9.005010\du}}
\pgfpathcurveto{\pgfpoint{17.550000\du}{9.150010\du}}{\pgfpoint{21.100000\du}{11.200010\du}}{\pgfpoint{14.950000\du}{10.950000\du}}
\pgfusepath{stroke}
}
% setfont left to latex
\definecolor{dialinecolor}{rgb}{0.000000, 0.000000, 0.000000}
\pgfsetstrokecolor{dialinecolor}
\node[anchor=west] at (19.865000\du,8.700010\du){output};
\pgfsetlinewidth{0.100000\du}
\pgfsetdash{}{0pt}
\pgfsetdash{}{0pt}
\pgfsetmiterjoin
\pgfsetbuttcap
{
\definecolor{dialinecolor}{rgb}{0.000000, 0.000000, 0.000000}
\pgfsetfillcolor{dialinecolor}
% was here!!!
\pgfsetarrowsend{stealth}
\definecolor{dialinecolor}{rgb}{0.000000, 0.000000, 0.000000}
\pgfsetstrokecolor{dialinecolor}
\pgfpathmoveto{\pgfpoint{14.950000\du}{10.475000\du}}
\pgfpathcurveto{\pgfpoint{16.150000\du}{10.500010\du}}{\pgfpoint{16.200000\du}{9.250010\du}}{\pgfpoint{14.900000\du}{9.110000\du}}
\pgfusepath{stroke}
}
\pgfsetlinewidth{0.100000\du}
\pgfsetdash{}{0pt}
\pgfsetdash{}{0pt}
\pgfsetmiterjoin
\pgfsetbuttcap
{
\definecolor{dialinecolor}{rgb}{0.000000, 0.000000, 0.000000}
\pgfsetfillcolor{dialinecolor}
% was here!!!
\pgfsetarrowsend{stealth}
\definecolor{dialinecolor}{rgb}{0.000000, 0.000000, 0.000000}
\pgfsetstrokecolor{dialinecolor}
\pgfpathmoveto{\pgfpoint{14.900000\du}{8.160000\du}}
\pgfpathcurveto{\pgfpoint{16.100000\du}{8.185010\du}}{\pgfpoint{16.150000\du}{6.900010\du}}{\pgfpoint{14.850000\du}{6.760000\du}}
\pgfusepath{stroke}
}
\pgfsetlinewidth{0.100000\du}
\pgfsetdash{}{0pt}
\pgfsetdash{}{0pt}
\pgfsetmiterjoin
\pgfsetbuttcap
{
\definecolor{dialinecolor}{rgb}{0.000000, 0.000000, 0.000000}
\pgfsetfillcolor{dialinecolor}
% was here!!!
\pgfsetarrowsend{stealth}
\definecolor{dialinecolor}{rgb}{0.000000, 0.000000, 0.000000}
\pgfsetstrokecolor{dialinecolor}
\pgfpathmoveto{\pgfpoint{14.850000\du}{5.810000\du}}
\pgfpathcurveto{\pgfpoint{16.050000\du}{5.835010\du}}{\pgfpoint{16.100000\du}{4.615010\du}}{\pgfpoint{14.800000\du}{4.475000\du}}
\pgfusepath{stroke}
}
\pgfsetlinewidth{0.100000\du}
\pgfsetdash{}{0pt}
\pgfsetdash{}{0pt}
\pgfsetbuttcap
\pgfsetmiterjoin
\pgfsetlinewidth{0.100000\du}
\pgfsetbuttcap
\pgfsetmiterjoin
\pgfsetdash{}{0pt}
\definecolor{dialinecolor}{rgb}{1.000000, 1.000000, 1.000000}
\pgfsetfillcolor{dialinecolor}
\fill (22.315000\du,8.005010\du)--(22.315000\du,10.005010\du)--(27.215000\du,10.005010\du)--(27.215000\du,8.005010\du)--cycle;
\definecolor{dialinecolor}{rgb}{0.000000, 0.000000, 0.000000}
\pgfsetstrokecolor{dialinecolor}
\draw (22.315000\du,8.005010\du)--(22.315000\du,10.005010\du)--(27.215000\du,10.005010\du)--(27.215000\du,8.005010\du)--cycle;
\pgfsetbuttcap
\pgfsetmiterjoin
\pgfsetdash{}{0pt}
\definecolor{dialinecolor}{rgb}{0.000000, 0.000000, 0.000000}
\pgfsetstrokecolor{dialinecolor}
\draw (22.805000\du,8.005010\du)--(22.805000\du,10.005010\du);
\pgfsetbuttcap
\pgfsetmiterjoin
\pgfsetdash{}{0pt}
\definecolor{dialinecolor}{rgb}{0.000000, 0.000000, 0.000000}
\pgfsetstrokecolor{dialinecolor}
\draw (26.725000\du,8.005010\du)--(26.725000\du,10.005010\du);
% setfont left to latex
\definecolor{dialinecolor}{rgb}{0.000000, 0.000000, 0.000000}
\pgfsetstrokecolor{dialinecolor}
\node at (24.765000\du,9.205010\du){TCP/IP};
\pgfsetlinewidth{0.100000\du}
\pgfsetdash{}{0pt}
\pgfsetdash{}{0pt}
\pgfsetmiterjoin
\pgfsetbuttcap
{
\definecolor{dialinecolor}{rgb}{0.176471, 0.596078, 0.176471}
\pgfsetfillcolor{dialinecolor}
% was here!!!
\pgfsetarrowsend{stealth}
\definecolor{dialinecolor}{rgb}{0.176471, 0.596078, 0.176471}
\pgfsetstrokecolor{dialinecolor}
\pgfpathmoveto{\pgfpoint{14.950000\du}{11.900000\du}}
\pgfpathcurveto{\pgfpoint{19.900000\du}{11.900010\du}}{\pgfpoint{22.300000\du}{12.150010\du}}{\pgfpoint{22.315000\du}{10.005010\du}}
\pgfusepath{stroke}
}
% setfont left to latex
\definecolor{dialinecolor}{rgb}{0.176471, 0.596078, 0.176471}
\pgfsetstrokecolor{dialinecolor}
\node[anchor=west] at (16.815000\du,11.600010\du){set output};
% setfont left to latex
\definecolor{dialinecolor}{rgb}{0.000000, 0.000000, 0.000000}
\pgfsetstrokecolor{dialinecolor}
\node[anchor=west] at (24.665000\du,10.950010\du){tcpip output};
% setfont left to latex
\definecolor{dialinecolor}{rgb}{0.000000, 0.000000, 0.000000}
\pgfsetstrokecolor{dialinecolor}
\node[anchor=west] at (15.915000\du,7.000010\du){send};
% setfont left to latex
\definecolor{dialinecolor}{rgb}{0.000000, 0.000000, 0.000000}
\pgfsetstrokecolor{dialinecolor}
\node[anchor=west] at (15.915000\du,7.800010\du){packet};
% setfont left to latex
\definecolor{dialinecolor}{rgb}{0.000000, 0.000000, 0.000000}
\pgfsetstrokecolor{dialinecolor}
\node[anchor=west] at (15.715000\du,4.300010\du){send};
% setfont left to latex
\definecolor{dialinecolor}{rgb}{0.000000, 0.000000, 0.000000}
\pgfsetstrokecolor{dialinecolor}
\node[anchor=west] at (15.715000\du,5.100010\du){packet ?};
\pgfsetlinewidth{0.100000\du}
\pgfsetdash{}{0pt}
\pgfsetdash{}{0pt}
\pgfsetmiterjoin
\pgfsetbuttcap
{
\definecolor{dialinecolor}{rgb}{1.000000, 0.000000, 0.000000}
\pgfsetfillcolor{dialinecolor}
% was here!!!
\pgfsetarrowsend{stealth}
\definecolor{dialinecolor}{rgb}{1.000000, 0.000000, 0.000000}
\pgfsetstrokecolor{dialinecolor}
\pgfpathmoveto{\pgfpoint{13.712500\du}{11.900000\du}}
\pgfpathcurveto{\pgfpoint{13.500000\du}{13.750010\du}}{\pgfpoint{18.300000\du}{13.000010\du}}{\pgfpoint{22.300000\du}{12.987500\du}}
\pgfusepath{stroke}
}
% setfont left to latex
\definecolor{dialinecolor}{rgb}{1.000000, 0.000000, 0.000000}
\pgfsetstrokecolor{dialinecolor}
\node[anchor=west] at (11.415000\du,12.700010\du){output};
% setfont left to latex
\definecolor{dialinecolor}{rgb}{1.000000, 0.000000, 0.000000}
\pgfsetstrokecolor{dialinecolor}
\node[anchor=west] at (11.415000\du,13.500010\du){Callback};
\end{tikzpicture}

\end{frame}
%....... note .................................................................
%\note{ }
%-------------------------------------------------------------------------------
\begin{frame}{Nachrichtenaustausch}{Ausgehende Nachricht (RIME)}
	% Graphic for TeX using PGF
% Title: /home/mkain/Documents/workspace/latex/contiki-os-htwdd/figs/NetStackSendRIME.dia
% Creator: Dia v0.97.2
% CreationDate: Sun Nov 25 11:45:47 2012
% For: mkain
% \usepackage{tikz}
% The following commands are not supported in PSTricks at present
% We define them conditionally, so when they are implemented,
% this pgf file will use them.
\ifx\du\undefined
  \newlength{\du}
\fi
\setlength{\du}{15\unitlength}
\begin{tikzpicture}
\pgftransformxscale{1.000000}
\pgftransformyscale{-1.000000}
\definecolor{dialinecolor}{rgb}{0.000000, 0.000000, 0.000000}
\pgfsetstrokecolor{dialinecolor}
\definecolor{dialinecolor}{rgb}{1.000000, 1.000000, 1.000000}
\pgfsetfillcolor{dialinecolor}
\definecolor{dialinecolor}{rgb}{1.000000, 1.000000, 1.000000}
\pgfsetfillcolor{dialinecolor}
\fill (10.000000\du,10.000000\du)--(10.000000\du,11.900000\du)--(14.950000\du,11.900000\du)--(14.950000\du,10.000000\du)--cycle;
\pgfsetlinewidth{0.100000\du}
\pgfsetdash{}{0pt}
\pgfsetdash{}{0pt}
\pgfsetmiterjoin
\definecolor{dialinecolor}{rgb}{0.000000, 0.000000, 0.000000}
\pgfsetstrokecolor{dialinecolor}
\draw (10.000000\du,10.000000\du)--(10.000000\du,11.900000\du)--(14.950000\du,11.900000\du)--(14.950000\du,10.000000\du)--cycle;
% setfont left to latex
\definecolor{dialinecolor}{rgb}{0.000000, 0.000000, 0.000000}
\pgfsetstrokecolor{dialinecolor}
\node at (12.475000\du,10.745000\du){Network};
% setfont left to latex
\definecolor{dialinecolor}{rgb}{0.000000, 0.000000, 0.000000}
\pgfsetstrokecolor{dialinecolor}
\node at (12.475000\du,11.545000\du){RIME};
\definecolor{dialinecolor}{rgb}{1.000000, 1.000000, 1.000000}
\pgfsetfillcolor{dialinecolor}
\fill (10.000000\du,5.335000\du)--(10.000000\du,7.235000\du)--(14.850000\du,7.235000\du)--(14.850000\du,5.335000\du)--cycle;
\pgfsetlinewidth{0.100000\du}
\pgfsetdash{}{0pt}
\pgfsetdash{}{0pt}
\pgfsetmiterjoin
\definecolor{dialinecolor}{rgb}{0.000000, 0.000000, 0.000000}
\pgfsetstrokecolor{dialinecolor}
\draw (10.000000\du,5.335000\du)--(10.000000\du,7.235000\du)--(14.850000\du,7.235000\du)--(14.850000\du,5.335000\du)--cycle;
% setfont left to latex
\definecolor{dialinecolor}{rgb}{0.000000, 0.000000, 0.000000}
\pgfsetstrokecolor{dialinecolor}
\node at (12.425000\du,6.480000\du){RDC};
\definecolor{dialinecolor}{rgb}{1.000000, 1.000000, 1.000000}
\pgfsetfillcolor{dialinecolor}
\fill (10.000000\du,7.685000\du)--(10.000000\du,9.585000\du)--(14.900000\du,9.585000\du)--(14.900000\du,7.685000\du)--cycle;
\pgfsetlinewidth{0.100000\du}
\pgfsetdash{}{0pt}
\pgfsetdash{}{0pt}
\pgfsetmiterjoin
\definecolor{dialinecolor}{rgb}{0.000000, 0.000000, 0.000000}
\pgfsetstrokecolor{dialinecolor}
\draw (10.000000\du,7.685000\du)--(10.000000\du,9.585000\du)--(14.900000\du,9.585000\du)--(14.900000\du,7.685000\du)--cycle;
% setfont left to latex
\definecolor{dialinecolor}{rgb}{0.000000, 0.000000, 0.000000}
\pgfsetstrokecolor{dialinecolor}
\node at (12.450000\du,8.830000\du){MAC};
% setfont left to latex
\definecolor{dialinecolor}{rgb}{0.000000, 0.000000, 0.000000}
\pgfsetstrokecolor{dialinecolor}
\node[anchor=west] at (19.750000\du,7.250000\du){};
% setfont left to latex
\definecolor{dialinecolor}{rgb}{0.000000, 0.000000, 0.000000}
\pgfsetstrokecolor{dialinecolor}
\node[anchor=west] at (16.020000\du,8.917500\du){send};
% setfont left to latex
\definecolor{dialinecolor}{rgb}{0.000000, 0.000000, 0.000000}
\pgfsetstrokecolor{dialinecolor}
\node[anchor=west] at (16.020000\du,9.717500\du){packet};
% setfont left to latex
\definecolor{dialinecolor}{rgb}{0.000000, 0.000000, 0.000000}
\pgfsetstrokecolor{dialinecolor}
\node[anchor=west] at (14.400000\du,-0.712500\du){Nachricht};
\definecolor{dialinecolor}{rgb}{1.000000, 1.000000, 1.000000}
\pgfsetfillcolor{dialinecolor}
\fill (22.315000\du,5.372500\du)--(22.315000\du,7.272500\du)--(27.200000\du,7.272500\du)--(27.200000\du,5.372500\du)--cycle;
\pgfsetlinewidth{0.100000\du}
\pgfsetdash{}{0pt}
\pgfsetdash{}{0pt}
\pgfsetmiterjoin
\definecolor{dialinecolor}{rgb}{0.000000, 0.000000, 0.000000}
\pgfsetstrokecolor{dialinecolor}
\draw (22.315000\du,5.372500\du)--(22.315000\du,7.272500\du)--(27.200000\du,7.272500\du)--(27.200000\du,5.372500\du)--cycle;
% setfont left to latex
\definecolor{dialinecolor}{rgb}{0.000000, 0.000000, 0.000000}
\pgfsetstrokecolor{dialinecolor}
\node at (24.757500\du,6.517500\du){Framer};
\pgfsetlinewidth{0.100000\du}
\pgfsetdash{}{0pt}
\pgfsetdash{}{0pt}
\pgfsetbuttcap
{
\definecolor{dialinecolor}{rgb}{0.000000, 0.000000, 0.000000}
\pgfsetfillcolor{dialinecolor}
% was here!!!
\pgfsetarrowsend{stealth}
\definecolor{dialinecolor}{rgb}{0.000000, 0.000000, 0.000000}
\pgfsetstrokecolor{dialinecolor}
\draw (14.850000\du,6.285000\du)--(22.315000\du,6.322500\du);
}
% setfont left to latex
\definecolor{dialinecolor}{rgb}{0.000000, 0.000000, 0.000000}
\pgfsetstrokecolor{dialinecolor}
\node[anchor=west] at (18.265000\du,6.017500\du){create};
\pgfsetlinewidth{0.100000\du}
\pgfsetdash{}{0pt}
\pgfsetdash{}{0pt}
\pgfsetbuttcap
{
\definecolor{dialinecolor}{rgb}{0.000000, 0.000000, 1.000000}
\pgfsetfillcolor{dialinecolor}
% was here!!!
\definecolor{dialinecolor}{rgb}{0.000000, 0.000000, 1.000000}
\pgfsetstrokecolor{dialinecolor}
\draw (22.767400\du,3.143060\du)--(24.757500\du,5.372500\du);
}
\pgfsetlinewidth{0.100000\du}
\pgfsetdash{}{0pt}
\pgfsetdash{}{0pt}
\pgfsetbuttcap
\pgfsetmiterjoin
\pgfsetlinewidth{0.100000\du}
\pgfsetbuttcap
\pgfsetmiterjoin
\pgfsetdash{}{0pt}
\definecolor{dialinecolor}{rgb}{1.000000, 1.000000, 1.000000}
\pgfsetfillcolor{dialinecolor}
\fill (9.965000\du,3.037495\du)--(12.410000\du,3.037495\du)--(11.187500\du,1.020740\du)--cycle;
\definecolor{dialinecolor}{rgb}{0.749020, 0.749020, 0.749020}
\pgfsetstrokecolor{dialinecolor}
\draw (9.965000\du,3.037495\du)--(12.410000\du,3.037495\du)--(11.187500\du,1.020740\du)--cycle;
% setfont left to latex
\definecolor{dialinecolor}{rgb}{0.000000, 0.000000, 0.000000}
\pgfsetstrokecolor{dialinecolor}
\node at (11.187500\du,2.710401\du){ISR};
\pgfsetlinewidth{0.100000\du}
\pgfsetdash{}{0pt}
\pgfsetdash{}{0pt}
\pgfsetbuttcap
{
\definecolor{dialinecolor}{rgb}{0.000000, 0.000000, 1.000000}
\pgfsetfillcolor{dialinecolor}
% was here!!!
\definecolor{dialinecolor}{rgb}{0.000000, 0.000000, 1.000000}
\pgfsetstrokecolor{dialinecolor}
\draw (14.800000\du,3.550000\du)--(19.669400\du,2.516670\du);
}
\pgfsetlinewidth{0.100000\du}
\pgfsetdash{}{0pt}
\pgfsetdash{}{0pt}
\pgfsetmiterjoin
\pgfsetbuttcap
{
\definecolor{dialinecolor}{rgb}{0.000000, 0.000000, 1.000000}
\pgfsetfillcolor{dialinecolor}
% was here!!!
\definecolor{dialinecolor}{rgb}{0.000000, 0.000000, 1.000000}
\pgfsetstrokecolor{dialinecolor}
\pgfpathmoveto{\pgfpoint{21.734700\du}{3.143060\du}}
\pgfpathcurveto{\pgfpoint{20.650000\du}{5.750010\du}}{\pgfpoint{21.050000\du}{10.700000\du}}{\pgfpoint{14.950000\du}{10.475000\du}}
\pgfusepath{stroke}
}
\pgfsetlinewidth{0.100000\du}
\pgfsetdash{}{0pt}
\pgfsetdash{}{0pt}
\pgfsetbuttcap
\pgfsetmiterjoin
\pgfsetlinewidth{0.100000\du}
\pgfsetbuttcap
\pgfsetmiterjoin
\pgfsetdash{}{0pt}
\definecolor{dialinecolor}{rgb}{1.000000, 1.000000, 1.000000}
\pgfsetfillcolor{dialinecolor}
\fill (10.000000\du,3.087500\du)--(10.000000\du,4.937500\du)--(14.800000\du,4.937500\du)--(14.800000\du,3.087500\du)--cycle;
\definecolor{dialinecolor}{rgb}{0.000000, 0.000000, 0.000000}
\pgfsetstrokecolor{dialinecolor}
\draw (10.000000\du,3.087500\du)--(10.000000\du,4.937500\du)--(14.800000\du,4.937500\du)--(14.800000\du,3.087500\du)--cycle;
\pgfsetbuttcap
\pgfsetmiterjoin
\pgfsetdash{}{0pt}
\definecolor{dialinecolor}{rgb}{0.000000, 0.000000, 0.000000}
\pgfsetstrokecolor{dialinecolor}
\draw (10.480000\du,3.087500\du)--(10.480000\du,4.937500\du);
\pgfsetbuttcap
\pgfsetmiterjoin
\pgfsetdash{}{0pt}
\definecolor{dialinecolor}{rgb}{0.000000, 0.000000, 0.000000}
\pgfsetstrokecolor{dialinecolor}
\draw (14.320000\du,3.087500\du)--(14.320000\du,4.937500\du);
% setfont left to latex
\definecolor{dialinecolor}{rgb}{0.000000, 0.000000, 0.000000}
\pgfsetstrokecolor{dialinecolor}
\node at (12.400000\du,4.212500\du){Radio};
\pgfsetlinewidth{0.100000\du}
\pgfsetdash{}{0pt}
\pgfsetdash{}{0pt}
\pgfsetbuttcap
\pgfsetmiterjoin
\pgfsetlinewidth{0.100000\du}
\pgfsetbuttcap
\pgfsetmiterjoin
\pgfsetdash{}{0pt}
\definecolor{dialinecolor}{rgb}{1.000000, 1.000000, 1.000000}
\pgfsetfillcolor{dialinecolor}
\fill (19.669400\du,0.637500\du)--(19.669400\du,3.143056\du)--(23.799956\du,3.143056\du)--(23.799956\du,0.637500\du)--cycle;
\definecolor{dialinecolor}{rgb}{0.000000, 0.000000, 1.000000}
\pgfsetstrokecolor{dialinecolor}
\draw (19.669400\du,0.637500\du)--(19.669400\du,3.143056\du)--(23.799956\du,3.143056\du)--(23.799956\du,0.637500\du)--cycle;
\pgfsetbuttcap
\pgfsetmiterjoin
\pgfsetdash{}{0pt}
\definecolor{dialinecolor}{rgb}{0.000000, 0.000000, 1.000000}
\pgfsetstrokecolor{dialinecolor}
\draw (20.082456\du,0.637500\du)--(20.082456\du,3.143056\du);
\pgfsetbuttcap
\pgfsetmiterjoin
\pgfsetdash{}{0pt}
\definecolor{dialinecolor}{rgb}{0.000000, 0.000000, 1.000000}
\pgfsetstrokecolor{dialinecolor}
\draw (19.669400\du,0.888056\du)--(23.799956\du,0.888056\du);
% setfont left to latex
\definecolor{dialinecolor}{rgb}{0.000000, 0.000000, 0.000000}
\pgfsetstrokecolor{dialinecolor}
\node at (21.941206\du,1.815556\du){Packet};
% setfont left to latex
\definecolor{dialinecolor}{rgb}{0.000000, 0.000000, 0.000000}
\pgfsetstrokecolor{dialinecolor}
\node at (21.941206\du,2.615556\du){Buffer};
\pgfsetlinewidth{0.100000\du}
\pgfsetdash{}{0pt}
\pgfsetdash{}{0pt}
\pgfsetmiterjoin
\pgfsetbuttcap
{
\definecolor{dialinecolor}{rgb}{0.000000, 0.000000, 1.000000}
\pgfsetfillcolor{dialinecolor}
% was here!!!
\definecolor{dialinecolor}{rgb}{0.000000, 0.000000, 1.000000}
\pgfsetstrokecolor{dialinecolor}
\pgfpathmoveto{\pgfpoint{20.702100\du}{3.143060\du}}
\pgfpathcurveto{\pgfpoint{20.000000\du}{3.950010\du}}{\pgfpoint{19.900000\du}{5.650010\du}}{\pgfpoint{14.850000\du}{6.285000\du}}
\pgfusepath{stroke}
}
\pgfsetlinewidth{0.100000\du}
\pgfsetdash{}{0pt}
\pgfsetdash{}{0pt}
\pgfsetbuttcap
\pgfsetmiterjoin
\pgfsetlinewidth{0.100000\du}
\pgfsetbuttcap
\pgfsetmiterjoin
\pgfsetdash{}{0pt}
\definecolor{dialinecolor}{rgb}{1.000000, 1.000000, 1.000000}
\pgfsetfillcolor{dialinecolor}
\fill (22.300000\du,11.487500\du)--(22.300000\du,13.487500\du)--(27.200000\du,13.487500\du)--(27.200000\du,11.487500\du)--cycle;
\definecolor{dialinecolor}{rgb}{0.000000, 0.000000, 0.000000}
\pgfsetstrokecolor{dialinecolor}
\draw (22.300000\du,11.487500\du)--(22.300000\du,13.487500\du)--(27.200000\du,13.487500\du)--(27.200000\du,11.487500\du)--cycle;
\pgfsetbuttcap
\pgfsetmiterjoin
\pgfsetdash{}{0pt}
\definecolor{dialinecolor}{rgb}{0.000000, 0.000000, 0.000000}
\pgfsetstrokecolor{dialinecolor}
\draw (22.790000\du,11.487500\du)--(22.790000\du,13.487500\du);
\pgfsetbuttcap
\pgfsetmiterjoin
\pgfsetdash{}{0pt}
\definecolor{dialinecolor}{rgb}{0.000000, 0.000000, 0.000000}
\pgfsetstrokecolor{dialinecolor}
\draw (26.710000\du,11.487500\du)--(26.710000\du,13.487500\du);
% setfont left to latex
\definecolor{dialinecolor}{rgb}{0.000000, 0.000000, 0.000000}
\pgfsetstrokecolor{dialinecolor}
\node at (24.750000\du,12.687500\du){APP};
\pgfsetlinewidth{0.100000\du}
\pgfsetdash{}{0pt}
\pgfsetdash{}{0pt}
\pgfsetbuttcap
{
\definecolor{dialinecolor}{rgb}{0.000000, 0.000000, 0.000000}
\pgfsetfillcolor{dialinecolor}
% was here!!!
\pgfsetarrowsend{stealth}
\definecolor{dialinecolor}{rgb}{0.000000, 0.000000, 0.000000}
\pgfsetstrokecolor{dialinecolor}
\draw (24.750000\du,11.487500\du)--(24.725000\du,9.900000\du);
}
\pgfsetlinewidth{0.100000\du}
\pgfsetdash{}{0pt}
\pgfsetdash{}{0pt}
\pgfsetbuttcap
{
\definecolor{dialinecolor}{rgb}{0.000000, 0.000000, 0.000000}
\pgfsetfillcolor{dialinecolor}
% was here!!!
\pgfsetarrowsend{stealth}
\definecolor{dialinecolor}{rgb}{0.000000, 0.000000, 0.000000}
\pgfsetstrokecolor{dialinecolor}
\draw (12.410000\du,3.037500\du)--(14.600000\du,-0.212490\du);
}
\pgfsetlinewidth{0.100000\du}
\pgfsetdash{}{0pt}
\pgfsetdash{}{0pt}
\pgfsetmiterjoin
\pgfsetbuttcap
{
\definecolor{dialinecolor}{rgb}{1.000000, 0.000000, 0.000000}
\pgfsetfillcolor{dialinecolor}
% was here!!!
\pgfsetarrowsend{stealth}
\definecolor{dialinecolor}{rgb}{1.000000, 0.000000, 0.000000}
\pgfsetstrokecolor{dialinecolor}
\pgfpathmoveto{\pgfpoint{10.000000\du}{5.810000\du}}
\pgfpathcurveto{\pgfpoint{6.850000\du}{6.535010\du}}{\pgfpoint{7.050000\du}{9.750010\du}}{\pgfpoint{10.000000\du}{10.475000\du}}
\pgfusepath{stroke}
}
% setfont left to latex
\definecolor{dialinecolor}{rgb}{1.000000, 0.000000, 0.000000}
\pgfsetstrokecolor{dialinecolor}
\node[anchor=west] at (6.715000\du,5.350010\du){packet};
% setfont left to latex
\definecolor{dialinecolor}{rgb}{1.000000, 0.000000, 0.000000}
\pgfsetstrokecolor{dialinecolor}
\node[anchor=west] at (6.715000\du,6.150010\du){sent};
\pgfsetlinewidth{0.100000\du}
\pgfsetdash{}{0pt}
\pgfsetdash{}{0pt}
\pgfsetmiterjoin
\pgfsetbuttcap
{
\definecolor{dialinecolor}{rgb}{0.000000, 0.000000, 0.000000}
\pgfsetfillcolor{dialinecolor}
% was here!!!
\pgfsetarrowsend{stealth}
\definecolor{dialinecolor}{rgb}{0.000000, 0.000000, 0.000000}
\pgfsetstrokecolor{dialinecolor}
\pgfpathmoveto{\pgfpoint{22.300000\du}{8.950000\du}}
\pgfpathcurveto{\pgfpoint{17.535000\du}{9.095000\du}}{\pgfpoint{21.100000\du}{11.200000\du}}{\pgfpoint{14.950000\du}{10.950000\du}}
\pgfusepath{stroke}
}
% setfont left to latex
\definecolor{dialinecolor}{rgb}{0.000000, 0.000000, 0.000000}
\pgfsetstrokecolor{dialinecolor}
\node[anchor=west] at (20.015000\du,8.000010\du){rime};
% setfont left to latex
\definecolor{dialinecolor}{rgb}{0.000000, 0.000000, 0.000000}
\pgfsetstrokecolor{dialinecolor}
\node[anchor=west] at (20.015000\du,8.800010\du){output};
\pgfsetlinewidth{0.100000\du}
\pgfsetdash{}{0pt}
\pgfsetdash{}{0pt}
\pgfsetmiterjoin
\pgfsetbuttcap
{
\definecolor{dialinecolor}{rgb}{0.000000, 0.000000, 0.000000}
\pgfsetfillcolor{dialinecolor}
% was here!!!
\pgfsetarrowsend{stealth}
\definecolor{dialinecolor}{rgb}{0.000000, 0.000000, 0.000000}
\pgfsetstrokecolor{dialinecolor}
\pgfpathmoveto{\pgfpoint{14.950000\du}{10.475000\du}}
\pgfpathcurveto{\pgfpoint{16.150000\du}{10.500000\du}}{\pgfpoint{16.200000\du}{9.250010\du}}{\pgfpoint{14.900000\du}{9.110000\du}}
\pgfusepath{stroke}
}
\pgfsetlinewidth{0.100000\du}
\pgfsetdash{}{0pt}
\pgfsetdash{}{0pt}
\pgfsetmiterjoin
\pgfsetbuttcap
{
\definecolor{dialinecolor}{rgb}{0.000000, 0.000000, 0.000000}
\pgfsetfillcolor{dialinecolor}
% was here!!!
\pgfsetarrowsend{stealth}
\definecolor{dialinecolor}{rgb}{0.000000, 0.000000, 0.000000}
\pgfsetstrokecolor{dialinecolor}
\pgfpathmoveto{\pgfpoint{14.900000\du}{8.160000\du}}
\pgfpathcurveto{\pgfpoint{16.100000\du}{8.185010\du}}{\pgfpoint{16.150000\du}{6.900010\du}}{\pgfpoint{14.850000\du}{6.760000\du}}
\pgfusepath{stroke}
}
\pgfsetlinewidth{0.100000\du}
\pgfsetdash{}{0pt}
\pgfsetdash{}{0pt}
\pgfsetmiterjoin
\pgfsetbuttcap
{
\definecolor{dialinecolor}{rgb}{0.000000, 0.000000, 0.000000}
\pgfsetfillcolor{dialinecolor}
% was here!!!
\pgfsetarrowsend{stealth}
\definecolor{dialinecolor}{rgb}{0.000000, 0.000000, 0.000000}
\pgfsetstrokecolor{dialinecolor}
\pgfpathmoveto{\pgfpoint{14.850000\du}{5.810000\du}}
\pgfpathcurveto{\pgfpoint{16.050000\du}{5.835010\du}}{\pgfpoint{16.100000\du}{4.615010\du}}{\pgfpoint{14.800000\du}{4.475000\du}}
\pgfusepath{stroke}
}
% setfont left to latex
\definecolor{dialinecolor}{rgb}{0.000000, 0.000000, 0.000000}
\pgfsetstrokecolor{dialinecolor}
\node[anchor=west] at (24.965000\du,11.000000\du){abc send};
% setfont left to latex
\definecolor{dialinecolor}{rgb}{0.000000, 0.000000, 0.000000}
\pgfsetstrokecolor{dialinecolor}
\node[anchor=west] at (15.915000\du,7.000010\du){send};
% setfont left to latex
\definecolor{dialinecolor}{rgb}{0.000000, 0.000000, 0.000000}
\pgfsetstrokecolor{dialinecolor}
\node[anchor=west] at (15.915000\du,7.800010\du){packet};
% setfont left to latex
\definecolor{dialinecolor}{rgb}{0.000000, 0.000000, 0.000000}
\pgfsetstrokecolor{dialinecolor}
\node[anchor=west] at (15.715000\du,4.300010\du){send};
% setfont left to latex
\definecolor{dialinecolor}{rgb}{0.000000, 0.000000, 0.000000}
\pgfsetstrokecolor{dialinecolor}
\node[anchor=west] at (15.715000\du,5.100010\du){packet ?};
\pgfsetlinewidth{0.100000\du}
\pgfsetdash{}{0pt}
\pgfsetdash{}{0pt}
\pgfsetmiterjoin
\pgfsetbuttcap
{
\definecolor{dialinecolor}{rgb}{1.000000, 0.000000, 0.000000}
\pgfsetfillcolor{dialinecolor}
% was here!!!
\pgfsetarrowsend{stealth}
\definecolor{dialinecolor}{rgb}{1.000000, 0.000000, 0.000000}
\pgfsetstrokecolor{dialinecolor}
\pgfpathmoveto{\pgfpoint{14.950000\du}{11.900000\du}}
\pgfpathcurveto{\pgfpoint{20.950000\du}{12.000000\du}}{\pgfpoint{18.750000\du}{9.650000\du}}{\pgfpoint{22.300000\du}{9.425000\du}}
\pgfusepath{stroke}
}
% setfont left to latex
\definecolor{dialinecolor}{rgb}{1.000000, 0.000000, 0.000000}
\pgfsetstrokecolor{dialinecolor}
\node[anchor=west] at (19.315000\du,12.900000\du){recv};
% setfont left to latex
\definecolor{dialinecolor}{rgb}{1.000000, 0.000000, 0.000000}
\pgfsetstrokecolor{dialinecolor}
\node[anchor=west] at (19.315000\du,13.700000\du){Callback};
\definecolor{dialinecolor}{rgb}{1.000000, 1.000000, 1.000000}
\pgfsetfillcolor{dialinecolor}
\fill (22.300000\du,8.000000\du)--(22.300000\du,9.900000\du)--(27.150000\du,9.900000\du)--(27.150000\du,8.000000\du)--cycle;
\pgfsetlinewidth{0.100000\du}
\pgfsetdash{}{0pt}
\pgfsetdash{}{0pt}
\pgfsetmiterjoin
\definecolor{dialinecolor}{rgb}{0.000000, 0.000000, 0.000000}
\pgfsetstrokecolor{dialinecolor}
\draw (22.300000\du,8.000000\du)--(22.300000\du,9.900000\du)--(27.150000\du,9.900000\du)--(27.150000\du,8.000000\du)--cycle;
% setfont left to latex
\definecolor{dialinecolor}{rgb}{0.000000, 0.000000, 0.000000}
\pgfsetstrokecolor{dialinecolor}
\node at (24.725000\du,8.745000\du){RIME};
% setfont left to latex
\definecolor{dialinecolor}{rgb}{0.000000, 0.000000, 0.000000}
\pgfsetstrokecolor{dialinecolor}
\node at (24.725000\du,9.545000\du){abc};
% setfont left to latex
\definecolor{dialinecolor}{rgb}{1.000000, 0.000000, 0.000000}
\pgfsetstrokecolor{dialinecolor}
\node[anchor=west] at (14.865000\du,12.810000\du){abc sent};
\pgfsetlinewidth{0.100000\du}
\pgfsetdash{}{0pt}
\pgfsetdash{}{0pt}
\pgfsetmiterjoin
\pgfsetbuttcap
{
\definecolor{dialinecolor}{rgb}{1.000000, 0.000000, 0.000000}
\pgfsetfillcolor{dialinecolor}
% was here!!!
\pgfsetarrowsend{stealth}
\definecolor{dialinecolor}{rgb}{1.000000, 0.000000, 0.000000}
\pgfsetstrokecolor{dialinecolor}
\pgfpathmoveto{\pgfpoint{22.300000\du}{9.900000\du}}
\pgfpathcurveto{\pgfpoint{20.400000\du}{10.150000\du}}{\pgfpoint{20.400000\du}{12.300000\du}}{\pgfpoint{22.300000\du}{12.487500\du}}
\pgfusepath{stroke}
}
\end{tikzpicture}

\end{frame}
%....... note .................................................................
%\note{ }
%-------------------------------------------------------------------------------
%\begin{frame}{Network und Framer -- layer}{Schnittstelle}
%	\lstinputlisting[language=C,title={core/net/netstack.h (Treiber)}]{mac-xmpl/netstack-network.h}
%	\lstinputlisting[language=C,title={core/net/mac/framer.h (Treiber)}]{mac-xmpl/framer.h}
%\end{frame}
%....... note .................................................................
%\note{ }
%-------------------------------------------------------------------------------
%\begin{frame}{Radio -- layer}{Schnittstelle}
%	\lstinputlisting[language=C,title={core/dev/radio.h (Treiber)}]{mac-xmpl/radio.h}
%\end{frame}
%....... note .................................................................
%\note{ }
%-------------------------------------------------------------------------------
%\begin{frame}{RDC -- layer}{Schnittstelle}
%	\lstinputlisting[language=C,title={core/net/rdc.h (Treiber)}]{mac-xmpl/rdc.h}
%\end{frame}
%....... note .................................................................
%\note{ }
%-------------------------------------------------------------------------------
%\begin{frame}{MAC -- layer}{Schnittstelle}
%	\lstinputlisting[language=C,title={core/net/mac.h (Treiber)}]{mac-xmpl/mac.h}
%\end{frame}
%....... note .................................................................
%\note{ }
%-------------------------------------------------------------------------------

